Outline:  60-120 Lines

Question:  Was the logistics of the British Army fit for purpose in supporting
the British Army in operations in NW Europe from 1944-1945.  

/*****************************************************************************/
Outline as Follows:

No indentation is a top level <h2> type heading / Latex \section{} unit.  
Each successive tab indentation reduces the heading by one level.  Thus, a 
single tab denotes <h3> / \subsection{} whereas two tabs denote 
<h4> / \subsubsection{}.  Constrained to no more than three tabs/levels of 
sectioning.

Comments will use either C style comments for embedded remarks (/* comment */) 
or Latex comments where the whole line after `%` will be ignored on output.

Outline begins on line 25 and will continue till line 85-145.

Output target length 50-70 pages. 

Note:  POL stands for petrol, oil, & lubricants.  
/***************************************************************************/

% Conops, perhaps use a narrative flow moving from before invasion for the
% prep work and training, to the actual invasion for execution and adaptation,
% to transition around or after TOTALIZE?

% TLDR:  if you accept that the British CA soldier was tactically worse than
% the Germans but were able to defeat the Germans by unloading an ungodly
% amount of steel and explosives over their head at the first sign of 
% difficulty, then the British logistical soldier was far superior.  A war 
% cannot be won by killing alone!

% Thesis: 

% L1:  Historical community is quite hard on the British for lacklustre
% results vis-a-vis armoured warfare and armour-infantry co-operation; 
% however, what this is missing is that the British Army --- Army, not 
% economy --- work quite well at a logistical level.  This is what, in the
% field, enabled the British Army to defeat the Germans despite tactical
% mediocrity.  Logistics worked to support flawed tactics.

% L2:  Criticality of the services to the effective waging of war.

% L3:  We ought to stop viewing armies as mere fighting forces.  Understanding
% them using a systems approach explains why we can win wars.  Merely 
% examining how an army can outfight its opponent at a tactical level misses
% the operational reasons the tactical level can even function.


% CONOPS:  What if I do this as a microhistory of 90 Coy?  I can use them as 
% a way to talk about the criticality of logistics.  What if I used them as
% the pivot around which an army (tbh, 27 Armd Bde, and 3 Div) can turn?  I 
% can use it to skirt around the wider issue that I don't have evidence for
% a macro view but I have 90 Coy's WD.  Alas, I can't find as many decorations
% for them as I would like to really make it personal.  Could I use examples 
% from other units as a `take Cpl Bloggins from X, note the work he did'?
% time to run git branch I suppose.  It also allows me to be narrative which
% is always fun.  Their 4 day history was already 7 pages, I could breath life
% into.  I can harp on about them for 8-10 x that, right? 1-2 pages 

% introducing (let's do it as if they hit a beach and they're driving a 
% convoy up the road to Benouville with that preload of supplies for 6 Airborne
% whence we pause to discuss what we currently talk about in the 
% historiography.   I have 5-10 pages on what other people have been writing 
% about Normandy, from there, perhaps another 5ish to narrow down on Sword 
% Beach, the whys, the objectives, who was involved etc. --- I should re-read
% return of martin guerre I think.  

	% so 20-30pgs here, I'll probably do more once I footnote it all.
% With this context, we return to the convoy and lay out the theoretical 
% framework by which logistics operates.  Depots, lengths of supply lines,
% how distribution happens, etc.  How this work integrates with the rest of
% the army.  We deliver the supplies and talk about the RV.  Let's also start
% talking about the arduous work over the next few days to supply 3 Div and 
% 27th Bde as there are no other 2nd line transport units.  As we continue 
% here, I can intersect the sources for 27th Armoured Bde to give context
% as to what 90 Coy was doing.  I suppose I"ll have to do some probables and
% perhapses.  We can talk about how 90 Coy supported the Bde until the Bde was
% disbanded.  Then, continue talking about how the Coy supported other units
% till Totalize maybe?  Then, have a final almost pre-conclusion discussing
		% Perhaps this takes 4-6 pages-ish?
% how, whilst we don't write about it, these operations would have been 
% impossible without the supporting arms and concluding with how the British
% ability to have tactical mediocrity and an unwillingness to spend lives
% meant that their logistics had to be good, really good.  

% Expand this to how armies function, and who this paper doesn't cover.  I 
% don't discuss workshops, clerks --- the whole of A branch actually --- MPs, 
% BADs, beach detachments, etc. yet they're all important. (5 pgs)  Maybe 
% paint a picture of the whole rear area and why it's important (landing 
% tickets maybe?).
% Tie it back to historiography and methods like  how a lot in the sources 
% requires you to intersect doctrine with the WDs to figure out what supply 
% units were doing --- making a meal dely isn't something that really gets 
% recorded in the WD, etc.  (5 pgs)

% Finally, conclude.


\section{Introduction}

	%Estimate 2-3 pgs?

	% Set scene, we're in Normandy, 6 [time] Jun 44, we just landed, we're
	% driving from Queen Beach to Benouville to link up with 6 Abrn.  
	% you're X lorries are carrying preloads of ammunition, rations, and
	% other supplies for 6 Abrn.  They're holding the Anglo-American 
	% left flank from a possible German counter attack.  what if I put
	% it in the perspective of that Lt who did the initial recce?  You 
	% weren't originally scheduled to land yet but an accident of war
	% means here you are.  Describe more about what 6 Airborne is doing
	% you don't belong to 6 Airborne but are instead of 27 Bde, you're 
	% just tasked to help them on D day, their unit 2nd line unit will
	% arrive from Juno later /* Come back to this when discussing import */

	% THESIS
	% You know what you're doing is important, without these supplies and 
	% the war will ground to a halt.  

	% You wonder what the historians who write about these events 80 years 
	% in the future will think about it all, what drives your focus?

	% Maybe have a bitter reflection, it's always the fighting troops or
	% the generals who get the cheers.  No-one applauds the cooks!  Use
	% the perhapses of history for this

	% Shift to that historiographical frame:  With the benefit of 
	% hindsight, we know that the drive to Caen would not be a quick 
	% drive, but as a logistician, one asks oneself, was there a 
	% significant logistical constraint or was it due to something
	% else?

	% Segue to historiographical review?

	\subsection{Clarification of Terms}
		% How I use D-Day to actually mean D Day and not 6 Jun 44

\section{Historiographical Review}

	% `Of course we know that 80 years after the fact, historians...'

	% Complain loudly and long-windedly at the lack of discussion on log 
	% in /*insert list */  Consider subsections or flow of text?

	% Discuss the current logistical scholarship done
		% Namely historical work in 50s and Supplying War, Julian
		% Thompson, Great Feat...

	% Introduce Section

	% General observations:
		% Much work done on great men, tactics, the merits of German
		% Armour; and some work has been done on American logistics,
		% as well as logistical peculiarities like the mulberry 
		% harbours.  I fear little has been done on the actual
		% military administration of the war.  Indeed, the logistical
		% section of military history is fairly poorly written about
		% by historians, some more work done military academies
		% reflecting its importance to them
	\subsection{On WW2} % there's got to be a better name than this
		\subsubsection{\textit{Britain's Other Army:  The Story of
			the ATS}}
			% This feels kinda awkward to put here
		\subsubsection{\textit{Why the Allies Won}}

	\subsection{On Normandy}
% reorder this, chrono flow or based on argument?

% really flesh out the common critiques and defences of the British Army, 
% that it was slow to develop, under manned, unimaginative, morale problems,
% materiel over lives

		\subsubsection{\textit{Clash of Arms}}
			% Argues that the British were slow and failed in
			% innovating.  
		\subsubsection{\textit{Overlord}}
		\subsubsection{\textit{Fields of Fire:  Canadians in 
			Normandy}}

			
		\subsubsection{\textit{Montgomery and `Colossal Cracks':  
			The 21st Army Group in Northwest Europe, 1944-45}}
			% manpower Constraints
			% Morale problems Be sure to emphasize both of these
			% as supplies are critical to this
		\subsubsection{\textit{The Normandy Campaign 1944}}
		\subsubsection{\textit{Gators of Neptune: Naval Amphibious
			Planning for the Normandy Invasions}}
		\subsubsection{\textit{Neptune:  the Allied Invasion of 
			Europe and the D-Day Landings}}
		\subsubsection{\textit{From the Normandy Beaches to the 
			Baltic Sea: The North West Europe Campaign
			1944-1945}}
		\subsubsection{\textit{Feeding Mars:  The Role of Logistics
			in the German Defeat in Normandy, 1944}}
			% Toss this in Normandy or Log?

	\subsection{On Logistics}
% reorder this

%% Do I wanna put the books for WW2 logistics here or in WW2?  I'm tempted
% to concentrate it here but I also like the idea of keeping something from
% the field of military science and not history separate --- military science
% cares much more on actually executing operations.  Split this into two 
% subsubs and make the rest subsubsubs?  One for history, one for less so?
% Could also compress, maybe compressing and not sectioning this section 
% will flow better.  In any case, I wonder if it's better to chk pg ct

		\subsubsection{\textit{Supplying War:  Logistics from
			Wallenstein to Patton}}
			% Foundational in logistics scholarship, but, given
			% it's broad scope, it lacks depth
		\subsubsection{\textit{The Lifeblood of War: Logistics in
			Armed Conflict}}
			% Limited and tries to cover a lot of periods
		\subsubsection{\textit{A Great Feat of Improvisation}}
			% British but only really to shortly after Dunkirk

			% Use it to talk about how supply developed during
			% interwar years, namely, motorization.

		\subsubsection{\textit{War of Supply:  World War II Allied
			Logistics in the Mediterranean}}
			% Chiefly American in the Med
		\subsubsection{\textit{Supplying the Troops:  General 
			Somervell and American Logistics in WWII}}
			% Very much a great man history.  Logistics in the
			% form of a biography
		\subsubsection{\textit{Military Logistics and Strategic 
			Performance}}
		\subsubsection{\textit{The Story of the Royal Army Service
			Corps}}
		\subsubsection{\textit{Logistics and Modern War}}
		\subsubsection{\textit{Logistics Diplomacy at Casablanca: 
			The Anglo-American Failure to Integrate Shipping and
			Military Strategy}}
		\subsubsection{\textit{Strategy and Logistics:  Allied
			Allocation of Assault Shipping in the Second World
			War}}
		\subsubsection{\textit{The Science of the Soldier's Food}}
		\subsubsection{\textit{D Day to VE Day with the RASC}}
	
		

		

	\subsection{Tools of the Trade} % maybe rename this

	% The historiographical Gap.  Introduce an inattentiveness to log
	% here, that the materiel advantage is contingent on getting this
	% right.
	
	\subsection{A Note on My Sources}
		
		% Heavy reliance on digital records

		% paper records from Canadian sources owing to funding but
		% argue for the soundness of the methodology anyways given
		% how Canadian stuff is of Br doctrine

% Start with Context %%%%%%%%%%%%%%%%%%%%%%%%%%%%%%%%%%%%%%%%%%%%%%%%%%%%%%%%%%

\section{WTF are we doing here!?} % rename this

	% We return to our convoy, you're thinking and a bit day-dreamy 
	% driving through the French countryside thinking about the 
	% magnitude of what you're actually doing.  DO I want to go this
	% tone??

	% Overlord & Neptune:  Mission, invade France, push the Germans
	% across the Rhine sooner or later

	% Division of beaches, start from the W and move E.  How beaches
	% are subdivided.  Start thinking about the beach exits maybe and
	% why they matter?  Maybe reflect on the massive traffic jams?

	% Talk about the 'bloody army', Q branch, A branch, and G branch.

\section{} % move to Hermanville, 90 Coy's control point in running convoys

	% Talk about how supply works, the DIDs, supply lines, POL differences

	% complaint about the flimsies and containerized fuel

	% talk about how you sustain client units, march length, the end goal
	% of ending the day with filled tanks, full ammo, etc.

	% Keep that story telling feeling, you're at the control point mourning
	% the fact that there are no other second line units, you're vastly 
	% overworked.  If only you had the additional units to support you.
	% if only ____ instead, you're left trying to maintains [sustainment
	% standards] and just keep what's needed flowing as hard as you can.

	{Logistics Working Practices in Theory}

	% Use to establish what we're actually dealing with so importance
	% becomes self-evident
	
	% How supply chains work from dumps, and depots, to distribution, to
	% 1st line units.  

	% ``The principle of supply is that field units should always have
	% with them, or within reach, two days' rations and forage, and one 
	% iron ration, and that these stocks should be replenished by 
	% delivery, at a point within reach of the troops, of one day's ration
	% and forage each day.'' %cite No27 Trg Course p. 32, quoted in,
	% refs F.S.R. Vol. I. Sec107(I). %validate this

	% Key was placed on  on rabild concentration of force, incl sup, LofC
	% etc. to permit strategic use of force %cite 27 trg crs p. 32

	\subsection{The Structure of Supply}
		
		% 4 main areas, the BSA, L of C, Corps/GHQ area, Div area
		% goes from 

		%cite No 27 Trg Course p. 34ish (diagram's on 34)
		\subsubsection{The Base Supply-Area (BSA)}
			% The Docks
				% Bulk Petrol
				% Cold Storage

			% Main Supply Depot --- Field Bakery, DID

			% Petrol Sub Depot 

			% Base Marshalling Yard (start of railway line)

		\subsubsection{L of C Area}

			% Envisioned as a railway line, lorries will do 
			% just fine but this theory came at the dawn of
			% mechanization.  Point was to take stores from
			% the BSA and transport them near the combat zone.
			% It's basically a transport area.  IT goes from 
			% just outside the BSA to, and including the railhead.

			% If distance from the BSA to the railhead is < 12
			% hr trip, BSA will control.  Else, there will be
			% a regulating station around 6 hrs from the rail
			% head. %cite no27 Winter Trg p. 32 5c
			
			% The railheads also had field supply depots, to
			% supply local resources.
			
			% If using trucks, it's 150 tons per 21 trucks with
			% old pat, 59 tonnes across 12 trucks (5 supplies, 2 
			% ord, 2 R E, 2 stores, 1 RAMC)  railway trucks 
			% %cite No 27 course, p. 32 para 5

		\subsubsection{Corps or GHQ Area}

			% Start of what really starts to feel like the
			% field army and not just transport areas.
			% Transport at this point is by road.

			% Separate Chain for 'stuff' and POL.  POL is the
			% Corps Petrol Park (25 Mi for Div, 75 mi for C.T)
			% Explain the logic of this.  POL used containerized
			% distribution rather than tankers for end-point
			% distribution.  Used flimsies or jerrycans

			% Supply Coln. HQ handles the rest of it.  Here, 
			% stores are bulk broken and sent further forward.

		\subsubsection{Div Area}
			
			% POL side, POL is held at the Div Petrol Coy 
			% and should have a 50 mi. supply (chk this).  
			% The div Pet Coy will set up petrol points and
			% unit transport will collect POL from those points.

			% General Supplies are handled at a rendezvous 
			% where supplies will be delivered at delivery 
			% points via meeting points.  Range from the L of C
			% area to the DPs is 30-40 mi.  DPs are where 
			% individual units come to collect stores.
	

	\subsection{Warehousing}

		% Especially in those early weeks when a lot of it was dumping
		% rather than warehousing per-se.  

		% DID (maybe??) 3000 Tons/acre= gross stacking area.  x 4-10
		% for expansion %cite No 27 trg 32 (this might be a general
		% rule too)

		\subsubsection{Base Supply Depots}
			% Located near Base Marshalling Yard, outside docks
			% to permit expansion and dispersion.  Consider `water
			% light, telephones, good roads, office accommodation'
			% when selecting the location. Area = Strength X 
			% stock X weight one ration / (unintelligible) = 
			% 8.5 sq ft /ton = stacking 
			% +50\% for stock spacing = gross stacking area X 4
			% to 10 for expansion, etc.  (that's a quote)
			%cite 27 trg course p.32, 4
	% zoom out a little and talk about what you'll be controlling.
	

/* Figure out where to incorporate the fact that the British/Canadians focused
on firepower over manpower.  This means materiel is critical --- A is for ammo,
B is for beans, C cold water, D: diesel, E-everything else... */

/* Do I want to expand to include things like traffic control?  Traffic jams
on Sword Beach may have made the Br fail to capture Caen on D.  10m of dry
beach between water and sea wall at high tide.  Perhaps an MP or two would
have solved the issue. IIRC, RAF beach sqn dealt with it. (See
RAF beach sqn/det  Was this a critical
oversight?  Not a lack of tenacity or anything else, but a good, old fashioned
traffic jam VI's-a-vis Toronto at rush-hour caused the failure to take Caen?
lol --- what a way to win a war! */

\section{Return to the moment} 
%New title, want to get back to what 90 Coy was up to till D+7
% sum up the Jun Hist Rep for 90 Coy 

\section{Operation } %name, if any?  Was hasty operation.  
% Maybe just consolidations?  Oh, how about the Dumps at Ranville

% give context for the situation IVO Ranville at the time.  British holding
% several KM^2 E of the Orne at IVO Ranville.  This land was supplied from main
% beaches over the Benouville-Ranville bridges.  Single point of failure.  
% fear of DE cntr attack as DE tps moving IVO E of region probing for points of
% failure.  27th Armd Bde involved in supporting local infantry.  90 Coy
% supporting Bde

% Operation with Paras of 13/18th with paras, 90 Coy dumping stores in support
	% 12 June, mail arrives, first letters since D Day

% 13/18 moved E of river to defend against counter attack %cite 13/18 WD 12 Jun
% A, B Sqns supporting attack of German positions by 7 Para & black watch.  
% on 10 Jun, Germans were seating themselves into a wood IVO Le Marquette, 
% threatened to separate 5 & 3 para Bde as well as supporting infantry units.  
% Paras ordered to sweep woods to dislodge Germans with 13/18th in support.
% part of wider fears of a German counter attack.  
%cite 7 para jun WD appendix p. 5
% 90 Coy established a dump using 12x 3tonners.  Mainly Amn & POL, started
% 12 1800 Jun 44.  Increases to 20x lorries next day (how many lorries in 
% a coy again?)
% dump at Ranville (1173) %cite 90 coy WD 12 Jun 
% Do I wanna mention decorations here for 90 Coy or maybe do it earlier
% IVO Ranville.  Positioning of this dump wise.  Only route to supply troops
% E of Orne and canal is via Benouville/Ranville bridges.  This would permit
% bridgehead to keep fighting if the bridges were destroyed.  Infantry armour
% co-operation was a noted problem.  6 tks KO.  Paras take 40 prisoners, 0 KIA,
% 9 Wounded. 
%cite 7 para jun WD appendix p. 6


% Highlight that preparation is key.  If there was a counter attack and the
% bridge was blown, gallantry matters naught if you aren't supplied.  What 
% will you do, fix bayonets --- well, yes actually...?
% C Pl 90 Coy continue to provide light support till 16 Jun

\section{The Arrivals of A \& B Plns}
% non critical section, setting the tone of a lull.  A/B Pl arrive 
% 15 2000 Jun 44.  Go to Coy HQ in Cresserons GR 0379.  Brings with them 
% 59 Veh, 165 Pers.  Arrives 2300 hrs. Some DE bombing.  Over the next 3 days,
% new platoons involved in dewatering and reorganizing in prep for ops.  
% sporadic bombing and shelling of veh pk.  Veh dispersed 75 yds between 
% each veh --- quite exposed with no protection then (they're in an open 
% field).  

% Queen Beach still being shelled.  (appears both in 90 Coy and 13/18WD)
% buildup thus slowed.  13/18th don't get their D+9-10 residues by 18 June
% This is a failure in supply.  Delays due to weather.

% Generally fairly quiet for Bde till end of month %cite 27 Armd Bde WD
% quiet for 90 Coy until 23 Jun %cite WD

% Maybe talk about the perhapses?  Amn still was consumed, as was POL, in 
% addition, what about general transport of rations, etc.  I don't have
% a src, but these must have happened.

% Talk about that wonderfully dull stuff like stove and flour shortages in
% the 27th Bde WD Admin instructions (and defecate in the latrines!).  
% Also, traffic flow patterns, spare parts, uniforms, water points etc.
% Highlight the importance of the ordinary and mundane. 

% Talk about how a POL lorry catches fire in the 13/18 rgmt area on 20 Jun.  
% Spreads to Amn dump.  %cite 13/18WD
% Not in 90 Coy's WD.  Why not, routine fetch and carry.  Fire happens, 
% amn is replaced, maybe only a few lorry loads, no bother.  Talk about
% the need to look between the sources.  An armoured rgmt is unlikely to
% have the integral transport to replenish that dump and frankly, it's not
% the job of 1st line units to maintain a 2nd line dump.  If you just read
% the WD, you would assume they're twiddling their thumbs most days.  This
% is unimaginable!

\section{Operation Mitten 27--28 June 1944}

% Want to show that operations impossible without such units.  Focus on what
% worked, not what didn't work for the British.  British tanks bad, arty good.
% this is a useful moment to talk about the nature of Br Arty.

% Objective (no idea TBH, check 27th Bde WD) TLDR form google search,
% eliminate a DE salient at a chateaux not on my map IVO GR 0372

% 90 Coy's Work

% 30x 3Ton loads of 105mm amn held on wheels for 3 Br div on 23 Jun.  
% Likely for M7 Priest.  Should clarify, this isn't 90 tons but 30 
% truckloads. wonder if I can find out how many 105mm shells a 3 tonner can 
% load onto it.  Does it volume out or weight out?  Emphasize that 90 Coy
% was only helping this unit.  Technically, as the 27th Bde unit, they 
% don't do arty stuff.

% Amn is delivered 26 Jun to batteries located IVO GR 0378, it's D-1.

% don't forget about the stuff for flamethrowers!!

	\subsection{British Artillery} % amn consumption in barrage, arty
		% usage, etc.  Show just how dependant the Br were with
		% arty.  Maybe get the mass of a 105 shell and propellant?
		% there's no way to get an accurate estimate though, if
		% only they were 25 pdrs...

	\subsection{Support to Operations}
		% Operation used flamethrower tanks from 141 RAC (Churchill
		% Crocodiles)  Needed special fuel and nitrogen cylinders.
		
		% Coy sets up dump of 3000 Gal (13640L) of fuel and 90 N cyl
		% at Gazelle (GR 0276) on 27th, withdrawn on 29th.

		% D+1, Maint Point set up at Le Vey (GR976757) distributing
		% 2400 Gal of FTF.  

		% D+1 for Staffs Yeo, 0300, 1082 rds 75mm HE delivered, 
		% 600 rds at 1600 hrs, 1200 rds at 2000 hrs delivered --- what
		% where the Staffs doing?  Find out  Also for context, give
		% amn capacity of tank  

		% in Op Inst No 2 for 27th Bde (28 Jun 44) sqn shoots limited
		% to 50 rds/gun.  Tanks were to harass Germans as long as 
		% Br infantry weren't harmed by this fire and the tanks move
		% as soon as counter battery fire is encountered
		%cite 27th Armd Bde WD, Jun 44 p12
		% This tells us there's no great fear of running out of 
		% ammo.  If you could fire off half your ammo in sqn shoots
		% and still be operationally ready for further operations or
		% movement, you're confident you could be replenished.
		

	\subsection{Figure out a name} %rename
		% conclude this section maybe this doesn't actually need a 
		% subsection.  This much work was needed for a simple 2 day
		% operation.  Work vital but it was doable.  Sum it back up.
		% Maybe the tanks weren't the best, but their inadequacies were
		% floated by a supply chain that kept up so that they could be
		% bad but still effective.

\section{Operation Aberlour}
% Never took place, do I wanna talk about it?  Issued 27 Jun 44.  A lot of 
% admin stuff for 90 coy like amn dumps, etc.  (Cite p26 27th WD for Jun)
% more on p30.  I almost like the idea of talking about an op that never
% takes place.  It eliminates the fog that arises from what happens when
% there's contact made with emy.  Here we can focus on how operations
% take place.  Hmm...

\section{} % Run up to Op SHERWOOD

	% extensive minefields to be laid (do sq footage) 
	% (See pg 13 of 27th Bde WD)

\section{The Lead up to Charnwood}

% Use this section to warm up the reader as to how logistics works
% in operations.

% Context for Charnwood

% D = 8 Jul

% Will continue using the Gazelle ammo dump until exhausted (smart, why move 
%it)  This is the only initially authorized dump. 

% 90 Coy sets up POL pt just S of Hermanville 2100-2400 8 Jul (likely units 
% moving through, north to south, fuelling as they pass)
%cite 90 WD 8 July

% AP set up NW of Cresserons at Coy HQ-ish area %cite 90 WD 8 July
% Rgmts draw ammo next morning, ERY withdrawn 1400 when Caen's captured

% 90 Coy is carrying 1000 Gal FTF, 35 N bottles on wheels for crocs at
% Cresserons

% 3 days compo rats issued, additional 3 days AFV packs carried in tks if
% compo not avail.

% Blankets to be distr as situation permits


% tracked traffic to use tracks rather than roads --- degradation
%cite 27th Jul orders 34-6  Sandbags used to store casualty's kit

\section{Pre Goodwood}
% 27th HQ moves to Douvres 10 Jul

% CONTEXT
% Post Charnwood, Bde is withdrawn on 24 hrs notice on 10th, increased to
% 48 hrs notice on 13th.  Men get some rest but in the background a number
% of conferences occur in preparation for Goodwood.  Vehicles are likely
% being maintained and losses replenished in this time.  2nd Army's being
% reorganized and a CO's conference on battle lessons occurs on the 13th.
%cite 27th WD


% talk about how this is often seen as an interlude but life continues.
% first bread ration issued on 11 Jul from 35 Fd Bakery at Luc sur mer.
% 2 oz issued / man increased to 4 oz next day --- men would have eaten 
% largely tinned food for a month.  Fresh food likely quite welcome.
%cite 90 coy 12 Jul

% Moreover, important work continues.

% Note the shortage of messtins and KFS at hospitals and the chronic shortage
% of tires.  Requirement to return amn casings, empty ammo cans, etc.
% 27 Bde Jul orders p37

% note how issuance of rations, units were overdrawing, and fresh rats were
% scarce.  Officers sent round to audit.  Talk about how this isn't just about
% REMF harassing front line troops, but a necessary part of military admin.
%cite 27 Bde Jul Os p38

% TRANSITION

\section{Goodwood (18-20 Jul 44)}
% ARGUMENT:  Use Goodwood to really start to show the centrality of logistics
% to ops.  No food, no amn, no POL, no fighting. (Goodwood detailed)
% Argue that trucks, not just tanks are what it means for an army to be mobile.

% STRUCTURE:  go day by day in the Run up and keep very chrono-narrative.  
% each day seems to sort nicely into themes; thus, use those themes and
% talk about them

% Point of operation  How do i want to manage this transition?  This is
% too abrupt

% THEME FOR 14TH:  Battle sustainment
% WngO (formal or informal) likely received for Goodwood 14th late morning or
% early afternoon.  90 Coy spends the afternoon organizing, then
% 14 Jul, over five hours (1700-2200), 90 Coy 
% converts the 13/18 rgmt amn & POL dumps IVO Ranville into dumps fit to 
% supply the Bde (likely approx 2 sq km in size).  
% dump consists of 38 lorry loads of amn to the 13/18th Rgmt dump enough
% to provide 97 rds/tk (a full load of amn) for the whole Bde.
% POL:  18 lorry loads of Pet & Derv (diesel) (7000 (either units or gal) of
% each).  This is enough POL to take the Bde 30 mi.  Coy does this via the BAD
% and PD.  Vehs fuelled as well before return to Coy HQ by 2400.  Speculate this
% was in receipt of WngO On 15th at 1900, Coy formally 
% establishes a det at the Ranville dumps to control R&I.  Det consists of 
% Capt Duffus, 46 ORs and 16 lorries.  Useful time to talk about fighting
% range and what it means.
%cite 90 Coy WD

% By the 15th, preparations for Goodwood occur in earnest across the Bde.  
% 3 Div holds another conference and, over the next few days, the Bde starts
% moving E of Orne likely via Ranville.  %cite 27th WD 15th Jul-ish
% Ranville det loads load 4 days compo into 6 lorries
% (enough to sustain whole Bde) as a mobile reserve.  8 lorries are devoted to
% keeping croc flamethrowers amned.  2 lorries leftover for odds and ends.
% Useful time to talk about importance of flexibility maybe?  
%cite 90 coy WD

% 16th:  RATIONS
% Bde HQ moves closer to Ranville but stays W of Orne.  Bde Cmdr holds 
% conference with Bde COs. %cite 27th WD
% As a result of this, all units W of Orne received 3 days extra compo, E of
% Orne, 4 days. %cite 90 WD
% Likely, whilst not stated in the WD, that 90 Coy made these regular 
% deliveries to the battalions.

% Total rations thus 4 days in vehicles, and 4 at 2nd line transport.  Ready 
% to move for 8 days.  Therefore, maximum planned speed for brigade's advance
% is approximately 30 mi (let's be realistic, likely 20 mi to account for 
% general movements) over 8 days by accounting for POL.  A good day of 
% advance would be 3.75 mi (6km)/day.  


% 17th:  Comms
% 27th HQ issues it's OpO for Goodwood today.  The Bde will help hold 8 Corps'
% left flank as 8 Corps breaks through.  Bde establishes it's battle net today.
%cite 27th WD and Orders
% 90 Coy receives two wireless lorries, one at Coy HQ, and 1 at the Ranville
% det.  Each lorry comes with 3 wireless ops.  %cite 90th WD
% Curiously, the OpO hardly mentions logistics.  Log is just expected to work.

% 18th:  D-day H-hr 0745
% Bde attack successful.  Objectives met around 1100.  At 2000 Coy HQ sends
% 3 lorry loads of PET and 3 of DERV (figure out range from 
% 18 loads = 30 mi) to Det Ranville. They're stranded, traffic is heavy,
% only E bound traffic permitted over the bridges.  %cite 27/90 WDs
% Start talking about what sustainment looks like.  If there are 6 lorries
% prepared, surely they'll be needed.

% 19th:  DE send 8 JU88 to attack Orne bridges.  Bde makes minor gains.
% ditto 20th.  Little change happens at this point as the front stabilizes.
%cite 27th WD
% 90 Coy uses this as a time to resupply units from Det Ranville.  
% talk about how ammunition moves.  The Ranville dump goes from X lorry 
% loads to 45 loads of amn when the dump closes and the amn is brought back
% to BAD Hermanville. %cite 90th WD for 25 Jul

% If Goodwood failed to break out, it wasn't logistical constraints.
\section{Post Goodwood}

% use it as a chance to talk about infantry transport --- Br infantry don't 
% have organic transport.  Also how more normal RASC transport units work

% 22nd, the Bde is informed they'll be broken up.  They get assigned to 
% 2nd Army's 22 Transport Coln at months end.  
% Det Ranville shrinks by 45 lorry loads of amn
% and the coy moves to Camilly (GR 933704).  Det Ranville is handed over to 
% different unit and rejoins the Coy.  Coy provides transport to Staff Yeo
% to Arromaches docks.  Coy returns the 2nd line amn holding of 30x 3ton 
% truckloads and 27 6-ton loads to BAD.  %cite 90 WD p.7 mention this holding
% earlier on too.

\section{} %not sure what to call it, something about being an ordinary
% 2nd line RASC unit


\section{Criticality of Supply}

% Ask the question of what would happen if not for these activities. 
	% Amn
	% Beans
	% Water (note location of WP)
	% POL
	% Everything else (white paint for turrets lest one gets shot lol)

% Invite community to see an army as a system and merely disconnected fighting
% groups.

% Discuss my gaps, I barely touch on maintenance units, etc.  What I could
% have talked about, what other sources say.  Consider Tiger or Panther without
% Maint units

% talk about sources, how 90 Coy WD doesn't really mention resupplies unless
% it's major, but just because it's not written, doesn't mean it wasn't done.
% by asking the question of who would do X, one starts to uncover the Y, and Z.
% talk about how entries like 6 Jun are very long --- likely because of the
% initial excitement.  The WD's progressively shorter and more concise as time
% wares on, likely because it's tedious and, "I want to go to bed!".  Supply
% logs are tedious.  Talk about how this study is actually quite constrained
% by sources in so far as a heavy dependence on electronic sources.  War diaries
% don't mention routine replenishments but it doesn't mean they don't happen.
% alas, I doubt they kept the waybills!

%%%%%%%%%%%%%%%%%%%%%%%%%%%%%%%%%%%%%%%%%%%%%%%%%%%%%%%%%%%%%%%%%%%%%%%%%%%%%%%
% Stop Work Point 22 1846 Jul 25
%
% Preparing to talk about what happens to the company after the 27th is 
% disbanded.  Starting work on August diary.  Talk till end of Aug to show
% other side of log.
%
% Flesh out \section{Criticality of Supply}
%
% Working note, fleshing out the analysis more
%
% Should flesh out the explanation of sup as sys in earlier section
%
% Wondering what to do about the sectioning where I talk about the structure
% of supply. 
% 
% Don't forget to email prof sooner rather than later!
%%%%%%%%%%%%%%%%%%%%%%%%%%%%%%%%%%%%%%%%%%%%%%%%%%%%%%%%%%%%%%%%%%%%%%%%%%%%%%%

\section{Conclusion}

/******************************************************************************
END OF OUTLINE
******************************************************************************/

/* Header as fol *************************************************************/

\documentclass[noraggedright]{turabian-researchpaper}

\title{}  
\subtitle{}

\date{\today} % alter?
\author{Albert Duan}

% Sets UK Date format
\usepackage[british]{babel}

% Auto-punctuation outside of quotes
\usepackage{csquotes} 

% Auto link URLs, hyphenate long URLs in appearance.
\PassOptionsToPackage{hyphens}{url}\usepackage{hyperref}

% Find pkg to wrap long URLs

% Sets Bibliographic Style 
\usepackage[notes]{biblatex-chicago}
\addbibresource{} % add path

/* Library includes */

% Formatting notes, do I want to divide this into multiple source code files?
