\documentclass[noraggedright]{turabian-researchpaper}

\title{}  
\subtitle{}

\date{\today} % alter?
\author{}

% Sets UK Date format
\usepackage[british]{babel}

% Auto-punctuation outside of quotes
\usepackage{csquotes} 

% Auto link URLs, hyphenate long URLs in appearance.
\PassOptionsToPackage{hyphens}{url}\usepackage{hyperref}

% Find pkg to wrap long URLs

% Sets Bibliographic Style 
\usepackage[notes]{biblatex-chicago}
\addbibresource{src/sec.bib} 
\addbibresource{src/pri-uk.bib}
\addbibresource{src/pri-can.bib}

%%%%%%%%%%%%%%%%%%%%%%%%%%%%%%%%%%%%%%%%%%%%%%%%%%%%%%%%%%%%%%%%%%%%%%%%%%%%%%%
% Citation Aids
% All citation aids will start with capital letters for sake of namespaces

% For citing 27 course's Lecture No. 12 Petrol p 35-7.  Largely on bulk vs 
% container
\newcommand{\Petrol}{Precis of Lecture No. 12:  Petrol}
%cite Author is Maj W. P. Pessell RASC.  Figure out how to do it when LAC 
% website works.  It's on p 35 of my scan
% For use with {27course}

% For Supplies in War Lecture pt 2 by Lt.Col. P. L. SPAFFORD, 0.B.E.
% For use with {27course}
\newcommand{\SupInWar}{Precis on Lecture ``Supplies in War'', (Part II)}
%%%%%%%%%%%%%%%%%%%%%%%%%%%%%%%%%%%%%%%%%%%%%%%%%%%%%%%%%%%%%%%%%%%%%%%%%%%%%%%

% Formatting notes, do I want to divide this into multiple source code files?
% Outline:  60-120 Lines

% Question:  Was the logistics of the British Army fit for purpose in supporting
% the British Army in operations in NW Europe from 1944-1945.  

% /*****************************************************************************/
% Outline as Follows:

% No indentation is a top level <h2> type heading / Latex \section{} unit.  
% Each successive tab indentation reduces the heading by one level.  Thus, a 
% single tab denotes <h3> / \subsection{} whereas two tabs denote 
% <h4> / \subsubsection{}.  Constrained to no more than three tabs/levels of 
% sectioning.

% Comments will use either C style comments for embedded remarks (/* comment */) 
% or Latex comments where the whole line after `%` will be ignored on output.

% Outline begins on line 25 and will continue till line 85-145.

% Output target length 50-70 pages. 

% Note:  POL stands for petrol, oil, & lubricants.  
% /***************************************************************************/

%%%%%%%%%%%%%%%%%%%%%%%%%%%%%%%%%%%%%%%%%%%%%%%%%%%%%%%%%%%%%%%%%%%%%%%%%%%%%%%
% MARKS SET

% i	Introduction
% h	Historiography
% c	Conclusion
% m	Mark Set (this)
% w	Working Point
% f	WTF are we doing here!?
% 

%%%%%%%%%%%%%%%%%%%%%%%%%%%%%%%%%%%%%%%%%%%%%%%%%%%%%%%%%%%%%%%%%%%%%%%%%%%%%%%

% Conops, perhaps use a narrative flow moving from before invasion for the
% prep work and training, to the actual invasion for execution and adaptation,
% to transition around or after TOTALIZE?

% TLDR:  if you accept that the British CA soldier was tactically worse than
% the Germans but were able to defeat the Germans by unloading an ungodly
% amount of steel and explosives over their head at the first sign of 
% difficulty, then the British logistical soldier was far superior.  A war 
% cannot be won by killing alone!

% Thesis: 

% L1:  Historical community is quite hard on the British for lacklustre
% results vis-a-vis armoured warfare and armour-infantry co-operation; 
% however, what this is missing is that the British Army --- Army, not 
% economy --- work quite well at a logistical level.  This is what, in the
% field, enabled the British Army to defeat the Germans despite tactical
% mediocrity.  Logistics worked to support flawed tactics.

% L2:  Criticality of the services to the effective waging of war.

% L3:  We ought to stop viewing armies as mere fighting forces.  Understanding
% them using a systems approach explains why we can win wars.  Merely 
% examining how an army can outfight its opponent at a tactical level misses
% the operational reasons the tactical level can even function.


% CONOPS:  What if I do this as a microhistory of 90 Coy?  I can use them as 
% a way to talk about the criticality of logistics.  What if I used them as
% the pivot around which an army (tbh, 27 Armd Bde, and 3 Div) can turn?  I 
% can use it to skirt around the wider issue that I don't have evidence for
% a macro view but I have 90 Coy's WD.  Alas, I can't find as many decorations
% for them as I would like to really make it personal.  Could I use examples 
% from other units as a `take Cpl Bloggins from X, note the work he did'?
% time to run git branch I suppose.  It also allows me to be narrative which
% is always fun.  Their 4 day history was already 7 pages, I could breath life
% into.  I can harp on about them for 8-10 x that, right? 1-2 pages 

% introducing (let's do it as if they hit a beach and they're driving a 
% convoy up the road to Benouville with that preload of supplies for 6 Airborne
% whence we pause to discuss what we currently talk about in the 
% historiography.   I have 5-10 pages on what other people have been writing 
% about Normandy, from there, perhaps another 5ish to narrow down on Sword 
% Beach, the whys, the objectives, who was involved etc. --- I should re-read
% return of martin guerre I think.  

	% so 20-30pgs here, I'll probably do more once I footnote it all.
% With this context, we return to the convoy and lay out the theoretical 
% framework by which logistics operates.  Depots, lengths of supply lines,
% how distribution happens, etc.  How this work integrates with the rest of
% the army.  We deliver the supplies and talk about the RV.  Let's also start
% talking about the arduous work over the next few days to supply 3 Div and 
% 27th Bde as there are no other 2nd line transport units.  As we continue 
% here, I can intersect the sources for 27th Armoured Bde to give context
% as to what 90 Coy was doing.  I suppose I"ll have to do some probables and
% perhapses.  We can talk about how 90 Coy supported the Bde until the Bde was
% disbanded.  Then, continue talking about how the Coy supported other units
% till Totalize maybe?  Then, have a final almost pre-conclusion discussing
		% Perhaps this takes 4-6 pages-ish?
% how, whilst we don't write about it, these operations would have been 
% impossible without the supporting arms and concluding with how the British
% ability to have tactical mediocrity and an unwillingness to spend lives
% meant that their logistics had to be good, really good.  

% Expand this to how armies function, and who this paper doesn't cover.  I 
% don't discuss workshops, clerks --- the whole of A branch actually --- MPs, 
% BADs, beach detachments, etc. yet they're all important. (5 pgs)  Maybe 
% paint a picture of the whole rear area and why it's important (landing 
% tickets maybe?).
% Tie it back to historiography and methods like  how a lot in the sources 
% requires you to intersect doctrine with the WDs to figure out what supply 
% units were doing --- making a meal dely isn't something that really gets 
% recorded in the WD, etc.  (5 pgs)

% Finally, conclude.

\begin{document}

\maketitle

\section{Introduction}

	%Estimate 2-3 pgs?

	% Set scene, we're in Normandy, 6 [time] Jun 44, we just landed, we're
	% driving from Queen Beach to Benouville to link up with 6 Abrn.  
	% you're X lorries are carrying preloads of ammunition, rations, and
	% other supplies for 6 Abrn.  They're holding the Anglo-American 
	% left flank from a possible German counter attack.  what if I put
	% it in the perspective of that Lt who did the initial recce?  You 
	% weren't originally scheduled to land yet but an accident of war
	% means here you are.  Describe more about what 6 Airborne is doing
	% you don't belong to 6 Airborne but are instead of 27 Bde, you're 
	% just tasked to help them on D day, their unit 2nd line unit will
	% arrive from Juno later /* Come back to this when discussing import */

It's 1630 hrs on 6 June 1944, Captain Foreman just arrived at his company 
harbour near Colleville.  An hour earlier, he and the 11 lorries of C Platoon
90 Company RASC (90 Coy) disembarked the LSTs they had been stuck on for the 
past six days waiting to cross the English Channel to support Operation 
Overlord, the Anglo-American invasion of Normandy 
France.\autocite[1--6 June 1944]{90wd}  Loaded in these 
11 lorries are supplies for 6 Airborne Division currently operating to secure 
the British left flank over the Orne.  These loads consist of `pet[rol], 
[ammunition], R[oyal] E[ngineer] stores, and water', stores vital for the 
paras of 6 Airborne Division to resist a German counter 
attack.\autocite[S \& T Report (June History Report) p 4]{90wd} 
%cite check this later
Alas, despite the urgency of these stores, Major Cuthbertson, 90 Company's 
Officer Commanding has yet to make contact with 6 Airborne so C Platoon has 
little to do but wait for contact to be 
established.\autocite[6 June 1944]{90wd}
Thus, doubtless, the men of C Platoon, 90 Coy would have dismounted their 
lorries and pause.  Likely, they would have appreciated being once more
on dry land having spent the last few days being bounced up and down in the
English Channel.  A few kilometres away, the men of the 6th Airborne Division,
the 3rd British Infantry Division, and 90 Coy's home brigade, 27th Armoured
Brigade were, in the case of 6th Airborne, guarding the British flank, or
in the case of 3 Div and 27 Armoured Bde, pushing inland to try to reach
Caen.

	% THESIS
	% You know what you're doing is important, without these supplies and 
	% the war will ground to a halt.  

	% You wonder what the historians who write about these events 80 years 
	% in the future will think about it all, what drives your focus?

	% Maybe have a bitter reflection, it's always the fighting troops or
	% the generals who get the cheers.  No-one applauds the cooks!  Use
	% the perhapses of history for this

	% Shift to that historiographical frame:  With the benefit of 
	% hindsight, we know that the drive to Caen would not be a quick 
	% drive, but as a logistician, one asks oneself, was there a 
	% significant logistical constraint or was it due to something
	% else?

%e reword the this a little to emphisize my thesis in the context of expending
% material / men
Of course, the vital efforts of the 6th Airborne Division and the other 
fighting troops of the British Army in Normandy have been fairly well studied.
Extensive critiques and justifications have been made on British 
infantry-armour co-operation, the aggression --- or lack thereof --- displayed
by British troops, Allied inadequacies in armour, Montgomery's personality,
tactics, vs firepower, etc.  In short, we often discuss what went wrong or 
how we fought; however, what we often ignore is the critical question of what
enabled us to fight. The work done by troops a few kilometres behind the front 
line is generally ignored as a side-show; yet, the work of ensuring the combat 
arms are well supplied with all the minutiae of war from ammunition, to food, 
to water, and other general supplies is what will make or break an army. Thus,
in light of this gap, I hope to argue for the centrality of logistics in the
British preference to expend firepower rather than lives.  The British Army
seems quite helpless compared to the might of the Wehrmacht until one looks
at this Army from a systems approach.  It is however, this systems approach
that reveals the British Army's strengths.  %e I hate this para!

% add section on structure of argument --- maybe later once I un-f that para
To examine the centrality of logistics in British Army operations, we will 
follow Major %something
and 90 Company RASC as they work their way across the English channel, landing
in Normandy and following them as the units they support attempt to capture the
city of Caen, and we will examine their role in the closure of the Falaise 
Pocket in August.  Along the way, we will first examine how the British Army 
structured logistics administratively, before joining 90 Coy as they support
the 27 Armoured Brigade as they partake in the Battle for Caen.  
After 27 Armoured Brigade is broken up at the end of July,
we will see how 90 Coy integrated into a larger and longer supply column as
they support infantry units through Normandy. Following this, we will have 
a brief discussion on historical methods and how they apply to military 
logistics.  %e I also hate this para, later problem!


	% \subsection{Clarification of Terms} % maybe omit?
		% How I use D-Day to actually mean D Day and not 6 Jun 44
\section{Historiographical Review}

	% `Of course we know that 80 years after the fact, historians...'

	% Complain loudly and long-windedly at the lack of discussion on log 
	% in /*insert list */  Consider subsections or flow of text?

	% Discuss the current logistical scholarship done
		% Namely historical work in 50s and Supplying War, Julian
		% Thompson, Great Feat...

	% Introduce Section

	% General observations:
		% Much work done on great men, tactics, the merits of German
		% Armour; and some work has been done on American logistics,
		% as well as logistical peculiarities like the mulberry 
		% harbours.  I fear little has been done on the actual
		% military administration of the war.  Indeed, the logistical
		% section of military history is fairly poorly written about
		% by historians, some more work done military academies
		% reflecting its importance to them

%e massage a transition into this

The Battle of Normandy is of course, a well studied topic.  Much has been
written on this battle from books on the Second World War at large to 
publications that focus squarely on operations and tactics in Normandy.
Curiously, there is also a second historiography which discusses logistics
at large; however, the precise area of military logistics in Normandy is
less well covered. 
%e I'm feeling narative and I don't wnat to sum up these books again.  Later
% problem!

	\subsection{On WW2} % there's got to be a better name than this
		\subsubsection{\textit{Britain's Other Army:  The Story of
			the ATS}}
			% This feels kinda awkward to put here
		\subsubsection{\textit{Why the Allies Won}}

	\subsection{On Normandy}
% reorder this, chrono flow or based on argument?

% really flesh out the common critiques and defences of the British Army, 
% that it was slow to develop, under manned, unimaginative, morale problems,
% materiel over lives

		\subsubsection{\textit{Clash of Arms}}
			% Argues that the British were slow and failed in
			% innovating.  
		\subsubsection{\textit{Overlord}}
		\subsubsection{\textit{Fields of Fire:  Canadians in 
			Normandy}}

			
		\subsubsection{\textit{Montgomery and `Colossal Cracks':  
			The 21st Army Group in Northwest Europe, 1944-45}}
			% manpower Constraints
			% Morale problems Be sure to emphasize both of these
			% as supplies are critical to this
		\subsubsection{\textit{The Normandy Campaign 1944}}
		\subsubsection{\textit{Gators of Neptune: Naval Amphibious
			Planning for the Normandy Invasions}}
		\subsubsection{\textit{Neptune:  the Allied Invasion of 
			Europe and the D-Day Landings}}
		\subsubsection{\textit{From the Normandy Beaches to the 
			Baltic Sea: The North West Europe Campaign
			1944-1945}}
		\subsubsection{\textit{Feeding Mars:  The Role of Logistics
			in the German Defeat in Normandy, 1944}}
			% Toss this in Normandy or Log?

	\subsection{On Logistics}
% reorder this

%% Do I wanna put the books for WW2 logistics here or in WW2?  I'm tempted
% to concentrate it here but I also like the idea of keeping something from
% the field of military science and not history separate --- military science
% cares much more on actually executing operations.  Split this into two 
% subsubs and make the rest subsubsubs?  One for history, one for less so?
% Could also compress, maybe compressing and not sectioning this section 
% will flow better.  In any case, I wonder if it's better to chk pg ct

		\subsubsection{\textit{Supplying War:  Logistics from
			Wallenstein to Patton}}
			% Foundational in logistics scholarship, but, given
			% it's broad scope, it lacks depth
		\subsubsection{\textit{The Lifeblood of War: Logistics in
			Armed Conflict}}
			% Limited and tries to cover a lot of periods
		\subsubsection{\textit{A Great Feat of Improvisation}}
			% British but only really to shortly after Dunkirk

			% Use it to talk about how supply developed during
			% interwar years, namely, motorization.

		\subsubsection{\textit{War of Supply:  World War II Allied
			Logistics in the Mediterranean}}
			% Chiefly American in the Med
		\subsubsection{\textit{Supplying the Troops:  General 
			Somervell and American Logistics in WWII}}
			% Very much a great man history.  Logistics in the
			% form of a biography
		\subsubsection{\textit{Military Logistics and Strategic 
			Performance}}
		\subsubsection{\textit{The Story of the Royal Army Service
			Corps}}
		\subsubsection{\textit{Logistics and Modern War}}
		\subsubsection{\textit{Logistics Diplomacy at Casablanca: 
			The Anglo-American Failure to Integrate Shipping and
			Military Strategy}}
		\subsubsection{\textit{Strategy and Logistics:  Allied
			Allocation of Assault Shipping in the Second World
			War}}
		\subsubsection{\textit{The Science of the Soldier's Food}}
		\subsubsection{\textit{D Day to VE Day with the RASC}}
	
		

		

	\subsection{Tools of the Trade} % maybe rename this

	% The historiographical Gap.  Introduce an inattentiveness to log
	% here, that the materiel advantage is contingent on getting this
	% right.
	
	\subsection{A Note on My Sources}
		
		% Heavy reliance on digital records

		% paper records from Canadian sources owing to funding but
		% argue for the soundness of the methodology anyways given
		% how Canadian stuff is of Br doctrine

% Start with Context %%%%%%%%%%%%%%%%%%%%%%%%%%%%%%%%%%%%%%%%%%%%%%%%%%%%%%%%%%

\section{Overlord as Planned} % rename this

	% We return to our convoy, you're thinking and a bit day-dreamy 
	% driving through the French countryside thinking about the 
	% magnitude of what you're actually doing.  DO I want to go this
	% tone??

	% Overlord & Neptune:  Mission, invade France, push the Germans
	% across the Rhine sooner or later

	% Division of beaches, start from the W and move E.  How beaches
	% are subdivided.  Start thinking about the beach exits maybe and
	% why they matter?  Maybe reflect on the massive traffic jams?

	% We're intersted in Sword beach.  Over the Orne, Paras of 6 Para Div
	% holding the Left Flank.  From Sword Beach, 3rd Br Div was assigned
	% area from Swrod beach to Caen.  In this region, we have 27 Armd Bde.  
	% they provided tank support for 3 Br Div.  90 Coy was 27 Bde's 
	% 2nd line transport coy.  It basically provided the Bde's logistical
	% support.  On D-Day, ___ pl 90 Coy was also tasked to run supplies to
	% preloaded supplies to 6 Para over the river.  6 Para's 2nd line
	% transport was landing at Juno.

	% Talk about the 'bloody army', Q branch, A branch, and G branch.

Op Overlord was made up of a number of smaller operations.  The seaborne
landings were part of Op Neptune.  This was the operation that established a
50 km wide logistical beachhead in Normandy.  Neptune divided this section of
Normandy coastline into five discontinuous beaches.  The Allied right was 
anchored by Utah beach on the Cotentin Peninsula and the Allied left was 
anchored by the River Orne and the Caen Canal at Sword beach.  Between
these flank beaches was Omaha, Gold, and Juno beach.  The Americans were 
responsible for Utah and Omaha, whilst Anglo-Canadian forces were responsible
for Gold, Juno, and Sword beaches.  Each beach was subdivided into a 2 -- 4
sub-beaches and assigned a letter from A to R.  This study will primarily 
concern itself with the affairs of the troops of the 3rd British Infantry 
Division and 27 Armoured Bde that landed at Sword beach, specifically, Queen 
beach.  

This study will also concern itself with the work done by 6th Airborne
Division as part of Op Tonga.  Their objective was to execute a series of
airborne landings East of the River Orne, Caen Canal, and Sword Beach to 
secure the British left flank.  They were also to capture the only bridge 
crossing
these water features North of Caen along a road running between Benouville
and Ranville.  All this was to be done during the night before the forces of
Op Neptune landed.  For approximately six hours, the paras of 6th Airborne
would be cut off.  Once the British landed at Sword beach, they would push
inland, to Benouville, cross the bridges if they were still intact, and
reinforce and resupply 6th Airborne.  That is how the 11 lorries of C Platoon 
90 Coy finds itself waiting in Colleville, around 4km away from Benouville
waiting for their CO to link up with the Paras so that C Platoon could 
resupply 6th Airborne who would likely be running low on stores by this point.
C Pl would then keep the paras supplied via Queen Beach until 6th Airborne's 
RASC unit could take over on D + 1 after landing at Juno.\autocite[1]{90wdjun}

By 1800, C pl made contact with the Paras and, as the Paras had successfully
captured the Orne and Caen Canal bridges, C pl was able to replenish the 
depleting ammunition of 6th Airborne by 2300 hrs on D - Day --- a five hour
job.  As 6th Airborne's area of operations had yet to be fully secured, the
drivers of C pl faced sniper fire throughout the day.\autocite[6 June 1944]
{90wd}

Not all of 90 Coy landed on D - Day however, whilst A and D Pls stayed in the
UK to be brought across the channel on % date
B Pl landed on D - Day.  
Their tasking to simply support 27 Armd Bde primarily in terms
of their fuel requirements and to otherwise keep the Bde supplied.  Their 13
lorries were mainly loaded with fuel for the Bde's Sherman tanks.  Alas,
Due to the heavy shelling of Queen Beach however, only 9 lorries actually 
landed by 1200 hrs.  The lorries that landed proceeded to the 27 Armd Bde's 
A Echelon Area in Hermanville-Sur-Mer and would quickly be put to work keeping
the Bde supplied with fuel and ammunition.\autocite[6 June 1944]{90wd}
Hermanville, situated along the main road departing Queen Beach --- location
of the Beach Sector Stores --- rapidly became 90 Coy's control point where
vehicles would check in before proceeding to the beaches or to the units.

As a point of
curiosity, you may have noticed how B Pl was not preloaded with ammunition.  
This was because the Bde brought their own ammunition ashore firstly with
the ammunition they carried in their tanks, but also with the ammunition 
they towed behind their tanks in \textit{Porpoise} sledges.\autocite[2--3]
{90wdjun}  These sledges
would be released shortly after the tanks made it ashore.  Collecting the
ammunition in these sledges also became one of B Pl's tasks in the first
hours of the invasion.%cite 

Perhaps as a happy co-incidence, Neptune had failed to meet it's D Day 
objective of pushing all the way to Caen --- an optimistic goal anyway.%cite
This meant that supply lines were shorter than planned which doubtless 
decreased the stress on the 9 lorries of B Pl.  
It is difficult to understate how heavy the 
fighting was.  Indeed, there were many instances where tanks were replenished
with tanks still `in their forward positions'.\autocite[2]{90wdjun}
This single under strength platoon was trying to keep a whole brigade supplied.
Tasks which would ordinarily been reasonably simple tasks were now incredibly
onerous.  Take for example the task of refuelling and reammunitioning the tanks.
What should have been a simple task done at the end of each day to ensure the
Brigade was ready for the next day's operations became a night long ordeal 
requiring the initiative of the 9 lorry drivers of B Pl who had to understand
the requirements of their client unit before returning to the beaches to try 
to obtain the critical stores required by their units.  It was paramount that
these drivers not only knew what was needed, but the priority of what was 
needed in the event that there were insufficient stores available to meet
an urgent order.  This way, lorries were always moving and stores were 
always flowing.  Fortunately, by nightfall on D - Day, a small Brigade supply
dump was beginning to form in Hermanville --- an act that would logistics 
chains.  Even still, this put a great strain on the men who were
worked day and night until D + 4.\autocite[2]{90wdjun}


% insert a transition into how B Pl was running around from BAD to units via
% Hermanville
Thus was the dispositions 90 Coy on D-Day, two Pls would make their way ashore:
one to support their parent unit, 27th Armd Bde and one help the Division to
their left --- 6th Airborne --- until their own RASC unit could make it. Here,
one can begin to see the role of 2nd line transport companies such as 90 Coy.  
They form the final interface between the wider supply system and the fighting
units ---  it is these units that \textit{deliver the goods} --- however, how
did these 90 Coy interface with the rest of Army?  
% mabybe rethink this transition


\section{The Supply Chain in the Field} 
% move to Hermanville, 90 Coy's control point in running convoys

Whilst admittedly, the supply system on D - Day did appear somewhat improvised
and ramshackle, there was good reason for this.  Because the British failed to
advance as far forward as planned, the supply dumps that were to be set up all
along Sword Beach failed to materialize in the same way as planned.  Still,
the logisticians of the British Army tried to beat a formal planned system
into an effective supply chain however much improvised.  It is worth recalling
that, even without additional planning, the British Army's baseline doctrine 
included a supply chain.  This was after all, an army that could expect to be
deployed to not just fight a large, European Army, but also fight small wars
across vast stretches of the British Empire.  To do so, the British Army already
had an organic logistical capacity that Overlord adapted to its use.  
At it's core

\begin{quotation}
	 The principle of supply [in the British Army was] that field units 
	should always have
	 with them, or within reach, two days' rations and forage, and one 
	 iron ration, and that these stocks should be replenished by 
	 delivery, at a point within reach of the troops, of one day's ration
	 and forage each day. %cite No27 Trg Course p. 32, quoted in,
	% refs F.S.R. Vol. I. Sec107(I). %validate this
	
	% add remark on fuel range
\end{quotation}

Moreover, as the British Army was fully mechanized by the Second World War,
it was the aim that all vehicles would have full petrol tanks at the end of
each day. To enable operational mobility, 2nd line transport was also to have 
immediately available, an additional 50 miles of fuel; and 3rd line transport,
a further 25 miles instantly available for use.\autocite[\Petrol][s 3]
{27course}
Of course, it is unlikely that this exact fuel holding was available on 
D - Day;
however, this was the standard the British Army would have expected.  These
principles meant that, at any one point, the British Army was expected to be
able to advance independent of it's bases for slightly over 75 mi over the 
course of three days.  Thus, this formed it's maximum operating range.

Of course, it is suboptimal for an Army to operate for long without access to
its supply chain so, to support the Army, the supply chain was broken up into 
four main areas, ordered from furthest to nearest the front line, 
the Base Sub-Area(BSA), the Line of Communication Area (LoC), the Corps or 
GHQ Area, and finally, the Divisional Area.  Those depots that 90 Coy went to
along the beach?  Those were Beach Sub-Areas (BSA).  

\subsection{The Base/Beach Sub-Area and Line of Communication}

In the first days at 
Normandy, it appears that Beach and  Base Sub-Areas were treated as one and
the same.  Whatever the `B' stands for, BSAs functioned as the British Army's
initial interface between sea and land.  The BSA had %was it within it?
the docks, the base railway marshalling yard, a main supply depot, a petrol
sub-depot, field bakery, and detailed issue depot. Cold storage was also 
available for rations such as sides of meat, etc --- of course, it is unlikely
that such niceties were available in the first days of the invasion, fresh
rations weren't even available for quite some time. %cite graph in 27course

The BSA would then theoretically interface with the Line of Communication
Area (LofC).  These were railway networks or truck convoys that transported
stores from the BSA to the field army.  Now, the supply lines in Normandy were
quite short, measuring in the ones or tens of kilometres.  It was simply 
unnecessary to have a strict LofC area per se.  The field army could simply
draw stores directly from the BSA --- the LofC area really is not necessary
until the field army is some distance away from the BSA.  The LofC would 
become necessary as the British Army advanced through France and into 
Germany.  As they went deeper, scheduled and intentional convoys to convey
the stores would become more useful in relieving the field army of such 
transport network.  

\subsection{Supplies in the GQH, Corps, and Divisional Areas}

In any case, regardless of whether the Army was drawing stores directly from 
the BSAs or from the LofC, eventually,  Army would have to start drawing
stores.  To such ends, the Army was divided into two sections the 
Corps / GHQ Area and the Divisional Area.  Typically, the distance --- and 
thus, also depth of the Army --- from the LofC area to the delivery points
was 30 -- 40 mi (50 -- 65 km)  At the GHQ level, one begins to
see how the British Army sorted supplies.  POL and other stores were handled
in two theoretically separate systems.  In either case, it is at the GHQ level
that stores were bulk broken.  

Let's handle the general stores first.  Stores are delivered to the Supply
Column (Sup Coln) where stores are bulk broken.  Think of this bulk breaking 
with the analogy of a grocery store.  A grocery store may receive it's goods in 
wholesale, bulk form, but then repackage it into smaller, more usable units to
be easier to sell --- a retail customer may want 1 lb of almonds, not 1 ton
for example.  In the case of prepackaged stores, bulk breaking is more similar
to the procedure that occurs when a grocery store receives a palette of cereal
which is subsequently unpacked and loaded as single units on a shelf.  Thus,
the Sup Coln HQ can function as an interface where the Army's bulk handling 
meets it's piece handing functions.\autocite[\SupInWar][3]{27course}

\subsubsection{Petrol, Oil, and Lubercants (POL)}

Likewise, fuel could, at times be shipped in bulk initially however fuel for
the British Army was never delivered to field units as such.  It was always
containerized first into tins.  There are few modern equivalents to this in
our modern world.  When we buy fuel at the petrol station, we pump it from a 
massive underground tank into our cars where it's sold by volume.  Rarely do
we buy a pre-packed can of fuel.  This was however how the British Army 
preferred to receive it's fuel --- in 4 Gal (18L) of petrol per tin.\footnote
{\cite[\Petrol][3]{27course}.
For reference, the 2025 Toyota Corolla sedan has an approximately 50 l fuel 
tank whilst the 2025 Ford F150 Raptor pickup truck has a 136l tank.}%cite
These tins were nicknamed flimsies, and it was not an ironic term of affection. 
They were meant to be disposable so they were built cheap; however, the 
design teams were perhaps overzealous.  The flimsies had an unfortunate habit
of breaking or leaking such that it was quite common for them to arrive 
damaged leading to fairly severe losses in fuel as well as a notable fire risk.  
Indeed, the flimsies were so bad that the British Army began to simply use 
captured German (Jerry) petrol cans --- hence our modern term jerrycan 
(a German petrol can).  

Nevertheless, despite the questionable durability of flimsies, the British 
Army had some sound reasons for using containerized, as opposed to than bulk 
distribution.  Firstly, tanker lorries weren't nearly so common in 1940 as the 
are today.  Secondly,
containers are compartmentalized.  If a bullet pierces a tanker lorry, one may
loose thousands of litres of fuel before one notices; however, if a bullet 
travels through a containerized fuel transport (i.e. lorry full of flimsies),
one may loose only a few tins worth of fuel.  Moreover, containerized fuel has
far fewer mechanical requirements.  For bulk fuelling to work, one must have a 
working petrol pump.  This could be quite inconvenient.  Imagine having a 
tanker load of fuel but no simple way to get the fuel out of the tanker.  
Moreover, using this system, you can only fuel a few vehicles at a time.  
With containerized fuel, one merely pulls up to the vehicles, unload a few
tins at each vehicle, and each crew then subsequently fuels their vehicle
with a cheap tin funnel.  Of course, this system was quite laborious to 
use but even so, it was judged by the British Army that the additional labour
was worth the cost.  % packaged at factory?

%e define POL
%e para is awk, solve later.  It reads like it was inserted after original 
% writing --- newsflash: it was lol!
All told, the British POL supply chain was designed, to provide containerized
fuel for the Army.  As designed, it was intended for the Army to be able to
advance the whole army 75 mi (120 km) using only such reserves held by the 
field army (the GHQ/Corps areas, and the Divisional Areas).  50 mi (80 km) 
of fuel would be held by the the Divisions, whilst the Corps areas would hold
the remaining 25 mi for the divisions, plus an additional 75 mi for the corps'
organic transport.\autocite[\SupInWar][3]{27course}

Having been bulk broken at the Corps or GHQ levels, it was now up to the 2nd
line transport units like 90 Coy to then bring those stores forward into the
Divisional areas and deliver them to the end-user units.  Depending on 
operational requirements, this may mean delivering it directly to the 
individual end-users, or it could mean delivering such stores to the units who 
could then further distribute stores internally.  This formed the basic, 
theoretical structure of the British Army's supply chain; however, just as how
no plan survives first contact with the enemy, the supply chain had to adapt
to tactical and operational necessities.  

Already, you may have noticed that the 27th Armoured Brigade is a 
\textit{brigade}.  Why does it have it's own 2nd line transport?  The answer
is fairly simple, 27th Armd Bde's full name was 27th Armoured Brigade 
(Armoured Assault). %e fact check
The Bde was raised as an independent armoured brigade for Overlord.  As such,
it needed a way to ensure it could run its own logistics.  You may also recall
how 90 Coy was, on D-Day, delivering both POL as well as ammunition to 27 Armd
Bde.  This shows how the supply chain had to remain flexible.  Whilst in 
theory, there was a separate chain for POL and ammunition, in practice, this
was impossible.  This was the advantage of containerized fuel as fuel could 
simply be loaded into any available lorry.

\subsection{Storage and Dumping}
Finally, before we carry on with the affairs of 90 Coy, it may be prudent to 
clarify what is meant by a `dump' and other forms of storage.  In a perfect 
world, supply chains would be perfectly efficient.  Ever single item required
by an army would be produced when it's needed, sent to where that item was 
required without delay, and used immediately on receipt.  Alas, hiccups 
invariably appear.  Shipping gets stalled, major operations consume unusually
large quantities of supplies, supplies are lost to enemy action, etc.  Thus,
to ensure first-line units receive a continuous flow of supplies, it was ---
and remains --- necessary to store a reasonable reserve of stores at various
points along the supply chain.  

Ideally, this would be a large, dry, flat, 
climate controlled warehouse with good transport networks, but alas, 
conditions in the field often are not always ideally suited to the 
logistician.  Thus, supplies were often stored by stacking supplies in a 
field or some woodland and covering them with tarpaulins if they required
protection from the weather.  The precise requirements of this may seem quite
trivial and not terribly important to the profession of fighting wars; however, 
seemingly trivial tasks such as labelling and organizing are critical.  
Consider what would happen if there was a German counter attack and the supply
officer could not find the 76mm anti-tank shells because their boxes were
not properly labelled or because the dump was not given enough land so that 
the aisles were too narrow.  Moreover, what would happen to those same shells 
if they were dropped and the packaging was inadequate to protect their contents
--- and honestly, who hasn't dropped a heavy box before.  Damage to the shell
casing could prevent the casing from ejecting properly after firing leading to
a stoppage and possibly leading to the tank being out of action.  

Consider also what would happen if one of these these dumps was attacked and
caught fire.  Aisles do not merely provide access but function as fire breaks.
These fire breaks are critical for hazardous material dumps such as POL dumps
or ammunition dumps.  When these dumps catch fire, it is often too dangerous
to attempt to extinguish the fire --- POL burns and High Explosives explode.
Instead, standard operating procedures tend to relate to containing the fire
and letting it burn out on its own.  %e insert medal citation example

This may seem small but how do acts like this win wars?  Unlike the combat 
arms, logistics does not win wars by plunging a bayonet into the hearts of the
enemy.   
Instead, logistics wins wars by ensuring the combat arms can act without 
restrictions.  If there is insufficient ammunition or fuel to support an 
advance, a General cannot order that advance.  If reserves are not ready when
the enemy attacks, then the combat arms will have few options but to withdraw
or fix bayonets.  
Logistics enables and constrains but achieves nothing on its own but by doing
so, is a significant factor in determining if an operation is achievable or
foolhardy.  Let us return to Normandy in June of 1944 to see this in play.
%e how's this transition?  I kinda don't like it.

%%%%%%%%%%%%%%%%%%%%%%%%%%%%%%%%%%%%%%%%%%%%%%%%%%%%%%%%%%%%%%%%%%%%%%%%%%%%%%%
% Stop Work Point 29 1908 Jul 25
%
% Currently figuring out how I want to transition to the present day.  Thus
% far, we're at D-Day, C pl's delivered to paras, B pl is run off their feet
% delivering stuff.  Maybe touch D-Day evening and lead into D + 1?  A nacent
% dumb is now being formed at Hermanville.
%
% Crosscheck citations for DHH.
%
% I should probably define 1st line, 2nd line, and 3rd line units sometime.
%
% Preparing to talk about what happens to the company after the 27th is 
% disbanded.  Starting work on August diary.  Talk till end of Aug to show
% other side of log.
%
% Flesh out \section{Criticality of Supply} Do I want to make the 
% historiographical section a subsection or it's own section?
%
% Do I want to use more subsections?  I like how they organize.
%
% Working note, fleshing out the analysis more
%
% Should flesh out the explanation of sup as sys in earlier section
%
% Wondering what to do about the sectioning where I talk about the structure
% of supply. 
%%%%%%%%%%%%%%%%%%%%%%%%%%%%%%%%%%%%%%%%%%%%%%%%%%%%%%%%%%%%%%%%%%%%%%%%%%%%%%%
	% Talk about how supply works, the DIDs, supply lines, POL differences

	% complaint about the flimsies and containerized fuel

	% talk about how you sustain client units, march length, the end goal
	% of ending the day with filled tanks, full ammo, etc.

	% Keep that story telling feeling, you're at the control point mourning
	% the fact that there are no other second line units, you're vastly 
	% overworked.  If only you had the additional units to support you.
	% if only ____ instead, you're left trying to maintains [sustainment
	% standards] and just keep what's needed flowing as hard as you can.

	% Use to establish what we're actually dealing with so importance
	% becomes self-evident
	
	% How supply chains work from dumps, and depots, to distribution, to
	% 1st line units.  


	% Key was placed on  on rabild concentration of force, incl sup, LofC
	% etc. to permit strategic use of force %cite 27 trg crs p. 32

		% 4 main areas, the BSA, L of C, Corps/GHQ area, Div area
		% goes from 

		%cite No 27 Trg Course p. 34ish (diagram's on 34)

		%\subsubsection{L of C Area}

			% Envisioned as a railway line, lorries will do 
			% just fine but this theory came at the dawn of
			% mechanization.  Point was to take stores from
			% the BSA and transport them near the combat zone.
			% It's basically a transport area.  IT goes from 
			% just outside the BSA to, and including the railhead.

			% If distance from the BSA to the railhead is < 12
			% hr trip, BSA will control.  Else, there will be
			% a regulating station around 6 hrs from the rail
			% head. %cite no27 Winter Trg p. 32 5c
			
			% If using trucks, it's 150 tons per 21 trucks with
			% old pat, 59 tonnes across 12 trucks (5 supplies, 2 
			% ord, 2 R E, 2 stores, 1 RAMC)  railway trucks 
			% %cite No 27 course, p. 32 para 5


	

	%\subsection{Warehousing}

		% Especially in those early weeks when a lot of it was dumping
		% rather than warehousing per-se.  

		% DID (maybe??) 3000 Tons/acre= gross stacking area.  x 4-10
		% for expansion %cite No 27 trg 32 (this might be a general
		% rule too)

		%\subsubsection{Base Supply Depots}
			% Located near Base Marshalling Yard, outside docks
			% to permit expansion and dispersion.  Consider `water
			% light, telephones, good roads, office accommodation'
			% when selecting the location. Area = Strength X 
			% stock X weight one ration / (unintelligible) = 
			% 8.5 sq ft /ton = stacking 
			% +50\% for stock spacing = gross stacking area X 4
			% to 10 for expansion, etc.  (that's a quote)
			%cite 27 trg course p.32, 4
	% zoom out a little and talk about what you'll be controlling.
	

% /* Figure out where to incorporate the fact that the British/Canadians focused
% on firepower over manpower.  This means materiel is critical --- A is for ammo,
% B is for beans, C cold water, D: diesel, E-everything else... */

% /* Do I want to expand to include things like traffic control?  Traffic jams
% on Sword Beach may have made the Br fail to capture Caen on D.  10m of dry
% beach between water and sea wall at high tide.  Perhaps an MP or two would
% have solved the issue. IIRC, RAF beach sqn dealt with it. (See
% RAF beach sqn/det  Was this a critical
% oversight?  Not a lack of tenacity or anything else, but a good, old fashioned
% traffic jam VI's-a-vis Toronto at rush-hour caused the failure to take Caen?
% lol --- what a way to win a war! */

\section{Return to the moment} 
%New title, want to get back to what 90 Coy was up to till D+7
% sum up the Jun Hist Rep for 90 Coy 

\section{Operation } %name, if any?  Was hasty operation.  
% Maybe just consolidations?  Oh, how about the Dumps at Ranville

% give context for the situation IVO Ranville at the time.  British holding
% several KM^2 E of the Orne at IVO Ranville.  This land was supplied from main
% beaches over the Benouville-Ranville bridges.  Single point of failure.  
% fear of DE cntr attack as DE tps moving IVO E of region probing for points of
% failure.  27th Armd Bde involved in supporting local infantry.  90 Coy
% supporting Bde

% Operation with Paras of 13/18th with paras, 90 Coy dumping stores in support
	% 12 June, mail arrives, first letters since D Day

% 13/18 moved E of river to defend against counter attack %cite 13/18 WD 12 Jun
% A, B Sqns supporting attack of German positions by 7 Para & black watch.  
% on 10 Jun, Germans were seating themselves into a wood IVO Le Marquette, 
% threatened to separate 5 & 3 para Bde as well as supporting infantry units.  
% Paras ordered to sweep woods to dislodge Germans with 13/18th in support.
% part of wider fears of a German counter attack.  
%cite 7 para jun WD appendix p. 5
% 90 Coy established a dump using 12x 3tonners.  Mainly Amn & POL, started
% 12 1800 Jun 44.  Increases to 20x lorries next day (how many lorries in 
% a coy again?)
% dump at Ranville (1173) %cite 90 coy WD 12 Jun 
% Do I wanna mention decorations here for 90 Coy or maybe do it earlier
% IVO Ranville.  Positioning of this dump wise.  Only route to supply troops
% E of Orne and canal is via Benouville/Ranville bridges.  This would permit
% bridgehead to keep fighting if the bridges were destroyed.  Infantry armour
% co-operation was a noted problem.  6 tks KO.  Paras take 40 prisoners, 0 KIA,
% 9 Wounded. 
%cite 7 para jun WD appendix p. 6


% Highlight that preparation is key.  If there was a counter attack and the
% bridge was blown, gallantry matters naught if you aren't supplied.  What 
% will you do, fix bayonets --- well, yes actually...?
% C Pl 90 Coy continue to provide light support till 16 Jun

\section{The Arrivals of A \& B Plns}
% non critical section, setting the tone of a lull.  A/B Pl arrive 
% 15 2000 Jun 44.  Go to Coy HQ in Cresserons GR 0379.  Brings with them 
% 59 Veh, 165 Pers.  Arrives 2300 hrs. Some DE bombing.  Over the next 3 days,
% new platoons involved in dewatering and reorganizing in prep for ops.  
% sporadic bombing and shelling of veh pk.  Veh dispersed 75 yds between 
% each veh --- quite exposed with no protection then (they're in an open 
% field).  

% Queen Beach still being shelled.  (appears both in 90 Coy and 13/18WD)
% buildup thus slowed.  13/18th don't get their D+9-10 residues by 18 June
% This is a failure in supply.  Delays due to weather.

% Generally fairly quiet for Bde till end of month %cite 27 Armd Bde WD
% quiet for 90 Coy until 23 Jun %cite WD

% Maybe talk about the perhapses?  Amn still was consumed, as was POL, in 
% addition, what about general transport of rations, etc.  I don't have
% a src, but these must have happened.

% Talk about that wonderfully dull stuff like stove and flour shortages in
% the 27th Bde WD Admin instructions (and defecate in the latrines!).  
% Also, traffic flow patterns, spare parts, uniforms, water points etc.
% Highlight the importance of the ordinary and mundane. 

% Talk about how a POL lorry catches fire in the 13/18 rgmt area on 20 Jun.  
% Spreads to Amn dump.  %cite 13/18WD
% Not in 90 Coy's WD.  Why not, routine fetch and carry.  Fire happens, 
% amn is replaced, maybe only a few lorry loads, no bother.  Talk about
% the need to look between the sources.  An armoured rgmt is unlikely to
% have the integral transport to replenish that dump and frankly, it's not
% the job of 1st line units to maintain a 2nd line dump.  If you just read
% the WD, you would assume they're twiddling their thumbs most days.  This
% is unimaginable!

\section{Operation Mitten 27--28 June 1944}

% Want to show that operations impossible without such units.  Focus on what
% worked, not what didn't work for the British.  British tanks bad, arty good.
% this is a useful moment to talk about the nature of Br Arty.

% Objective (no idea TBH, check 27th Bde WD) TLDR form google search,
% eliminate a DE salient at a chateaux not on my map IVO GR 0372

% 90 Coy's Work

% 30x 3Ton loads of 105mm amn held on wheels for 3 Br div on 23 Jun.  
% Likely for M7 Priest.  Should clarify, this isn't 90 tons but 30 
% truckloads. wonder if I can find out how many 105mm shells a 3 tonner can 
% load onto it.  Does it volume out or weight out?  Emphasize that 90 Coy
% was only helping this unit.  Technically, as the 27th Bde unit, they 
% don't do arty stuff.

% Amn is delivered 26 Jun to batteries located IVO GR 0378, it's D-1.

% don't forget about the stuff for flamethrowers!!

	\subsection{British Artillery} % amn consumption in barrage, arty
		% usage, etc.  Show just how dependant the Br were with
		% arty.  Maybe get the mass of a 105 shell and propellant?
		% there's no way to get an accurate estimate though, if
		% only they were 25 pdrs...

	\subsection{Support to Operations}
		% Operation used flamethrower tanks from 141 RAC (Churchill
		% Crocodiles)  Needed special fuel and nitrogen cylinders.
		
		% Coy sets up dump of 3000 Gal (13640L) of fuel and 90 N cyl
		% at Gazelle (GR 0276) on 27th, withdrawn on 29th.

		% D+1, Maint Point set up at Le Vey (GR976757) distributing
		% 2400 Gal of FTF.  

		% D+1 for Staffs Yeo, 0300, 1082 rds 75mm HE delivered, 
		% 600 rds at 1600 hrs, 1200 rds at 2000 hrs delivered --- what
		% where the Staffs doing?  Find out  Also for context, give
		% amn capacity of tank  

		% in Op Inst No 2 for 27th Bde (28 Jun 44) sqn shoots limited
		% to 50 rds/gun.  Tanks were to harass Germans as long as 
		% Br infantry weren't harmed by this fire and the tanks move
		% as soon as counter battery fire is encountered
		%cite 27th Armd Bde WD, Jun 44 p12
		% This tells us there's no great fear of running out of 
		% ammo.  If you could fire off half your ammo in sqn shoots
		% and still be operationally ready for further operations or
		% movement, you're confident you could be replenished.
		

	\subsection{Figure out a name} %rename
		% conclude this section maybe this doesn't actually need a 
		% subsection.  This much work was needed for a simple 2 day
		% operation.  Work vital but it was doable.  Sum it back up.
		% Maybe the tanks weren't the best, but their inadequacies were
		% floated by a supply chain that kept up so that they could be
		% bad but still effective.

\section{Operation Aberlour}
% Never took place, do I wanna talk about it?  Issued 27 Jun 44.  A lot of 
% admin stuff for 90 coy like amn dumps, etc.  (Cite p26 27th WD for Jun)
% more on p30.  I almost like the idea of talking about an op that never
% takes place.  It eliminates the fog that arises from what happens when
% there's contact made with emy.  Here we can focus on how operations
% take place.  Hmm...

\section{} % Run up to Op SHERWOOD

	% extensive minefields to be laid (do sq footage) 
	% (See pg 13 of 27th Bde WD)

\section{The Lead up to Charnwood}

% Use this section to warm up the reader as to how logistics works
% in operations.

% Context for Charnwood

% D = 8 Jul

% Will continue using the Gazelle ammo dump until exhausted (smart, why move 
%it)  This is the only initially authorized dump. 

% 90 Coy sets up POL pt just S of Hermanville 2100-2400 8 Jul (likely units 
% moving through, north to south, fuelling as they pass)
%cite 90 WD 8 July

% AP set up NW of Cresserons at Coy HQ-ish area %cite 90 WD 8 July
% Rgmts draw ammo next morning, ERY withdrawn 1400 when Caen's captured

% 90 Coy is carrying 1000 Gal FTF, 35 N bottles on wheels for crocs at
% Cresserons

% 3 days compo rats issued, additional 3 days AFV packs carried in tks if
% compo not avail.

% Blankets to be distr as situation permits


% tracked traffic to use tracks rather than roads --- degradation
%cite 27th Jul orders 34-6  Sandbags used to store casualty's kit

\section{Pre Goodwood}
% 27th HQ moves to Douvres 10 Jul

% CONTEXT
% Post Charnwood, Bde is withdrawn on 24 hrs notice on 10th, increased to
% 48 hrs notice on 13th.  Men get some rest but in the background a number
% of conferences occur in preparation for Goodwood.  Vehicles are likely
% being maintained and losses replenished in this time.  2nd Army's being
% reorganized and a CO's conference on battle lessons occurs on the 13th.
%cite 27th WD


% talk about how this is often seen as an interlude but life continues.
% first bread ration issued on 11 Jul from 35 Fd Bakery at Luc sur mer.
% 2 oz issued / man increased to 4 oz next day --- men would have eaten 
% largely tinned food for a month.  Fresh food likely quite welcome.
%cite 90 coy 12 Jul

% Moreover, important work continues.

% Note the shortage of messtins and KFS at hospitals and the chronic shortage
% of tires.  Requirement to return amn casings, empty ammo cans, etc.
% 27 Bde Jul orders p37

% note how issuance of rations, units were overdrawing, and fresh rats were
% scarce.  Officers sent round to audit.  Talk about how this isn't just about
% REMF harassing front line troops, but a necessary part of military admin.
%cite 27 Bde Jul Os p38

% TRANSITION

\section{Goodwood (18-20 Jul 44)}
% ARGUMENT:  Use Goodwood to really start to show the centrality of logistics
% to ops.  No food, no amn, no POL, no fighting. (Goodwood detailed)
% Argue that trucks, not just tanks are what it means for an army to be mobile.

% STRUCTURE:  go day by day in the Run up and keep very chrono-narrative.  
% each day seems to sort nicely into themes; thus, use those themes and
% talk about them

% Point of operation  How do i want to manage this transition?  This is
% too abrupt

% THEME FOR 14TH:  Battle sustainment
% WngO (formal or informal) likely received for Goodwood 14th late morning or
% early afternoon.  90 Coy spends the afternoon organizing, then
% 14 Jul, over five hours (1700-2200), 90 Coy 
% converts the 13/18 rgmt amn & POL dumps IVO Ranville into dumps fit to 
% supply the Bde (likely approx 2 sq km in size).  
% dump consists of 38 lorry loads of amn to the 13/18th Rgmt dump enough
% to provide 97 rds/tk (a full load of amn) for the whole Bde.
% POL:  18 lorry loads of Pet & Derv (diesel) (7000 (either units or gal) of
% each).  This is enough POL to take the Bde 30 mi.  Coy does this via the BAD
% and PD.  Vehs fuelled as well before return to Coy HQ by 2400.  Speculate this
% was in receipt of WngO On 15th at 1900, Coy formally 
% establishes a det at the Ranville dumps to control R&I.  Det consists of 
% Capt Duffus, 46 ORs and 16 lorries.  Useful time to talk about fighting
% range and what it means.
%cite 90 Coy WD

% By the 15th, preparations for Goodwood occur in earnest across the Bde.  
% 3 Div holds another conference and, over the next few days, the Bde starts
% moving E of Orne likely via Ranville.  %cite 27th WD 15th Jul-ish
% Ranville det loads load 4 days compo into 6 lorries
% (enough to sustain whole Bde) as a mobile reserve.  8 lorries are devoted to
% keeping croc flamethrowers amned.  2 lorries leftover for odds and ends.
% Useful time to talk about importance of flexibility maybe?  
%cite 90 coy WD

% 16th:  RATIONS
% Bde HQ moves closer to Ranville but stays W of Orne.  Bde Cmdr holds 
% conference with Bde COs. %cite 27th WD
% As a result of this, all units W of Orne received 3 days extra compo, E of
% Orne, 4 days. %cite 90 WD
% Likely, whilst not stated in the WD, that 90 Coy made these regular 
% deliveries to the battalions.

% Total rations thus 4 days in vehicles, and 4 at 2nd line transport.  Ready 
% to move for 8 days.  Therefore, maximum planned speed for brigade's advance
% is approximately 30 mi (let's be realistic, likely 20 mi to account for 
% general movements) over 8 days by accounting for POL.  A good day of 
% advance would be 3.75 mi (6km)/day.  


% 17th:  Comms
% 27th HQ issues it's OpO for Goodwood today.  The Bde will help hold 8 Corps'
% left flank as 8 Corps breaks through.  Bde establishes it's battle net today.
%cite 27th WD and Orders
% 90 Coy receives two wireless lorries, one at Coy HQ, and 1 at the Ranville
% det.  Each lorry comes with 3 wireless ops.  %cite 90th WD
% Curiously, the OpO hardly mentions logistics.  Log is just expected to work.

% 18th:  D-day H-hr 0745
% Bde attack successful.  Objectives met around 1100.  At 2000 Coy HQ sends
% 3 lorry loads of PET and 3 of DERV (figure out range from 
% 18 loads = 30 mi) to Det Ranville. They're stranded, traffic is heavy,
% only E bound traffic permitted over the bridges.  %cite 27/90 WDs
% Start talking about what sustainment looks like.  If there are 6 lorries
% prepared, surely they'll be needed.

% 19th:  DE send 8 JU88 to attack Orne bridges.  Bde makes minor gains.
% ditto 20th.  Little change happens at this point as the front stabilizes.
%cite 27th WD
% 90 Coy uses this as a time to resupply units from Det Ranville.  
% talk about how ammunition moves.  The Ranville dump goes from X lorry 
% loads to 45 loads of amn when the dump closes and the amn is brought back
% to BAD Hermanville. %cite 90th WD for 25 Jul

% If Goodwood failed to break out, it wasn't logistical constraints.
\section{Post Goodwood}

% use it as a chance to talk about infantry transport --- Br infantry don't 
% have organic transport.  Also how more normal RASC transport units work

% 22nd, the Bde is informed they'll be broken up.  They get assigned to 
% 2nd Army's 22 Transport Coln at months end.  
% Det Ranville shrinks by 45 lorry loads of amn
% and the coy moves to Camilly (GR 933704).  Det Ranville is handed over to 
% different unit and rejoins the Coy.  Coy provides transport to Staff Yeo
% to Arromaches docks.  Coy returns the 2nd line amn holding of 30x 3ton 
% truckloads and 27 6-ton loads to BAD.  %cite 90 WD p.7 mention this holding
% earlier on too.

\section{} %not sure what to call it, something about being an ordinary
% 2nd line RASC unit


\section{Criticality of Supply}

% Ask the question of what would happen if not for these activities. 
	% Amn
	% Beans
	% Water (note location of WP)
	% POL
	% Everything else (white paint for turrets lest one gets shot lol)

% Invite community to see an army as a system and merely disconnected fighting
% groups.

% Discuss my gaps, I barely touch on maintenance units, etc.  What I could
% have talked about, what other sources say.  Consider Tiger or Panther without
% Maint units

% talk about sources, how 90 Coy WD doesn't really mention resupplies unless
% it's major, but just because it's not written, doesn't mean it wasn't done.
% by asking the question of who would do X, one starts to uncover the Y, and Z.
% talk about how entries like 6 Jun are very long --- likely because of the
% initial excitement.  The WD's progressively shorter and more concise as time
% wares on, likely because it's tedious and, "I want to go to bed!".  Supply
% logs are tedious.  Talk about how this study is actually quite constrained
% by sources in so far as a heavy dependence on electronic sources.  War diaries
% don't mention routine replenishments but it doesn't mean they don't happen.
% alas, I doubt they kept the waybills!


\section{Conclusion}

\newpage

\printbibliography

\end{document}
