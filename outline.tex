Outline:  60-120 Lines

Question:  Was the logistics of the British Army fit for purpose in supporting
the British Army in operations in NW Europe from 1944-1945.  

/*****************************************************************************/
Outline as Follows:

No indentation is a top level <h2> type heading / Latex \section{} unit.  
Each successive tab indentation reduces the heading by one level.  Thus, a 
single tab denotes <h3> / \subsection{} whereas two tabs denote 
<h4> / \subsubsection{}.  Constrained to no more than three tabs/levels of 
sectioning.

Comments will use either C style comments for embedded remarks (/* comment */) 
or Latex comments where the whole line after `%` will be ignored on output.

Outline begins on line 25 and will continue till line 85-145.

Output target length 50-70 pages. 

Note:  POL stands for petrol, oil, & lubercants.  
/***************************************************************************/

% Conops, perhaps use a narrative flow moving from before invasion for the
% prepwork and training, to the actual invasion for execution and adaptation,
% to transition around or after TOTALIZE?

\section{Introduction}

	% Importance of log, perhaps introduce ABCDE (ammo, beans, cold water, 
	% diesel, everything else).  Perhaps an "Imagine ..." line.  Maybe even
	% emphisize the importence of the 2nd S in SMESC.

	% Include thesis:  The critique that the British Army only beat the 
	% German army due to material superiority, and that the German Army
	% had better soldiers is groundless.  This is because, in modern
	% Wars (incl. ww2) it is material that matters.  That the British 
	% Army had good lgoistics is testimant to the quality of their
	% soldiers as supplies are inherently part of soldiering.  It is not
	% that the Germans had better soldiers but worse logistics leading to
	% their defeat by Anglo-American forces, it is that German soldiers
	% (officers really) were worse soldiers as they could not create a
	% serviceable supply chain.  How do you fight if you cannot eat or
	% if you have no bullets.  
	% /* perhaps I should make this more consise */

	% Set scope:  This is about the British *Army's* logistics starting
	% from ships to the units (TBH, maybe even from the beach, before the
	% high-tide line is a Navy problem)

	% Mission of Q branch

	% A tale of heroic pencil pushers and bean counters lol!

	% Useful quote to throw in maybe?  
	% ``The delivery at the right time and place of supplies required by 
	% an army in the field may frequently be the deciding factor in the
	% success of its operations''. 
	%cite No 27 trg crs p. 34 quoted fsr vol 1, s.29(3)

\section{Historiographical Review}

	% Complain loudly and long-windedly at the lack of discussion on log 
	% in /*insert list */  Consider subsections or flow of text?

	% Discuss the current logistical scholarship done
		% Namely historical work in 50s and Supplying War, Julian
		% Thompson, Great Feat...

	% Introduce Section

	% General observations:
		% Much work done on great men, tactics, the merits of German
		% Armour; and some work has been done on American logistics,
		% as well as logistical peculiarities like the mulberry 
		% harbours.  I fear little has been done on the actual
		% military administration of the war.  Indeed, the logistical
		% section of military history is fairly poorly written about
		% by hisotrians, some more work done military acadamies
		% reflecting its importance to them

	\subsection{On Normandy}
		\subsubsection{\textit{Clash of Arms}}
		\subsubsection{\textit{Overlord}}
		\subsubsection{\textit{Fields of Fire:  Canadians in 
			Normandy}}
		\subsubsection{\textit{Montgomery and `Colossal Cracks':  
			The 21st Army Group in Northwest Europe, 1944-45}}
		\subsubsection{\textit{The Normandy Campaign 1944}}

	\subsection{On Logistics}
		\subsubsection{\textit{Supplying War:  Logistics from
			Wallenstine to Patton}}
			% Foundational in logistics scholarship, but, given
			% it's broad scope, it lacks depth
		\subsubsection{\textit{The Lifeblood of War: Logistics in
			Armed Conflict}}
			% Limited and tries to cover a lot of periods
		\subsubsection{\textit{A Great Feat of Improvisation}}
			% British but only really to shortly after Dunkirk
		\subsubsection{\textit{War of Supply}}
			% Cheifly American in the Med

% Start with Context %%%%%%%%%%%%%%%%%%%%%%%%%%%%%%%%%%%%%%%%%%%%%%%%%%%%%%%%%%

/* Figure out where to incorporate the fact that the British/Canadians focused
on firepower over manpower.  This means materiel is critical --- A is for ammo,
B is for beans, C cold water, D: deisil, E-everything else... */

/* Do I want to expand to include things like traffic control?  Traffic jams
on Sword Beach may have made the Br fail to capture Caen on D.  10m of dry
beach between water and sea wall at high tide.  Perhaps an MP or two would
have solved the issue.  IIRC, RAF beach sqn dealt with it.  Was this a critcal
oversight?  Not a lack of tenacity or anything else, but a good, old fasioned
traffic jam vis-a-vis Toronto at rush-hour caused the failure to take Caen?
lol --- what a way to win a war! */

\section{Interwar Development} 

% Can I mostly lean on GFOI for it? Have a few things WRT mechanization and
% whatnot too. 

\section{Logistics Working Practices in Theory}

	
	% How supply chains work from dumps, and depots, to distribution, to
	% 1st line units.  

	% ``The principle of supply is that field units should always have
	% with them, or within reach, two days' rations and forage, and one 
	% iron ration, and that these stocks should be replenished by 
	% delivery, at a point within reach of the troops, of one day's ration
	% and forage each day.'' %cite No27 Trg Course p. 32, quoted in,
	% refs F.S.R. Vol. I. Sec107(I). %validate this

	% Key was placed on  on rabild concentration of force, incl sup, LofC
	% etc. to permit strategic use of force %cite 27 trg crs p. 32

	\subsection{The Structure of Supply}
		
		% 4 main areas, the BSA, L of C, Corps/GHQ area, Div area
		% goes from 

		%cite No 27 Trg Course p. 34ish (diagram's on 34)
		\subsubsection{The Base Supply-Area (BSA)}
			% The Docks
				% Bulk Petorl
				% Cold STorage

			% Main Supply Depot --- Field Bakery, DID

			% Petorl Sub Depot 

			% Base Marhsalling Yard (start of railway line)

		\subsubsection{L of C Area}

			% Einvisioned as a railway line, lorries will do 
			% just fine but this theory came at the dawn of
			% mechanization.  Point was to take stores from
			% the BSA and transport them near the combat zone.
			% It's basically a transprot area.  IT goes from 
			% just outside the BSA to, and including the railhead.

			% If distance from the BSA to the railhead is < 12
			% hr trip, BSA will control.  Else, there will be
			% a regulating station around 6 hrs from the rail
			% head. %cite no27 Winter Trg p. 32 5c
			
			% The railheads also had field supply depots, to
			% supply local resources.
			
			% If using trucks, it's 150 tons per 21 trucks with
			% old pat, 59 tonns across 12 trucks (5 supplies, 2 
			% ord, 2 R E, 2 stores, 1 RAMC)  railway trucks 
			% %cite No 27 course, p. 32 para 5

		\subsubsection{Corps or GHQ Area}

			% Start of what really starts to feel like the
			% field army and not just transport areas.
			% Transprt at thsi point is by road.

			% Seperate Chain for 'stuff' and POL.  POL is the
			% Coprs Petrol Park (25 Mi for Div, 75 mi for C.T)
			% Explain the logic of this.  POL used containerized
			% distribution rather than tankers for end-point
			% distriubtion.  Used flimsies or jerrycans

			% Supply Coln. HQ handles the rest of it.  Here, 
			% stores are bulk broken and sent further forward.

		\subsubsection{Div Area}
			
			% POL side, POL is held at the Div Petrol Coy 
			% and should have a 50 mi. supply (chk this).  
			% The div Pet Coy will set up petrol points and
			% unit transport will collect POL from those points.

			% General Supplies are handled at a rondezvous 
			% where supplies will be delivered at delivery 
			% points via meeting points.  Range from the L of C
			% area to the DPs is 30-40 mi.  DPs are where 
			% individual units come to collect stores.
	

	\subsection{Warehousing}

		% Especially in those early weeks when a lot of it was dumping
		% rather than warehousing per-se.  

		% DID (maybe??) 3000 Tons/acre= gross stacking area.  x 4-10
		% for expansion %cite No 27 trg 32 (this might be a general
		% rule too)

		\subsubsection{Base Supply Depots}
			% Located near Base Marshalling Yard, outside docks
			% to permit expansion and dispersion.  Consider `water
			% light, telephones, good roads, office accomodation'
			% when selecting the location. Area = Strength X 
			% stock X weight one ration / (unintelligable) = 
			% 8.5 sq ft /ton = stacking 
			% +50\% for stock spacing = gross stacking area X 4
			% to 10 for expansion, etc.  (that's a quote)
			%cite 27 trg course p.32, 4


	\subsection{Distribution and Transport}
		% Intermixing of the RASC and RAOC for moving the 'stuff' vs
		% transporting the 'stuff'

		% 'Harbours', how increadably quaint lol!

		% Discuss the role of MPs, you can't have supply without 
		% transport, and you can't have transport without a traffic 
		% light (MP)

	\subsection{Delivery} 
		% this may be rather light.  IDK if sources avail for action 
		% line unit's distr.  One feels they tend to assume the stuff 
		% just arrives, then hands it out.

	% where to talk about POL?  %cite No 27 trg course p. 35

\section{Logistics Working Practices as trained}

	% Describe the structure of the supply chain

	% Dump specifications
	
		% Land use
	
		% Capacitities

	% Qty of men involved

	% Transport options

\section{On Normandy}

	% Wasting Army --- Br could not sustain losses in men --- there were 
	% no more men to allow to die if you still wanted an army. 
	% \autocite[%cite]{collossal-cracks}

	% Allied preference for munitions over men and it's consequences.
	% How material works, need for arty, etc. %cite Gunfire or perhaps
	% somethign else?

	% Arty requirements and what it entails WRT 3 tonne lorries.
	%cite Gunfire

	% What's a good deapth to take it?  I should at least give a 
	% general description of it; however, how much can I take as
	% common knowledge in the field.

\section{Logistics Working Practices as Executed}

	% Routine nature of supply WRT A B C D E

	% Exceptional 

		% white turrent tops for AG recognician.

		% Supporting the paras

	% Length of supply lines approximate

	% Ammo shortage during first few days

	% Porpoise sledges

\section{Maybe add a section on how log is taken foregranted?}
% Maybe make this a subsection?  I wonder if I can find any examples though.

% Water, food, etc. have very few mention in the sources.  Some discussion
% made in AGFOI

% Discuss the hard requriements (meals/day, water rations 
% (this is in many OpsOs)

	% Find in RCASC trg docs, it's somewhere there --- perhaps when my
	% eyes stop flashing!

% Maybe discuss why it's important it doesnt fail?  I'm hungry, dehydrated,
% out of ammo, and freezing, what good am I?

	% Just how much evidence do i need?  So much of this is pointing
	% out the obvious, but, at the same time, unless you've been outside
	% hungry and shivering, I wonder if your ealize just how miserable
	% it is.

% the British fondness for artillary is quite useless without ammo --- a 
% 25-pdr isn't even a good paperweight, it'll only crush the paper!

	% Find rounds per hour, rounds per lorry, etc.  See if I can find 
	% surface area requriements for dumps to avoid a rapid incineration
	% of the dump when subjected to enemy fire.

\section{Failures and Successes}

% there's 90 coy RASC on D - D+3ish

% Coy maintained the 6th airborn bridgehead as well as the 3rd Br Infantry 
% Div and 27 Armd Brig for multiple days on Overlord.  They ferried 
% and arranged all transport on the D Day off Sword beach.  No other 2nd line
% unit avail.

% but I wonder if this merits a whole section. 

	% Only 2nd line RASC unit to land on d-day.  Supported 6 Airborn & 27
	% armd bde

	% planned was that 27th would be supported by by 3 div for POL.  Due
	% to extra shipping space, 1 platoon of 90 coy was to land with the
	% 105 amn for 3 div, amn, mines, POL, water, RE stores for 6 para 
	% (abv is quoteish)  Last minute, more space became avail, 13 3-tons
	% landed D-day 1st tide to support 27th with POL.  THey became first
	% to land at 1300.  POL would have been in tins and not bulk. 
	% POL, water, RE stores for 6 airborne.  Original lorries from 3div
	% with 27th's POL didn't arrive till D+2, tanks aren't useful without
	% POL.  13 lorries of B Pl thus supplied the whole bde. Check support
	% range in RASC traning docs.  Virtually no breaks until D+4 C pl 
	% did 6 airborne.  6 Airborne started resupply at 2300, returned to
	% harbour by 0430

	% Veh capacity 11 on D, 33 D+1, 23 D + 2, 20 by D + 4.  Decline due
	% to casualties.  Transprted 5-6 lorries to carry 3rd Br Inf Div 
	% re-inforcements, sups for 27 Armd Bde (quoteish) Experience useful
	% as 6th Abrn supply pers were unable to communicate requirements
	% or forecast them accurately.  As such, veh moved constantly between
	% the beach and 6th airborne.  Beach congestion very heavy.  
	% transport so short that no trip could be wasted.  Lorry drivers 
	% even were used as a quazi recce to ensure they always knew what
	% would be needed.  This was reported to Pl Harbour as they passed.

	% Trip from Coy Haroubr to beach and back was a 2 hr trip.  
	%cite Jul hist rep sheet 5

	% As it happens, 6 para supplies landed at Juno and were stuck till
	% D + 4.  

	% Harbour planned for W of Hermanville not secured till d+2, 
	% temp harbour E of Colleville 2-3 mi from enemy pos.  
	%cite june-history-report.pdf(4)

	% Chk pg ct, perhaps elaborate, perhaps not.

	%cite sourcery/wd/27armd/90coy/06/histrep p1

% Maybe traffic control and overlord fire control too?  Shoudl I divide into
% supply and transport?  

	% If this, what about MPs directing traffic under fire --- sounds
	% rediculous when you put it like that but it was certainly done.

	% Include ambulance drivers?  It feels a little too otuside the remit
	% though

\section{Transition to Future Ops} % Consider omission after Sep 44?  
% maybe swrod beach

% could I perhaps use some records from medal citations for this?

\section{Conclusion}

/******************************************************************************
END OF OUTLINE
******************************************************************************/

/* Header as fol *************************************************************/

\documentclass[noraggedright]{turabian-researchpaper}

\title{}  
\subtitle{}

\date{\today} % alter?
\author{Albert Duan}

% Sets UK Date format
\usepackage[british]{babel}

% Autopunctuation outside of quotes
\usepackage{csquotes} 

% Auto link URLs, hyphenate long URLs in appearance.
\PassOptionsToPackage{hyphens}{url}\usepackage{hyperref}

% Find pkg to wrap long URLs

% Sets Bibliographic Style 
\usepackage[notes]{biblatex-chicago}
\addbibresource{} % add path

/* Library includes */
