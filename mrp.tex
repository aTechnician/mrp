\documentclass[noraggedright]{turabian-researchpaper}

\title{}  
\subtitle{}

\date{\today} % alter?
\author{}

% Sets UK Date format
\usepackage[british]{babel}

% Auto-punctuation outside of quotes
\usepackage{csquotes} 

% Auto link URLs, hyphenate long URLs in appearance.
\PassOptionsToPackage{hyphens}{url}\usepackage{hyperref}

% Find pkg to wrap long URLs

% Sets Bibliographic Style 
\usepackage[notes]{biblatex-chicago}
\addbibresource{src/sec.bib} 
\addbibresource{src/pri-uk.bib}
\addbibresource{src/pri-can.bib}
\addbibresource{src/iwm.bib}


%%%%%%%%%%%%%%%%%%%%%%%%%%%%%%%%%%%%%%%%%%%%%%%%%%%%%%%%%%%%%%%%%%%%%%%%%%%%%%%
% Citation Aids
% All citation aids will start with capital letters for sake of namespaces

% For citing 27 course's Lecture No. 12 Petrol p 35-7.  Largely on bulk vs 
% container
\newcommand{\Petrol}{Precis of Lecture No. 12:  Petrol}
%cite Author is Maj W. P. Pessell RASC.  Figure out how to do it when LAC 
% website works.  It's on p 35 of my scan
% For use with {27course}

% For Supplies in War Lecture pt 2 by Lt.Col. P. L. SPAFFORD, 0.B.E.
% For use with {27course}
\newcommand{\SupInWar}{Precis on Lecture ``Supplies in War'', (Part II)}
%%%%%%%%%%%%%%%%%%%%%%%%%%%%%%%%%%%%%%%%%%%%%%%%%%%%%%%%%%%%%%%%%%%%%%%%%%%%%%%

% Formatting notes, do I want to divide this into multiple source code files?
% Outline:  60-120 Lines

% Question:  Was the logistics of the British Army fit for purpose in supporting
% the British Army in operations in NW Europe from 1944-1945.  

% /*****************************************************************************/
% Outline as Follows:

% No indentation is a top level <h2> type heading / Latex \section{} unit.  
% Each successive tab indentation reduces the heading by one level.  Thus, a 
% single tab denotes <h3> / \subsection{} whereas two tabs denote 
% <h4> / \subsubsection{}.  Constrained to no more than three tabs/levels of 
% sectioning.

% Comments will use either C style comments for embedded remarks (/* comment */) 
% or Latex comments where the whole line after `%` will be ignored on output.

% Outline begins on line 25 and will continue till line 85-145.

% Output target length 50-70 pages. 

% Note:  POL stands for petrol, oil, & lubricants.  
% /***************************************************************************/

%%%%%%%%%%%%%%%%%%%%%%%%%%%%%%%%%%%%%%%%%%%%%%%%%%%%%%%%%%%%%%%%%%%%%%%%%%%%%%%
% MARKS SET

% i	Introduction
% h	Historiography
% c	Conclusion
% m	Mark Set (this)
% w	Working Point
% f	WTF are we doing here!?
% 

%%%%%%%%%%%%%%%%%%%%%%%%%%%%%%%%%%%%%%%%%%%%%%%%%%%%%%%%%%%%%%%%%%%%%%%%%%%%%%%

% Conops, perhaps use a narrative flow moving from before invasion for the
% prep work and training, to the actual invasion for execution and adaptation,
% to transition around or after TOTALIZE?

% TLDR:  if you accept that the British CA soldier was tactically worse than
% the Germans but were able to defeat the Germans by unloading an ungodly
% amount of steel and explosives over their head at the first sign of 
% difficulty, then the British logistical soldier was far superior.  A war 
% cannot be won by killing alone!

% Thesis: 

% L1:  Historical community is quite hard on the British for lacklustre
% results vis-a-vis armoured warfare and armour-infantry co-operation; 
% however, what this is missing is that the British Army --- Army, not 
% economy --- work quite well at a logistical level.  This is what, in the
% field, enabled the British Army to defeat the Germans despite tactical
% mediocrity.  Logistics worked to support flawed tactics.

% L2:  Criticality of the services to the effective waging of war.

% L3:  We ought to stop viewing armies as mere fighting forces.  Understanding
% them using a systems approach explains why we can win wars.  Merely 
% examining how an army can outfight its opponent at a tactical level misses
% the operational reasons the tactical level can even function.


% CONOPS:  What if I do this as a microhistory of 90 Coy?  I can use them as 
% a way to talk about the criticality of logistics.  What if I used them as
% the pivot around which an army (tbh, 27 Armd Bde, and 3 Div) can turn?  I 
% can use it to skirt around the wider issue that I don't have evidence for
% a macro view but I have 90 Coy's WD.  Alas, I can't find as many decorations
% for them as I would like to really make it personal.  Could I use examples 
% from other units as a `take Cpl Bloggins from X, note the work he did'?
% time to run git branch I suppose.  It also allows me to be narrative which
% is always fun.  Their 4 day history was already 7 pages, I could breath life
% into.  I can harp on about them for 8-10 x that, right? 1-2 pages 

% introducing (let's do it as if they hit a beach and they're driving a 
% convoy up the road to Benouville with that preload of supplies for 6 Airborne
% whence we pause to discuss what we currently talk about in the 
% historiography.   I have 5-10 pages on what other people have been writing 
% about Normandy, from there, perhaps another 5ish to narrow down on Sword 
% Beach, the whys, the objectives, who was involved etc. --- I should re-read
% return of martin guerre I think.  

	% so 20-30pgs here, I'll probably do more once I footnote it all.
% With this context, we return to the convoy and lay out the theoretical 
% framework by which logistics operates.  Depots, lengths of supply lines,
% how distribution happens, etc.  How this work integrates with the rest of
% the army.  We deliver the supplies and talk about the RV.  Let's also start
% talking about the arduous work over the next few days to supply 3 Div and 
% 27th Bde as there are no other 2nd line transport units.  As we continue 
% here, I can intersect the sources for 27th Armoured Bde to give context
% as to what 90 Coy was doing.  I suppose I"ll have to do some probables and
% perhapses.  We can talk about how 90 Coy supported the Bde until the Bde was
% disbanded.  Then, continue talking about how the Coy supported other units
% till Totalize maybe?  Then, have a final almost pre-conclusion discussing
		% Perhaps this takes 4-6 pages-ish?
% how, whilst we don't write about it, these operations would have been 
% impossible without the supporting arms and concluding with how the British
% ability to have tactical mediocrity and an unwillingness to spend lives
% meant that their logistics had to be good, really good.  

% Expand this to how armies function, and who this paper doesn't cover.  I 
% don't discuss workshops, clerks --- the whole of A branch actually --- MPs, 
% BADs, beach detachments, etc. yet they're all important. (5 pgs)  Maybe 
% paint a picture of the whole rear area and why it's important (landing 
% tickets maybe?).
% Tie it back to historiography and methods like  how a lot in the sources 
% requires you to intersect doctrine with the WDs to figure out what supply 
% units were doing --- making a meal dely isn't something that really gets 
% recorded in the WD, etc.  (5 pgs)

% Finally, conclude.

\begin{document}

\maketitle

\section{Introduction}

	%Estimate 2-3 pgs?

	% Set scene, we're in Normandy, 6 [time] Jun 44, we just landed, we're
	% driving from Queen Beach to Benouville to link up with 6 Abrn.  
	% you're X lorries are carrying preloads of ammunition, rations, and
	% other supplies for 6 Abrn.  They're holding the Anglo-American 
	% left flank from a possible German counter attack.  what if I put
	% it in the perspective of that Lt who did the initial recce?  You 
	% weren't originally scheduled to land yet but an accident of war
	% means here you are.  Describe more about what 6 Airborne is doing
	% you don't belong to 6 Airborne but are instead of 27 Bde, you're 
	% just tasked to help them on D day, their unit 2nd line unit will
	% arrive from Juno later /* Come back to this when discussing import */

It's 1630 hrs on 6 June 1944, Captain Foreman just arrived at his company 
harbour near Colleville.  An hour earlier, he and the 11 lorries of C Platoon
90 Company RASC (90 Coy) disembarked the LSTs they had been stuck on for the 
past six days waiting to cross the English Channel to support Operation 
Overlord, the Anglo-American invasion of Normandy 
France.\autocite[1--6 June 1944]{90wd}  Loaded in these 
11 lorries are supplies for 6 Airborne Division currently operating to secure 
the British left flank over the Orne.  These loads consist of `pet[rol], 
[ammunition], R[oyal] E[ngineer] stores, and water', stores vital for the 
paras of 6 Airborne Division to resist a German counter 
attack.\autocite[S \& T Report (June History Report) p 4]{90wd} 
%cite check this later
Alas, despite the urgency of these stores, Major Cuthbertson, 90 Company's 
Officer Commanding has yet to make contact with 6 Airborne so C Platoon has 
little to do but wait for contact to be 
established.\autocite[6 June 1944]{90wd}
Thus, doubtless, the men of C Platoon, 90 Coy would have dismounted their 
lorries and pause.  Likely, they would have appreciated being once more
on dry land having spent the last few days being bounced up and down in the
English Channel.  A few kilometres away, the men of the 6th Airborne Division,
the 3rd British Infantry Division, and 90 Coy's home brigade, 27th Armoured
Brigade were, in the case of 6th Airborne, guarding the British flank, or
in the case of 3 Div and 27 Armoured Bde, pushing inland to try to reach
Caen.

The men of 90 Coy are perhaps not what you would initially think of when you
 think of soldier.  Whilst the majority of the men in 6 Airborne and the 3rd 
British Infantry Division were infantry, trained to kill the enemy with
their enemy with rifles, grenades, and bayonets; and whilst most of the men in
the armoured divisions were trained to kill with tanks; the men of 90 Coy
consisted mostly of drivers \textit{delivering the goods} wherever and whenever
they were needed.  Thus, we will follow Major Cuthbertson and 90 Company RASC
as they cross the English Channel and land in Normandy.  We will their journey
through Normandy, not fighting per se, but ensuring those who fought, had what
they needed to be able to fight as they attempted to capture Caen and close
the Falaise pocket.  Along the way, we will examine how the British Army
structured logistics at an administrative level, before joining 90 Coy as they
support the 27th Armoured Brigade as they partake in the battle for Caen.  
After 27 Armoured Brigade is broken up at the end of July, we will see how 
90 Coy integrated into a larger and longer supply column as they support units
through Normandy.  Following this , we will have a brief discussion on the 
historical method and how it applies to military logistics.  

Logistics, you may well argue, is boring:  all the boxes, cans, crates and
barrels filled with the minutiae of war.  On top of that, the mountains of
paperwork that go into it.  The Operations Orders, Routine Orders,
Administrative Orders, Standing Orders, waybills, lists, calculations, 
and forms --- so many indeed that they are often referred to by their form 
number instead of the name.  Why bore you with a story regarding this when 
instead, we could examine the fighting.  We could, like many have already,
examine the successes and failures (admittedly, mostly failures) of British
infantry-armour co-operation.  We could examine the aggression --- or lack 
thereof displayed by British troops.  We could examine any number of very
interesting topics like Allied inadequacies in armour, Montgomery's 
personality, tactics, the use of firepower, and many other topics that are
likely far more engaging to discuss.  Why then discuss the military side 
show that is logistics --- a mere means to an ends?  

The reality is that the distance of the combat arms for logisticians as lazy
people telling them to sign for x, y, or z is somewhat misplaced as the 
operational effectiveness of the combat arms is contingent on the skill of
the logistician.  Take 100 000 of the worlds finest combat troops and cast them 
against your enemy without logisticians.  In a day, they may run out of 
ammunition, in two or three they will be thirsty and their motor vehicles
will be out of fuel, and in a month you will have a famine.  

	% THESIS
	% You know what you're doing is important, without these supplies and 
	% the war will ground to a halt.  

	% You wonder what the historians who write about these events 80 years 
	% in the future will think about it all, what drives your focus?

	% Maybe have a bitter reflection, it's always the fighting troops or
	% the generals who get the cheers.  No-one applauds the cooks!  Use
	% the perhapses of history for this

	% Shift to that historiographical frame:  With the benefit of 
	% hindsight, we know that the drive to Caen would not be a quick 
	% drive, but as a logistician, one asks oneself, was there a 
	% significant logistical constraint or was it due to something
	% else?

%e reword the this a little to emphisize my thesis in the context of expending
% material / men
Of course, the vital efforts of the 6th Airborne Division and the other 
fighting troops of the British Army in Normandy have been fairly well studied.
Extensive critiques and justifications have been made on British 
infantry-armour co-operation, the aggression --- or lack thereof --- displayed
by British troops, Allied inadequacies in armour, Montgomery's personality,
tactics, vs firepower, etc.  %e avoid use of etc here
In short, we often discuss what went wrong or 
how we fought; however, what we often ignore is the critical question of what
enabled us to fight. The work done by troops a few kilometres behind the front 
line is generally ignored as a side-show; yet, the work of ensuring the combat 
arms are well supplied with all the minutiae of war from ammunition, to food, 
to water, and other general supplies is what will make or break an army. Thus,
in light of this gap, I hope to argue for the centrality of logistics in the
British preference to expend firepower rather than lives.  The British Army
seems quite helpless compared to the might of the Wehrmacht until one looks
at this Army from a systems approach.  It is however, this systems approach
that reveals the British Army's strengths.  %e I hate this para!

% add section on structure of argument --- maybe later once I un-f that para
To examine the centrality of logistics in British Army operations, we will 
follow Major Cuthbertson
and 90 Company RASC as they work their way across the English channel, landing
in Normandy and following them as the units they support attempt to capture the
city of Caen, and we will examine their role in the closure of the Falaise 
Pocket in August.  Along the way, we will first examine how the British Army 
structured logistics administratively, before joining 90 Coy as they support
the 27 Armoured Brigade as they partake in the Battle for Caen.  
After 27 Armoured Brigade is broken up at the end of July,
we will see how 90 Coy integrated into a larger and longer supply column as
they support infantry units through Normandy. Following this, we will have 
a brief discussion on historical methods and how they apply to military 
logistics.  %e I also hate this para, later problem!


	% \subsection{Clarification of Terms} % maybe omit?
		% How I use D-Day to actually mean D Day and not 6 Jun 44
\section{Historiographical Review}

%ep Show how what I'm doing is idfferent from standard military history,
% explain how non-eliete logistical labourers contribute to 
% war.  

% Note:  important how there's no Tim Cook without a stacy even if Stacy
% is a little boring

	% `Of course we know that 80 years after the fact, historians...'

	% Complain loudly and long-windedly at the lack of discussion on log 
	% in /*insert list */  Consider subsections or flow of text?

	% Discuss the current logistical scholarship done
		% Namely historical work in 50s and Supplying War, Julian
		% Thompson, Great Feat...

	% Introduce Section

	% General observations:
		% Much work done on great men, tactics, the merits of German
		% Armour; and some work has been done on American logistics,
		% as well as logistical peculiarities like the mulberry 
		% harbours.  I fear little has been done on the actual
		% military administration of the war.  Indeed, the logistical
		% section of military history is fairly poorly written about
		% by historians, some more work done military academies
		% reflecting its importance to them


%e massage a transition into this

The Battle of Normandy is of course, a well studied topic.  Much has been
written on this battle from books on the Second World War at large to 
publications that focus squarely on operations and tactics in Normandy.  What
is often ignored however is the work that went into that which enable 
operations to take place.  There has been some, but relatively limited work
on British logistics and, where extant, that work tends to be quite marginal.


%e I'm feeling narative and I don't wnat to sum up these books again.  Later
% problem!

Of course, books on the Second World War are quite common and they often
grapple with concepts such as \textit{Why the Allies Won} by Richard Overy
who notes the importance of the various technical and material advantages the 
Allies had over their German counterparts in explaining the successes of 
Anglo-American forces at a fairly high level.%cite
From there, historians have narrowed down on specific components of that
success.  

There is generally a debate that rages between British and American scholars
of the war in the merits of Montgomery as a C-in-C Allied Ground Forces as
well as a C-in-C 21st Army Group.  Russell Hart in \textit{Clash of Arms} takes
the view that the British appeared overly cautious and lacked initiative whilst 
the Americans by contrast often seemed quite daring.%cite
He argues that the British were generally slow and
were uninovative.  Stephen Ashley Hart in \textit{Montgomery and ``Colossal
Cracks''} takes a deeper look at this.  He argues that the British were 
generally more slow and cautious than their American counterparts, but that 
this slowness was justifyed as the British were unable to wear casualties from
both a supply and moral perspective.  Put plainly, Stephen Hart argues that by 
1944 the British had run out of men --- indeed, just scraping together the 
numbers to launch Overlord took a significant amount of effort\autocite[56-7]
{cracks}
--- and that, even if the British had the men to spare, the British soldier 
and public would not accept what they 
viewed as needless losses.\autocite[24-5]{cracks}  
Montgomery, Stephen Hart argues, was keen to maintian the `fighting spirit'
of the British soldier.  Thus, Stephen Hart argues that, whilst the British
methods were slow and methodical, that the British could afford to do so.  
%e add more books

The scholarship also goes into areas of greater specificity.  Books like 
Christopher Yung's \textit{The Gators of Normandy} and Craid Symond's 
\textit{Neptune: the Allied Invasion of Europe and the D-Day Landings} 
look into the seaborne components of Overlord.  \textit{Gators} very much
offers a perspective that principly focuses on the naval component of 
Overlord --- a narrative so often ignored given the general focus on 
leg infantry, airborne infantry, and armoured warfare.%e confirm

%%%%%%%%%%%%%%%%%%%%%%%%%%%%%%%%%%%%%%%%%%%%%%%%%%%%%%%%%%%%%%%%%%%%%%%%%%%%%%%
% Stop Work Point 11 2358 Sep 25
%
% Adding stuff on historiography.  Need to re-read parts of Neptune and Gators
% for this section.  Prior to this, was working principly on pre-Charnwood 
% activities.
%
% Adding stuff from the June Admin Os
%
% Add jerry cans common by D day (IWM film A70 70-1 8:30)  Use of human 
% chains also common (9:40) (10:55) jerry cans stacked to head height
%
% ADM 1995 10:05 POL shore tanks
%
%
% Current point 14-18 June.  
%
% Connect the D-Day supply difficulties to mention the historiography. Talk 
% how they talk about fighting but are blind to how razor thin the armoured
% war was.  They were very close to not being able to be fuelled.  Ref Armoured
% Campaing in Normandy index enteries for 27th armd Bde (at end of Br units)
%
% Ranville, wrote the transition, nwo talk about the combined operations 
% between 6 Airborne, 27 Armd Bde, and 90 Coy, but first, I need a break!
%
% dont' forget to metnion how supply evolves but plans work as a framework.  
% take Hermanville how the Brigade ammo dump was conviniently on land of BMA 
% Moon's amn dump or Ranville in the pre-conclusion
%
% Crosscheck citations for DHH.
%
% I should probably define 1st line, 2nd line, and 3rd line units sometime.
%
% Preparing to talk about what happens to the company after the 27th is 
% disbanded.  Starting work on August diary.  Talk till end of Aug to show
% other side of log.
%
% Flesh out \section{Criticality of Supply} Do I want to make the 
% historiographical section a subsection or it's own section?
%
% Do I want to use more subsections?  I like how they organize.
%
% Working note, fleshing out the analysis more
%
% Should flesh out the explanation of sup as sys in earlier section
%
% Wondering what to do about the sectioning where I talk about the structure
% of supply. 
%
% Play to the theme that there is so much more to war than fighting
%%%%%%%%%%%%%%%%%%%%%%%%%%%%%%%%%%%%%%%%%%%%%%%%%%%%%%%%%%%%%%%%%%%%%%%%%%%%%%%
	\subsection{On WW2} % there's got to be a better name than this
		\subsubsection{\textit{Britain's Other Army:  The Story of
			the ATS}}
			% This feels kinda awkward to put here
		\subsubsection{\textit{Why the Allies Won}}

	\subsection{On Normandy}
% reorder this, chrono flow or based on argument?

% really flesh out the common critiques and defences of the British Army, 
% that it was slow to develop, under manned, unimaginative, morale problems,
% materiel over lives

		\subsubsection{\textit{Clash of Arms}}
			% Argues that the British were slow and failed in
			% innovating.  
		\subsubsection{\textit{Overlord}}
		\subsubsection{\textit{Fields of Fire:  Canadians in 
			Normandy}}

			
		\subsubsection{\textit{Montgomery and `Colossal Cracks':  
			The 21st Army Group in Northwest Europe, 1944-45}}
			% manpower Constraints
			% Morale problems Be sure to emphasize both of these
			% as supplies are critical to this
		\subsubsection{\textit{The Normandy Campaign 1944}}
		\subsubsection{\textit{Gators of Neptune: Naval Amphibious
			Planning for the Normandy Invasions}}
		\subsubsection{\textit{Neptune:  the Allied Invasion of 
			Europe and the D-Day Landings}}
		\subsubsection{\textit{From the Normandy Beaches to the 
			Baltic Sea: The North West Europe Campaign
			1944-1945}}
		\subsubsection{\textit{Feeding Mars:  The Role of Logistics
			in the German Defeat in Normandy, 1944}}
			% Toss this in Normandy or Log?

	\subsection{On Logistics}
% reorder this

%% Do I wanna put the books for WW2 logistics here or in WW2?  I'm tempted
% to concentrate it here but I also like the idea of keeping something from
% the field of military science and not history separate --- military science
% cares much more on actually executing operations.  Split this into two 
% subsubs and make the rest subsubsubs?  One for history, one for less so?
% Could also compress, maybe compressing and not sectioning this section 
% will flow better.  In any case, I wonder if it's better to chk pg ct

		\subsubsection{\textit{Supplying War:  Logistics from
			Wallenstein to Patton}}
			% Foundational in logistics scholarship, but, given
			% it's broad scope, it lacks depth
		\subsubsection{\textit{The Lifeblood of War: Logistics in
			Armed Conflict}}
			% Limited and tries to cover a lot of periods
		\subsubsection{\textit{A Great Feat of Improvisation}}
			% British but only really to shortly after Dunkirk

			% Use it to talk about how supply developed during
			% interwar years, namely, motorization.

		\subsubsection{\textit{War of Supply:  World War II Allied
			Logistics in the Mediterranean}}
			% Chiefly American in the Med
		\subsubsection{\textit{Supplying the Troops:  General 
			Somervell and American Logistics in WWII}}
			% Very much a great man history.  Logistics in the
			% form of a biography
		\subsubsection{\textit{Military Logistics and Strategic 
			Performance}}
		\subsubsection{\textit{The Story of the Royal Army Service
			Corps}}
		\subsubsection{\textit{Logistics and Modern War}}
		\subsubsection{\textit{Logistics Diplomacy at Casablanca: 
			The Anglo-American Failure to Integrate Shipping and
			Military Strategy}}
		\subsubsection{\textit{Strategy and Logistics:  Allied
			Allocation of Assault Shipping in the Second World
			War}}
		\subsubsection{\textit{The Science of the Soldier's Food}}
		\subsubsection{\textit{D Day to VE Day with the RASC}}
	
		

		

	\subsection{Tools of the Trade} % maybe rename this

	% The historiographical Gap.  Introduce an inattentiveness to log
	% here, that the materiel advantage is contingent on getting this
	% right.
	
	\subsection{A Note on My Sources}
		
		% Heavy reliance on digital records

		% paper records from Canadian sources owing to funding but
		% argue for the soundness of the methodology anyways given
		% how Canadian stuff is of Br doctrine

% Start with Context %%%%%%%%%%%%%%%%%%%%%%%%%%%%%%%%%%%%%%%%%%%%%%%%%%%%%%%%%%

\section{Overlord as Planned} % rename this

	% We return to our convoy, you're thinking and a bit day-dreamy 
	% driving through the French countryside thinking about the 
	% magnitude of what you're actually doing.  DO I want to go this
	% tone??

	% Overlord & Neptune:  Mission, invade France, push the Germans
	% across the Rhine sooner or later

	% Division of beaches, start from the W and move E.  How beaches
	% are subdivided.  Start thinking about the beach exits maybe and
	% why they matter?  Maybe reflect on the massive traffic jams?

	% We're intersted in Sword beach.  Over the Orne, Paras of 6 Para Div
	% holding the Left Flank.  From Sword Beach, 3rd Br Div was assigned
	% area from Swrod beach to Caen.  In this region, we have 27 Armd Bde.  
	% they provided tank support for 3 Br Div.  90 Coy was 27 Bde's 
	% 2nd line transport coy.  It basically provided the Bde's logistical
	% support.  On D-Day, ___ pl 90 Coy was also tasked to run supplies to
	% preloaded supplies to 6 Para over the river.  6 Para's 2nd line
	% transport was landing at Juno.

	% Talk about the 'bloody army', Q branch, A branch, and G branch.

Op Overlord was made up of a number of smaller operations.  The seaborne
landings were part of Op Neptune.  This was the operation that established a
50 km wide logistical beachhead in Normandy.  Neptune divided this section of
Normandy coastline into five discontinuous beaches.  The Allied right was 
anchored by Utah beach on the Cotentin Peninsula and the Allied left was 
anchored by the River Orne and the Caen Canal at Sword beach.  Between
these flank beaches was Omaha, Gold, and Juno beach.  The Americans were 
responsible for Utah and Omaha, whilst Anglo-Canadian forces were responsible
for Gold, Juno, and Sword beaches.  Each beach was subdivided into a 2 -- 4
sub-beaches and assigned a letter from A to R.  This study will primarily 
concern itself with the affairs of the troops of the 3rd British Infantry 
Division and 27 Armoured Bde that landed at Sword beach, specifically, Queen 
beach.  

This study will also concern itself with the work done by 6th Airborne
Division as part of Op Tonga.  Their objective was to execute a series of
airborne landings East of the River Orne, Caen Canal, and Sword Beach to 
secure the British left flank.  They were also to capture the only bridge 
crossing
these water features North of Caen along a road running between Benouville
and Ranville.  All this was to be done during the night before the forces of
Op Neptune landed.  For approximately six hours, the paras of 6th Airborne
would be cut off.  Once the British landed at Sword beach, they would push
inland, to Benouville, cross the bridges if they were still intact, and
reinforce and resupply 6th Airborne.  That is how the 11 lorries of C Platoon 
90 Coy finds itself waiting in Colleville, around 4km away from Benouville
waiting for their CO to link up with the Paras so that C Platoon could 
resupply 6th Airborne who would likely be running low on stores by this point.
C Pl would then keep the paras supplied via Queen Beach until 6th Airborne's 
RASC unit could take over on D + 1 after landing at Juno.\autocite[1]{90wdjun}

By 1800, C pl made contact with the Paras and, as the Paras had successfully
captured the Orne and Caen Canal bridges, C pl was able to replenish the 
depleting ammunition of 6th Airborne by 2300 hrs on D - Day --- a five hour
job.  As 6th Airborne's area of operations had yet to be fully secured, the
drivers of C pl faced sniper fire throughout the day.\autocite[6 June 1944]
{90wd}

Not all of 90 Coy landed on D - Day however, whilst A and D Pls stayed in the
UK to be brought across the channel on % date
B Pl landed on D - Day.  
Their tasking to simply support 27 Armd Bde primarily in terms
of their fuel requirements and to otherwise keep the Bde supplied.  Their 13
lorries were mainly loaded with fuel for the Bde's Sherman tanks.  Alas,
Due to the heavy shelling of Queen Beach however, only 9 lorries actually 
landed by 1200 hrs.  The lorries that landed proceeded to the 27 Armd Bde's 
A Echelon Area in Hermanville-Sur-Mer and would quickly be put to work keeping
the Bde supplied with fuel and ammunition.\autocite[6 June 1944]{90wd}
Hermanville, situated along the main road departing Queen Beach --- location
of the Beach Sector Stores --- rapidly became 90 Coy's control point where
vehicles would check in before proceeding to the beaches or to the units.

As a point of
curiosity, you may have noticed how B Pl was not preloaded with ammunition.  
This was because the Bde brought their own ammunition ashore firstly with
the ammunition they carried in their tanks, but also with the ammunition 
they towed behind their tanks in \textit{Porpoise} sledges.\autocite[2--3]
{90wdjun}  These sledges
would be released shortly after the tanks made it ashore.  Collecting the
ammunition in these sledges also became one of B Pl's tasks in the first
hours of the invasion.%cite 
%e assault division suggests they were quite bad

Perhaps as a happy co-incidence, Neptune had failed to meet it's D Day 
objective of pushing all the way to Caen --- an optimistic goal anyway.%cite
This meant that supply lines were shorter than planned which doubtless 
decreased the stress on the 9 lorries of B Pl.  
It is difficult to understate how heavy the 
fighting was.  Indeed, there were many instances where tanks were replenished
with tanks still `in their forward positions'.\autocite[2]{90wdjun}
This single under strength platoon was trying to keep a whole brigade supplied.
Tasks which would ordinarily been reasonably simple tasks were now incredibly
onerous.  Take for example the task of refuelling and reammunitioning the tanks.
What should have been a simple task done at the end of each day to ensure the
Brigade was ready for the next day's operations became a night long ordeal 
requiring the initiative of the 9 lorry drivers of B Pl who had to understand
the requirements of their client unit before returning to the beaches to try 
to obtain the critical stores required by their units.  It was paramount that
these drivers not only knew what was needed, but the priority of what was 
needed in the event that there were insufficient stores available to meet
an urgent order.  This way, lorries were always moving and stores were 
always flowing.  Fortunately, by nightfall on D - Day, a small Brigade supply
dump was beginning to form in Hermanville --- an act that would logistics 
chains.  Even still, this put a great strain on the men who were
worked day and night until D + 4.\autocite[2]{90wdjun}


% insert a transition into how B Pl was running around from BAD to units via
% Hermanville
Thus was the dispositions 90 Coy on D-Day, two Pls would make their way ashore:
onto support their parent unit, 27th Armd Bde and one help the Division to
their left --- 6th Airborne --- until their own RASC unit could make it. Here,
one can begin to see the role of 2nd line transport companies such as 90 Coy.  
They form the final interface between the wider supply system and the fighting
units ---  it is these units that \textit{deliver the goods} --- however, how
did these 90 Coy interface with the rest of Army?  
% mabybe rethink this transition


\section{The Supply Chain in the Field} 
% move to Hermanville, 90 Coy's control point in running convoys

Whilst admittedly, the supply system on D - Day did appear somewhat improvised
and ramshackle, there was good reason for this.  Because the British failed to
advance as far forward as planned, the supply dumps that were to be set up all
along Sword Beach failed to materialize in the same way as planned.  Still,
the logisticians of the British Army tried to beat a formal planned system
into an effective supply chain however much improvised.  It is worth recalling
that, even without additional planning, the British Army's baseline doctrine 
included a supply chain.  This was after all, an army that could expect to be
deployed to not just fight a large, European Army, but also fight small wars
across vast stretches of the British Empire.  To do so, the British Army already
had an organic logistical capacity that Overlord adapted to its use.  
At it's core

\begin{quotation}
	 The principle of supply [in the British Army was] that field units 
	should always have
	 with them, or within reach, two days' rations and forage, and one 
	 iron ration, and that these stocks should be replenished by 
	 delivery, at a point within reach of the troops, of one day's ration
	 and forage each day. %cite No27 Trg Course p. 32, quoted in,
	% refs F.S.R. Vol. I. Sec107(I). %validate this
	
	% add remark on fuel range
\end{quotation}

Moreover, as the British Army was fully mechanized by the Second World War,
it was the aim that all vehicles would have full petrol tanks at the end of
each day. To enable operational mobility, 2nd line transport was also to have 
immediately available, an additional 50 miles of fuel; and 3rd line transport,
a further 25 miles instantly available for use.\autocite[\Petrol][s 3]
{27course}
Of course, it is unlikely that this exact fuel holding was available on 
D - Day;
however, this was the standard the British Army would have expected.  These
principles meant that, at any one point, the British Army was expected to be
able to advance independent of it's bases for slightly over 75 mi over the 
course of three days.  Thus, this formed it's maximum operating range.

Of course, it is sub-optimal for an Army to operate for long without access to
its supply chain so, to support the Army, the supply chain was broken up into 
four main areas, ordered from furthest to nearest the front line, 
the Base Sub-Area(BSA), the Line of Communication Area (LoC), the Corps or 
GHQ Area, and finally, the Divisional Area.  Those depots that 90 Coy went to
along the beach?  Those were Beach Sub-Areas (BSA).  

\subsection{The Base/Beach Sub-Area and Line of Communication}

In the first days at 
Normandy, it appears that Beach and  Base Sub-Areas were treated as one and
the same.  Whatever the `B' stands for, BSAs functioned as the British Army's
initial interface between sea and land.  The BSA had %was it within it?
the docks, the base railway marshalling yard, a main supply depot, a petrol
sub-depot, field bakery, and detailed issue depot. Cold storage was also 
available for rations such as sides of meat, etc --- of course, it is unlikely
that such niceties were available in the first days of the invasion, fresh
rations weren't even available for quite some time. %cite graph in 27course

The BSA would then theoretically interface with the Line of Communication
Area (LofC).  These were railway networks or truck convoys that transported
stores from the BSA to the field army.  Now, the supply lines in Normandy were
quite short, measuring in the ones or tens of kilometres.  It was simply 
unnecessary to have a strict LofC area per se.  The field army could simply
draw stores directly from the BSA --- the LofC area really is not necessary
until the field army is some distance away from the BSA.  The LofC would 
become necessary as the British Army advanced through France and into 
Germany.  As they went deeper, scheduled and intentional convoys to convey
the stores would become more useful in relieving the field army of such 
transport network.  

\subsection{Supplies in the GHQ, Corps, and Divisional Areas}

In any case, regardless of whether the Army was drawing stores directly from 
the BSAs or from the LofC, eventually,  Army would have to start drawing
stores.  To such ends, the Army was divided into two sections the 
Corps / GHQ Area and the Divisional Area.  Typically, the distance --- and 
thus, also depth of the Army --- from the LofC area to the delivery points
was 30 -- 40 mi (50 -- 65 km)  At the GHQ level, one begins to
see how the British Army sorted supplies.  POL and other stores were handled
in two theoretically separate systems.  In either case, it is at the GHQ level
that stores were bulk broken.  

Let's handle the general stores first.  Stores are delivered to the Supply
Column (Sup Coln) where stores are bulk broken.  Think of this bulk breaking 
with the analogy of a grocery store.  A grocery store may receive it's goods in 
wholesale, bulk form, but then repackage it into smaller, more usable units to
be easier to sell --- a retail customer may want 1 lb of almonds, not 1 ton
for example.  In the case of prepackaged stores, bulk breaking is more similar
to the procedure that occurs when a grocery store receives a palette of cereal
which is subsequently unpacked and loaded as single units on a shelf.  Thus,
the Sup Coln HQ can function as an interface where the Army's bulk handling 
meets it's piece handing functions.\autocite[\SupInWar][3]{27course}

\subsubsection{Petrol, Oil, and Lubricants (POL)}

Likewise, fuel could, at times be shipped in bulk initially however fuel for
the British Army was never delivered to field units as such.  It was always
containerized first into tins.  There are few modern equivalents to this in
our modern world.  When we buy fuel at the petrol station, we pump it from a 
massive underground tank into our cars where it's sold by volume.  Rarely do
we buy a pre-packed can of fuel.  This was however how the British Army 
preferred to receive it's fuel --- in 4 Gal (18L) of petrol per tin.\footnote
{\cite[\Petrol][3]{27course}.
For reference, the 2025 Toyota Corolla sedan has an approximately 50 l fuel 
tank whilst the 2025 Ford F150 Raptor pickup truck has a 136l tank.}%cite
These tins were nicknamed flimsies, and it was not an ironic term of affection. 
They were meant to be disposable so they were built cheap; however, the 
design teams were perhaps overzealous.  The flimsies had an unfortunate habit
of breaking or leaking such that it was quite common for them to arrive 
damaged leading to fairly severe losses in fuel as well as a notable fire risk.  
Indeed, the flimsies were of such low quality that the British Army began to 
simply use captured German (Jerry) petrol cans --- hence our modern term 
jerrycan (a German petrol can).  Moreover, by Overlord, jerrycans were 
plentiful and it appears that flimsies were mostly relegated to carrying 
water.  Even containerized fuel was arriving ashore already loaded in
jerrycans and images of POL dumps post D-Day depict stacks of jerrycans and
not flimsies.\autocite[8:30 -- 10:55]{buildup}

%e fix transition

Nevertheless, despite the questionable durability of flimsies, the British 
Army had some sound reasons for using containerized, as opposed to than bulk 
distribution.  Firstly, tanker lorries weren't nearly so common in 1940 as the 
are today.  Secondly,
containers are compartmentalized.  If a bullet pierces a tanker lorry, one may
loose thousands of litres of fuel before one notices; however, if a bullet 
travels through a containerized fuel transport (i.e. lorry full of flimsies),
one may loose only a few tins worth of fuel.  Moreover, containerized fuel has
far fewer mechanical requirements.  For bulk fuelling to work, one must have a 
working petrol pump.  This could be quite inconvenient.  Imagine having a 
tanker load of fuel but no simple way to get the fuel out of the tanker.  
Moreover, using this system, you can only fuel a few vehicles at a time.  
With containerized fuel, one merely pulls up to the vehicles, unload a few
tins at each vehicle, and each crew then subsequently fuels their vehicle
with a cheap tin funnel.  Of course, this system was quite laborious to 
use but even so, it was judged by the British Army that the additional labour
was worth the cost.  % packaged at factory?

%e define POL
%e para is awk, solve later.  It reads like it was inserted after original 
% writing --- newsflash: it was lol!
All told, the British POL supply chain was designed, to provide containerized
fuel for the Army.  As designed, it was intended for the Army to be able to
advance the whole army 75 mi (120 km) using only such reserves held by the 
field army (the GHQ/Corps areas, and the Divisional Areas).  50 mi (80 km) 
of fuel would be held by the the Divisions, whilst the Corps areas would hold
the remaining 25 mi for the divisions, plus an additional 75 mi for the corps'
organic transport.\autocite[\SupInWar][3]{27course}

Having been bulk broken at the Corps or GHQ levels, it was now up to the 2nd
line transport units like 90 Coy to then bring those stores forward into the
Divisional areas and deliver them to the end-user units.  Depending on 
operational requirements, this may mean delivering it directly to the 
individual end-users, or it could mean delivering such stores to the units who 
could then further distribute stores internally.  This formed the basic, 
theoretical structure of the British Army's supply chain; however, just as how
no plan survives first contact with the enemy, the supply chain had to adapt
to tactical and operational necessities.  

Already, you may have noticed that the 27th Armoured Brigade is a 
\textit{brigade}.  Why does it have it's own 2nd line transport?  The answer
is fairly simple, 27th Armd Bde's full name was 27th Armoured Brigade 
(Armoured Assault). %e fact check
The Bde was raised as an independent armoured brigade for Overlord.  As such,
it needed a way to ensure it could run its own logistics.  You may also recall
how 90 Coy was, on D-Day, delivering both POL as well as ammunition to 27 Armd
Bde.  This shows how the supply chain had to remain flexible.  Whilst in 
theory, there was a separate chain for POL and ammunition, in practice, this
was impossible.  This was the advantage of containerized fuel as fuel could 
simply be loaded into any available lorry.

\subsection{Storage and Dumping}
Finally, before we carry on with the affairs of 90 Coy, it may be prudent to 
clarify what is meant by a `dump' and other forms of storage.  In a perfect 
world, supply chains would be perfectly efficient.  Ever single item required
by an army would be produced when it's needed, sent to where that item was 
required without delay, and used immediately on receipt.  Alas, hiccups 
invariably appear.  Shipping gets stalled, major operations consume unusually
large quantities of supplies, supplies are lost to enemy action, etc.  Thus,
to ensure first-line units receive a continuous flow of supplies, it was ---
and remains --- necessary to store a reasonable reserve of stores at various
points along the supply chain.  

Ideally, this would be a large, dry, flat, 
climate controlled warehouse with good transport networks, but alas, 
conditions in the field often are not always ideally suited to the 
logistician.  Thus, supplies were often stored by stacking supplies in a 
field or some woodland and covering them with tarpaulins if they required
protection from the weather.  The precise requirements of this may seem quite
trivial and not terribly important to the profession of fighting wars; however, 
seemingly trivial tasks such as labelling and organizing are critical.  
Consider what would happen if there was a German counter attack and the supply
officer could not find the 76mm anti-tank shells because their boxes were
not properly labelled or because the dump was not given enough land so that 
the aisles were too narrow.  Moreover, what would happen to those same shells 
if they were dropped and the packaging was inadequate to protect their contents
--- and honestly, who hasn't dropped a heavy box before.  Damage to the shell
casing could prevent the casing from ejecting properly after firing leading to
a stoppage and possibly leading to the tank being out of action.  

Consider also what would happen if one of these these dumps was attacked and
caught fire.  Aisles do not merely provide access but function as fire breaks.
These fire breaks are critical for hazardous material dumps such as POL dumps
or ammunition dumps.  When these dumps catch fire, it is often too dangerous
to attempt to extinguish the fire --- POL burns and High Explosives explode.
Instead, standard operating procedures tend to relate to containing the fire
and letting it burn out on its own.  %e insert medal citation example

This may seem small but how do acts like this win wars?  Unlike the combat 
arms, logistics does not win wars by plunging a bayonet into the hearts of the
enemy.   
Instead, logistics wins wars by ensuring the combat arms can act without 
restrictions.  If there is insufficient ammunition or fuel to support an 
advance, a General cannot order that advance.  If reserves are not ready when
the enemy attacks, then the combat arms will have few options but to withdraw
or fix bayonets.  
Logistics enables and constrains but achieves nothing on its own but by doing
so, is a significant factor in determining if an operation is achievable or
foolhardy.  Let us return to Normandy in June of 1944 to see this in play.
%e how's this transition?  I kinda don't like it.

	% Talk about how supply works, the DIDs, supply lines, POL differences

	% complaint about the flimsies and containerized fuel

	% talk about how you sustain client units, march length, the end goal
	% of ending the day with filled tanks, full ammo, etc.

	% Keep that story telling feeling, you're at the control point mourning
	% the fact that there are no other second line units, you're vastly 
	% overworked.  If only you had the additional units to support you.
	% if only ____ instead, you're left trying to maintains [sustainment
	% standards] and just keep what's needed flowing as hard as you can.

	% Use to establish what we're actually dealing with so importance
	% becomes self-evident
	
	% How supply chains work from dumps, and depots, to distribution, to
	% 1st line units.  


	% Key was placed on  on rabild concentration of force, incl sup, LofC
	% etc. to permit strategic use of force %cite 27 trg crs p. 32

		% 4 main areas, the BSA, L of C, Corps/GHQ area, Div area
		% goes from 

		%cite No 27 Trg Course p. 34ish (diagram's on 34)

		%\subsubsection{L of C Area}

			% Envisioned as a railway line, lorries will do 
			% just fine but this theory came at the dawn of
			% mechanization.  Point was to take stores from
			% the BSA and transport them near the combat zone.
			% It's basically a transport area.  IT goes from 
			% just outside the BSA to, and including the railhead.

			% If distance from the BSA to the railhead is < 12
			% hr trip, BSA will control.  Else, there will be
			% a regulating station around 6 hrs from the rail
			% head. %cite no27 Winter Trg p. 32 5c
			
			% If using trucks, it's 150 tons per 21 trucks with
			% old pat, 59 tonnes across 12 trucks (5 supplies, 2 
			% ord, 2 R E, 2 stores, 1 RAMC)  railway trucks 
			% %cite No 27 course, p. 32 para 5


	

	%\subsection{Warehousing}

		% Especially in those early weeks when a lot of it was dumping
		% rather than warehousing per-se.  

		% DID (maybe??) 3000 Tons/acre= gross stacking area.  x 4-10
		% for expansion %cite No 27 trg 32 (this might be a general
		% rule too)

		%\subsubsection{Base Supply Depots}
			% Located near Base Marshalling Yard, outside docks
			% to permit expansion and dispersion.  Consider `water
			% light, telephones, good roads, office accommodation'
			% when selecting the location. Area = Strength X 
			% stock X weight one ration / (unintelligible) = 
			% 8.5 sq ft /ton = stacking 
			% +50\% for stock spacing = gross stacking area X 4
			% to 10 for expansion, etc.  (that's a quote)
			%cite 27 trg course p.32, 4
	% zoom out a little and talk about what you'll be controlling.
	

% /* Figure out where to incorporate the fact that the British/Canadians focused
% on firepower over manpower.  This means materiel is critical --- A is for ammo,
% B is for beans, C cold water, D: diesel, E-everything else... */

% /* Do I want to expand to include things like traffic control?  Traffic jams
% on Sword Beach may have made the Br fail to capture Caen on D.  10m of dry
% beach between water and sea wall at high tide.  Perhaps an MP or two would
% have solved the issue. IIRC, RAF beach sqn dealt with it. (See
% RAF beach sqn/det  Was this a critical
% oversight?  Not a lack of tenacity or anything else, but a good, old fashioned
% traffic jam VI's-a-vis Toronto at rush-hour caused the failure to take Caen?
% lol --- what a way to win a war! */

\section{Return to the moment} %e rename this
%New title, want to get back to what 90 Coy was up to till D+7
% sum up the Jun Hist Rep for 90 Coy 

By the morning of D+1, the situation for 90 Coy was slowly improving.  90 Coy 
was still quite overwhelmed, but C Pl's 22, and B Pl's 4 lorries that were used
with the rest of the platoon, ferrying stores from the Beach Sector Stores 
dump to the nascent Bde dump at Hermanville.  

C Pl's greater number of lorries takes longer to land with elements being 
ferried ashore throughout the day.  As they landed, they delivered their
original preloads to their intended recipients before moving to supply 
6 Airborne however, by the afternoon, fears were beginning to materialize of
a German counter attack targeted at the Eastern bridgehead presently held by
6 Para.  As such, all available transport in the 3rd British Infantry Area
were ordered to assist in preparing for this German counterattack on the 
British left flank.  

C pl simply continued running supplies to 6 Airborne
as usual as the stores they were building up would be extremely useful if
the Germans attacked.   B Pl was however was busy establishing a reserve of 
critical stores for 27th Armd Bde, running up and down the congested road 
running between the Beach Sector Stores Dump and the Bde dumps at Hermanville.
When the order came through for B Pl transport a Battalion of British infantry
% find out the unit
4--5 km East to St Aubin d'Arquenas to meet the feared German counter 
attack, the Pl was around half way through the process of unloading jerries 
at the dump.  The situation was so urgent however, that the infantry battalion
was ordered to mount up on top of the jerries and they were rushed East.  After
this, B pl switch between continuing to build up the Hermanville dump and 
delivering stores to the forward elements of 27 Armd Bde.  This is perhaps 
representative of the role of logistics in warfare.  Logistics contributes to
military success by removing constraints, but it often does so not by reacting
to a threat per se, but by ensuring that the Army is ready to receive the 
enemy by prepositioning assets where they may \textit{foreseeably} be 
required whether that be by transporting troops or by establishing dumps.  
This establishment of dumps may seem fairly hum-drum; however, consider this:
by D+2, B Pl had been engaged had no more than 1-2 hours of rest over the 
course of 60 hours.  By D+2, B Pl was falling asleep at the wheel!  

This is
how critical the British Army considered the dumps at Hermanville.  The 
object of these dumps was to have a contingency in case the Beach Sector 
Stores dumps were attacked --- and frankly, Hermanville has better road access
 than Queen Beach.  This work may seem unimportant compared to combat 
operations however, so much of logistics is preparing for the next step. 
Yes, the German counter attack does not materialize nor is the Beach Sector
Stores dump lost; however, imagine what would happen if either of these 
eventualities occurred and the work was not done.  What would happen if 
critical troops or supplies could not be accessed when they were needed?  
This goes beyond anxiety.  As this was happening, along Sword Beach, the 
Luftwaffe was attacking various Beach Sector Stores and, at 1345, they attacked
a POL dump adjacent to the main beach exit.  The attack ignited the POL in the
dump and the fire spread to near by supply and ammunition dumps.  Over the next 
3 hours, 60000 gallons of POL and 400 tones of ammunition were consumed in 
the flames.\autocite[8 June 1944]{1raf}  Efforts to extinguish the flames 
% firefighting details --- if I can find it agian!
This was indeed, not the only fire, over the next few days, POL fires dot 
various Army war diaries and RAF Operations Record Books.  We can assume that
these fires, from their frequency, rapidly become non-events as these events
are increasingly reported as `P.O.L. Dump hit\ldots' followed by, `P.O.L. 
Dump fire extinguished'.\autocite[10 June 1944]{1raf}  Increasingly, a 
quantified estimate is not recorded in the war diaries or operations record
books.  Nevertheless, it is highly likely that preparatory actions such as
prepositioning firefighting apparatus, stacking POL with a mind to fire
breaks, and dispersing the storage locations for these dumps helped to minimize
losses.  Whilst this may appear mundane, preparations such as this are 
essential to keep an army mobile.  Consider that 90 Coy was, as these fires
were raging, running loads of petrol forward for the tanks.  Once again, it is
rare that logistics can win a war, but it can certainly loose it.  Without 
these standard preparations taking place, it is probable that the British 
Army of 1944 would have simply been unable to fight in Normandy as it would
have been much easier for the Germans to simply destroy the buildups the 
British were making.  Whilst these stacking and loading standards are quite 
mundane, they are important to actually winning wars.

Consider also unforeseen events.  The paras 6 Airborne fighting East of the 
Orne would, due to the general difficulties in providing sustainment from 
the air, often found itself short of rations or ammunition.  Why, you might as
was it difficult to ensure the paras were well supplied with rations, is it 
not a fairly simple affair?  You know the strength of a division, you know how
many days of rations to provide them and some simple multiplication reveals
the number of meals.  Take the number of meals, divide by the number of 
rations in a case, divide that by the number of cases that will fit in a 
lorry, all all that's left to do is to find the rations, load up the lorries 
and go.  Job done!  Nice and easy!

Alas, if only life was so simple!  See, dumps had to supply these rations
and this math is only accurate if the supply officers knew how many men they
had to feed.  Typically, this is solved by storing an excess of rations at 
these dumps to make up for any shortfall; however, in the first days of the
invasion, rations were in short supply so these reserves that would have 
been prudent to build up simply had not had time to amass ashore.  Thus, on
in the evening of D + 1 when Commander RASC (CRASC) 6 Airborne Division --- 
the officer in charge of supplies for 6 Airborne
--- found out that they had been reinforced and that these reinforcements 
were to be fed by him, he would have had his staff check their supplies.  
His team would have informed him that they simply did not have the rations 
available.\autocite[6]{90wdjun}
  What would have then likely happened was that he would calculate 
the rations required, put a message through to Beach Sector Stores and request 
those rations.  This would set into motion several chains of events from 
clerks and officers nervously eyeing ledgers, making sure that this requisition
could actually be met off hand.  If it could not, they would be figuring out 
where they could squeeze from the supply system for a little extra.  Maybe
transfer stores from a different dump, maybe reduce the size of a shipment for
the next morning in hopes that they could fill their evening request, etc.  

Whilst all this was happening, transport officers would be liaising with 
transport units like 90 Company and pushing through orders to arrange for 
the transport (in this case, three vehicles) to then get those rations from
BSS to the end user.  CRASC 6 Airborne whilst all this was happening would be
ensuring he actually had room to put the rations once they were delivered, 
figuring out how to ensure his new troops knew where and when draw stores,
etc.  

This is complicated further when the required stores just don't exist in the
quantities available ashore.  By the afternoon of D + 2, 6 Airborne was 
growing of 75 mm Pack Howitzer shells.  Thus far, the supply of this 75 mm 
ammunition had been air dropped; however, it was insufficient to keep the
division supplied and it was mainly the paras that used this exact ammunition.  
As such reserves of these shells simply did not exist ashore.  Thus, CRASC 6
Airborne made some inquires with the Navy and an officer of 90 Coy was sent to 
the Navy's Command Post to liaise with them as they attempted to locate the 
stores.\autocite[7]{90wdjun}

Locating stores in 1944 was difficult.  Its not like today where one can 
search a database for the stores required, find which ship the shells are on, 
and just ask that ship to expedite that delivery.  It required hours going 
through reams of paperwork trying to locate a single line in a ledger but, 
until 
someone worked out which ship these shells had been loaded onto, the paras
would not be able to use their artillery. 

By the morning of D + 3, these shells were still nowhere to be found and 6
Airborne was beginning to grow desperate.  We will discuss the importance of
artillery later, but sufficed to say, the British were reliant on their
guns.  They were so desperate indeed that, that morning, 6 lorries of 90 Coy
were held so that instant the shells made it 
ashore, they could be sped to 6 Airborne's gun lines.  To permit this, CRASC
6 Airborne made special arrangements with Beach Control to allow the DUKWs
--- amphibious lorries --- to make an inland delivery (typically the DUKWs
are just used as ferries to Beach Sector Stores).  %cite factcheck
Thus, when the ammunition was finally located on the afternoon of D + 3 by
6 Airborne RASC HQ's Ammunition Officer, Navy contacted the reliant ship,
the ship unloaded her stores into the DUKWs, and the
DUKWs drove directly to 90 Coy's Colleville harbour, the ammunition was 
cross loaded onto 90 Coy's 3 tonners, and that ammunition was rushed to 6
Airborne's gun lines which were, at the time, stood to and actively engaged 
with repelling a German attack.\autocite[7-8]{90wdjun}
The German attack was successfully repulsed
by element's of 27 Armd Bde --- also supported by 90 Coy.  It was not until 
the next day, D + 4, that 6 Airborne's own RASC transport made contact with
their parent unit.  Until that time, the 46 lorries of 90 Coy (reduced to 20 
by D + 4) had been supporting two divisions and one Brigade, a force which 
would have been undermanned to support even a single Brigade.  

Think about what it thus meant that 6 lorries (around 1/4 of 90 Coy's remaining
strength at the time) was held, standing by to ferry that 75 mm ammunition 
instead of delivering other critically needed stores --- granted, by this time,
some of
the 3rd British Infantry Division's own transport had landed as well.  What
would have happened if those shells were not located?  6 Airborne would have
lost much of its artillery support.  Moreover, think about how complex it was
to locate and deliver even a single load of artillery.  Teams involved included
at least 6 Airborne's CRASC (at least one officer and a few NCOs), the Navy
Command Post (at least one officer, a clerk, and a signaller), at least one
officer and six drivers from 90 Company, likely around six DUWK drivers, the
teams at sea loading and unloading cargo, the Beach Control point, dozens of 
MPs controlling traffic, and doubtless more I have yet to think of.  Whilst 
the combat arms rightly gets much of the credit for fighting wars, and the
Generals credited for figuring out where to put men, spare a thought to the
staff work done by the men keeping ledgers, speaking on the radio, 
co-coordinating actions and pushing forward supplies.  When times are desperate, 
one not only needs brave men, but highly organized logisticians to ensure
that which was needed was obtained. Why else would you have drivers driving
almost non-stop for some 60 hours if their work could be ignored?
%e this is kinda messy. is there a better way?  Also, tie together the
% rations thing too.

%e do I want to mention how but D+3 0815, plans to open BMA in original 
% planned locations was decided?  \autocite[9 June 1944]{1raf}

\section{Operations to Hold Ranville} %name, if any?  Was hasty operation.  

Based on our impression of the first few days of the invasion, you would be 
forgiven for thinking that supply in general was quite a ramshackle affair.
Thus far, the picture is probably exhausted lorry drivers ferrying materiel and
troops this way and that, creating hasty dumps of essential stores, with busy
supply officers running this way and that trying to scrape together what 
resources resources were available to support operations; however, as the
situation stabilized in Normandy, supply slowly starts to become more regular
and these quick and hasty names I keep bringing up like Hermanville, the
6 Airborne's Dumps, etc. start to become more important.  It is thus
worth pausing to assess the situation and to put some order to the chaos and
really consolidate the supply chain that both we and 90 Coy were working to 
navigate. 

\subsection{The Supply Chain to Ranville}

With the exception of the Paras who were being partly supplied by air, the
supply chain supported by 90 Coy --- at least, as far as the Coy was concerned
--- originates at sea on the various transport ships loaded down with any 
number of stores.  These could be landing craft, landing ships, or any other
vessel capable of carrying a large volume and tonnage of cargo.  If these
ships such as the LST could be beached directly ashore, then they were 
typically beached and their stores discharged via their bow ramps.  These 
supplies were then taken to the Beach Sector Stores where they would be 
stacked in an organized manner taking into account the need for creating
aisles for both access, and fire protection.  

If the ships however could not beach themselves, then the stores could be
brought ashore either by rhino ferry, or DUKW (pronounced `duck').  As 
mentioned before, the DUKW was an amphibious lorry with a 5000 lbs 
payload --- 2.25 tons --- or a tad smaller than the 3 tonners used by 90 Coy.
Whilst DUKWs could be driven quite far inland, after lessons at %e where?
DUKWs were mainly used to transport stores from ship to the supply
dumps nearest the beach --- any old lorry can drive miles inland but driving
into the sea with a common 3 tonner is unwise.  Of course, in emergency 
situations as we have already seen with the shipment of 75 mm pack howitzer
shells, occasional exceptions would be made; however, it was generally best
to use the DUKWs to fulfill the mission that only a DUKW could achieve.

DUKWs were useful for moving things that would fit in a lorry; however, for
transporting vehicles or if there was simply a shortage of DUKWs, then rhino
ferries were used.  The rhinos were essentially shallow draft barges assembled
from pontoon structures that could have a ramp fitted.  They were typically 
moved with rhino tugs going back and forth between from ship to shore and back
again.  Rhinos had the advantage over DUKWs that they could take several 
vehicles on board at a time and, once beached, the vehicles could just be
driven off and any stores in those vehicles, offloaded at the sector stores
dumps as they drove past.  

In any case, however the stores were brought from ship to shore, their first
port of call in these first days of the invasion would have been the Beach 
Sector Stores.  This would rapidly evolve into the fully fledged Base/Beach
Maintenance Area (BMA) Moon controlled by 101 Beach Sub Area.\autocite[See
traces in Neptune No. 1 RAF Beach Squadron Operation Order found in][]
{1raf}  BMA Moon started along Sword Beach's Peter, Queen, and Roger
sectors and extended around 2km inland.  The full BMA with it's organized
supply dumps do not appear to have been fully developed by D + 2; however,
those dumps 90 Coy created as a Brigade ammunition dump in the vicinity of
Hermanville was likely on the land that became BMA Moon's ammunition 
dump.\autocite[Trace of BMA Moon
annexed to Neptune RAF Beach Squadron Operation Order found in]
[Legend entry 67]{1raf}   From these first dumps, logistics units like 90 Coy
would then transfer the necessary stores to dumps further inland essentially
forming a chain of operational reserves.  For example, take 6 Airborne's 
Ranville maintenance area mainly drew stores from Hermanville and units 
working in 6 Airborne's Area of Operations (AO) would then draw stores from
the Ranville dump forming smaller, often less formal dumps along the way.

\subsection{Ranville}

% how do I transition to mostly Ranville focus?

% Maybe just consolidations?  Oh, how about the Dumps at Ranville

% give context for the situation IVO Ranville at the time.  British holding
% several KM^2 E of the Orne at IVO Ranville.  This land was supplied from main
% beaches over the Benouville-Ranville bridges.  Single point of failure.  
% fear of DE cntr attack as DE tps moving IVO E of region probing for points of
% failure.  27th Armd Bde involved in supporting local infantry.  90 Coy
% supporting Bde

% Operation with Paras of 13/18th with paras, 90 Coy dumping stores in support

How these dumps grow and evolve becomes of interest to the to the historian
of logistics; thus, let us return to Ranville.  Recall that the Germans were
probing the area to see if they can dislodge the British and 6 Airborne of 
their lodgement North-East of Caen and East of the River Orne
and the Caen Canal. The paras had been holding onto a number of disunited
pockets surrounding their objectives and drop zones.  At the time, the 
territory held by the paras was still quite disunited and there was no
continuous British front line per-se but pockets of British troops securing
local perimeters.  This is not a problem per se, rifle fire can have quite a
long range so there was no strict need to maintain a continuous line.  
Nevertheless, it did mean that the more weakly held areas in the British 
zone were subject to German attack or infiltration.  

On D + 4 (10 June), exactly this happened in the fields roughly between 
Ranville, and a town 2 -- 3 km to the North East called Breville.  The
Germans had managed to break into a DZ from Breville but their attempt
to cross the DZ was repulsed.  Having been repulsed, the Germans contented
themselves with holding a a wood near Le Mariquet using around a company
of troops.  The significance of 
this position is that it would separate `the 5th and 3rd Para Bdes, which 
had not actually made contact at this stage'.\autocite[June, Appendix 2, p 1]
{7parawd}  In light of this, 7 Para battalion, at the time holding the 
South-West corner of the drop zone (DZ) was ordered to `sweep the woods and
to clear the enemy out of them' and to do all of this in the pouring 
rain.\autocite[June 1944, Appendix 2, p 1]{7parawd}  The paras, having no 
organic armoured units, was to be supported by B Sqn, 13/18th Hussars, 27th 
Armd Bde as well as the 13/18th's Recce (reconnaissance) Troop 
(Tp).\autocites[June 1944, Appendix 2, p 1]{7parawd}[10 June 1944]{1318wd}

The plan of attack was simple.  The wood was divided into four separate
woods named W, X, Y, and Z and, at 1600 hrs, the infantry and armour would 
work together to sweep the woods.  
At this stage, we would normally expect to discuss 
infantry-armour co-operation --- it was awful with the paras not even
realizing how many tanks would be supporting them.\autocite[June 1944, 
Appendix 2, p 2] {7parawd}
We could then discuss how, despite the loss of 4 Sheramns and 2 Stuarts to 
German anti-tank guns, the attack was successful in clearing the wood and 
capturing `over 100 P[risoners of] W[ar]' and greatly improving the moral
of the Paras.\autocite[10 June 1944]{1318wd}
What is far more interesting 
however, is what came next.  

The next day, 11 June, 13/18th Hussars joined the rest of 27 Armd Bde on the
ridge north of Periers-sur le Dan; however, B Sqn, the same Sqn that supported
the Paras' attack the day earlier, remained with the paras.  Next day, 12 June,
the balance of 13/18 Hussars join B Sqn and are attached to 6 Airborne but would
be supported by 27 Armd Bde.  This meant that 90 Coy was now responsible for
not only 27 Armd Bde located on the ridge between Hermanville and Periers-sur
le Dan, but also for maintaining the 13/18th Hussars operating in the vicinity
of Ranville.    Over the next few days, the 13/18th Hussars would support a 
variety of British units in the vicinity of Breville who's effect was to 
neutralize the threat of a German attack on the Eastern flank of the Allied
beachhead.

Meanwhile, for 90 Company, 11 June was fairly quiet.  Their activities for the
day simply consisted of a mere 5 lorry loads of general supplies for 27 Armd
Bde.  As such, the under strength Coy took the time to do some maintenance 
having been worked to the bone since D-Day keeping 27th Armd Bde and 6 Airborne
supplied.\autocite[11 June 1944]{90wd}  Given the light day, it is likely that
the tired men of 90 Coy also took a moment for them selves and got some more 
sleep.  The next day would also come with some pleasantries for, for the first
time since boarding the landing ships from 1-3 June, they company at last 
received letters from home.  %e talk about casualties
There must have been a simple human joy in hearing from one's friends and 
family.  Captains Grey and Foreman must also have been quite pleased for, 
in this correspondence, they were nominated for recognition (i.e. nominated
for a medal) for their actions supporting 27 Armd Bde, and supporting 6 
Airborne division respectively only a few days earlier.  Finally, L/Cpl Jones
--- no known relation to the L/Cpl Jones of Dad's Army fame --- was nominated
for an award after rendering first aid during an air-raid on the night of 9 
and 10 June.\autocite[12 June 1944]{90wd}

It is worth noting that letters did not always bring joy.  The mail was how 
soldiers on active service received news of injuries, illnesses, and deaths 
from home.  Moreover, some letters might be from girl friends ending relations, 
or spouses recounting the difficulties of live at home in war time. 
Unfortunately, the sources I had access to included remarkably little 
personal correspondence; indeed, none between family member's and I am thus 
not in a position to make significant comment on this aspect of the war.
Nevertheless, when, in our modern world, we can usually communicate with our
friends and family pulling a smartphone out of our bags and sending a quick
text, the situation was not like that in 1944.  When the men were sealed
in transit camps pending embarkation in advance of the invasion in late May
and early June, they were largely cut off from those outside their unit.  
It is likely why this seemingly insignificant event was included in a war
diary entry was a page and a half long, as opposed to the more common several
entries per page.  

Despite the pleasantries however, there was still work to be done. In light of 
the 13/18th's attachment to 6 Airborne, 90 Coy begins a dumping program in 
Ranville to establish ammunition and POL dumps to supply the 13/18th, and, in 
light of the single point of supply chain failure along the Benouville-Ranville
road, to establish a reserve in case the 13/18th were cut off.\autocite
[12 June 1944] {90wd}  The actual process of lorries moving to dumps and 
collecting stores started at 1800 hrs; however, it is worth also thinking about
the volume of work done by officers ahead of time.  Doubtless, a number of 
staff officers at the Company or Brigade would have calculated the required 
quantities of ammunition, POL, and rations likely prudent to keep on hand at 
Ranville, making forecasts of ammunition and fuel draw, etc.  They would use
mathical guidance --- fuel consumption is fairly predictable --- but doubtless
also a level of judgment.  After all, on 12 June 1944, six days after the
start of the invasion, no-one could be certain how much ammunition would
actually be consumed in this theatre of war.  

%e push log vs pull log.  Was this push or pull?  I imagine pull?  Consider
% the 75 pack howitzer shells
Having made such a judgment, these officers would have filed indents with
the BMA.  %e validate
The BMA would then have to see if they could supply the the stores requested
on the indents.  Just as occurred earlier with the 6th Airborne supplies, if 
they could prudently supply the materiel, all's well.  Simply prepare the
stores to be picked up, and arrange a convenient time to draw the stores.  If
they could not however, there would doubtless have been efforts made across the
supply chain to acquire these stores and, only if this was impracticable, would 
it be likely that the request was denied.

Whilst all this was happening at the BMA, logisticians at 27 Armd Bde or 90
Coy must have been calculating the required number of lorries, the available
number for making the shipment, etc.  Evidently, the request was approved and
90 Coy decided it could spare 12 lorries for this dumping operation.\autocite
[12 June 1944]{90wd}  The operation continued to the next day, 13 June, when
 the Company's commitment increased to 20 lorries % as a proportion?
to the dumping operations completing the dumping program some time that day.  
The Company managed to get some rest on the 14th where, beyond some small 
deliveries, the Company had a maintenance day to look after themselves and, 
more importantly, their lorries.\autocite[14 June 1944]{90wd} 

% ask audience to consider the scale of work.
% what was the combat arms getting up to at this point?

One now might ask, why all this activity in the area around Ranville?  
Operations around this time to capture the wood are today remembered as the
Battle of Breville; however, this gives the impression of a set-piece battle
which this battle was not.  Instead, this was a brief period of fighting
surrounding this town which turned out to have strategic importance.  What 
started as a firefight to be handled by a unit fighting in their area of 
operations, evolved into a strategically significant battle involving units
drawn in from other divisions.  From a fighting standpoint, this has some 
minor interoperability concerns as well as some chain-of-command issues; 
however, simple co-operation such as the action that placed the 13/18
Hussars under the command of 6th Airborne.  

Sustainment however is a larger issue.  Place yourself as a supply officer in
this situation.  When someone moves an infantry battalion into your area,
one moves mouths to feed and rifles to fire.  Moving a unit of infantry into
an area already dominated by infantry does not cause a fundamental shift in
requirements.  All that has to happen is that the supply chain must expand to
be able to meet the requirements --- itself a challenge but less problematic
than what happens if you move units with new sustainment requirements. The 
issue is that a Second World War British infantry division tends not to have
organic armour --- certainly not the Paras nor 51 Highland division also
operating in the vicinity of Ranville.  This means that the supply chain 
must now be prepared not only for increased volumes, but also different 
proportions of stores.  An infantry division has lower POL requirements than
an armoured division, as is the ammunition requirement --- infantry have no
use for 75 mm tank rounds.  Moreover, tank units are tied to the supply chain
in ways the infantry is not.  Infantry can forage and men can be put on half
rations for short duration without significant consequences; however, a 
moving tank will always consume roughly the same amount of fuel if driven the 
same way, on the same terrain.  

Thus, if one wishes to use tanks --- tanks being quite important in warfare
even during the second world war --- one must have sound logistics.  This is
where the flexibility of logistics units come in play.  At this point, 90 Coy
still has a mere %number
vehicles; however, forethought, contingency planning, and adaptability was
doubtlessly helpful.  Detaching 13/18 Hussars from 27 Armd Bde was simple for
the combat arms but 90 Coy needed to think deeper.  It had to think about how
to schedule supply runs some five kilometres away from the main body of the 
Brigade.  Moreover, it had to consider contingencies.  What would happen if
the bridges at Benouville or Ranville were taken out of service and the 
13/18th's sector was attacked?  Bridging units were available and standing by
for such contingencies but building a bridge under active air attack is not
an enviable task.  In light of this, the decision was taken to expand the
dumps at Ranville so that it could support a few infantry divisions as well as
an armoured regiment. Doubtless, it was helpful that 90 Coy and the Paras 
likely already had a close working relationship seeing as how, just a few
days earlier, it was 90 Coy that supplied them; however, it is likely that
some significant effort was needed in order to establish and maintain the
new dump.  In a sense, whereas moving an combat arms unit is akin to moving
a body of men, moving the supply chain involves setting up new infrastructure
and it is this infrastructure that is critical for the effective conduct of 
modern war.
%e is this repetitive?  Can I play with this idea of infrastructure as war?
%e sort out transition

	% 12 June, mail arrives, first letters since D Day

% 13/18 moved E of river to defend against counter attack %cite 13/18 WD 12 Jun
% A, B Sqns supporting attack of German positions by 7 Para & black watch.  
% on 10 Jun, Germans were seating themselves into a wood IVO Le Marquette, 
% threatened to separate 5 & 3 para Bde as well as supporting infantry units.  
% Paras ordered to sweep woods to dislodge Germans with 13/18th in support.
% part of wider fears of a German counter attack.  
%cite 7 para jun WD appendix p. 5
% 90 Coy established a dump using 12x 3tonners.  Mainly Amn & POL, started
% 12 1800 Jun 44.  Increases to 20x lorries next day (how many lorries in 
% a coy again?)
% dump at Ranville (1173) %cite 90 coy WD 12 Jun 
% Do I wanna mention decorations here for 90 Coy or maybe do it earlier
% IVO Ranville.  Positioning of this dump wise.  Only route to supply troops
% E of Orne and canal is via Benouville/Ranville bridges.  This would permit
% bridgehead to keep fighting if the bridges were destroyed.  Infantry armour
% co-operation was a noted problem.  6 tks KO.  Paras take 40 prisoners, 0 KIA,
% 9 Wounded. 
%cite 7 para jun WD appendix p. 6


% Highlight that preparation is key.  If there was a counter attack and the
% bridge was blown, gallantry matters naught if you aren't supplied.  What 
% will you do, fix bayonets --- well, yes actually...?
% C Pl 90 Coy continue to provide light support till 16 Jun

\section{The Arrivals of A \& B Pls (14 -- 23 June)}
% non critical section, setting the tone of a lull.  A/B Pl arrive 
% 15 2000 Jun 44.  Go to Coy HQ in Cresserons GR 0379.  Brings with them 
% 59 Veh, 165 Pers.  Arrives 2300 hrs. Some DE bombing.  Over the next 3 days,
% new platoons involved in dewatering and reorganizing in prep for ops.  
% sporadic bombing and shelling of veh pk.  Veh dispersed 75 yds between 
% each veh --- quite exposed with no protection then (they're in an open 
% field).  

The Battle of Breville and establishment of the Breville dump having been
completed, both B and C platoons of 90 Coy spend the 14th of June maintaining 
their vehicles and, doubtless, getting some rest at Coy HQ located in a field,
some 500 m NW of Cresserons.  Here and there, the Coy
do some minor transport details --- delivering rations, ammunition, fuel, the
usual minutiae of war --- but the situation is quiet.  The next few days are 
fairly quiet for the Brigade.  Most of its forces are in defensive positions 
across the 3rd British Infantry Division front north of Caen or in the area 
East of the Orne.  Here and there, the Brigade takes small action defeating 
German strong points or repelling small attacks but nothing that, from a 
logistical standpoint, couldn't be managed through the usual supply runs.

Back in England, A, and the part of B platoon 90 Coy RASC that did not land on 
D-Day were, at this time mounting their lorries and embarking on LSTs for a 
their channel crossing.  The 59 vehicles and 165 personnel of this platoon 
group arrives and begins disembarking at Queen Beach around 2000 hrs on the 
15th.  Three 
hours later, they make the 6-7 km journey to Cresserons to join the rest of
the company.  Their arrival doubtless involved the greetings of friends, as
well as some good natured ribbing experienced by new troops joining old 
troops.  There must have been questions asking about the present situation,
the location of latrines, mess arrangements, and the usual questions one asks 
living in the field; however, the sporadic bombing likely helped to emphasize
the fact that there was indeed \textit{a war on}.%cite  
In light of this, the 
Company dig slit trenches to provide some cover against bombardment.  

Withe the new intake of vehicles and men, the Coy spend the next few days
reorganizing and dewaterproofing their new vehicles and doubtlessly, handling
routine supply runs, all whilst being sporadically shelled.  Beyond slit 
trenches there was little to be done beyond spread out the vehicles with 75 yds
between them to minimize the damage of a single bomb or shell.  One must wonder
what a dreadful inconvenience this must have been to have to go possibly
hundreds of metres just to get to one's lorry.  In addition, one wonders the
nature of the earth works in these areas as, with such dispersed vehicles, it
must have been dreadfully open to have been caught in the open during a 
shelling.  This harassing fire must have been irritating as the Germans did 
not do very much heavy shelling.  Instead, using 17 June as an example, the
Germans would lob a few shells (six in this case) over the course of a day and 
hope they hit something.  One wonders if slit trenches were dug at every vehicle
or if you hear the whistle of an incoming shell, if you just lie down and 
pray.  It is likely that sleeping positions were in slit trenches but even a 
trench was not always enough to protect the men.  Every few days, a few men 
would be evacuated with wounds from shelling.

As an aside, I should 
note that when I say `dewaterproof', it's not so much making it so that the 
vehicles would leak, but that they removed a series of minor modifications 
made to their lorries to ensure they would not be damaged during the crossing 
of the English Channel as well as when they waded ashore.  Much of this work
was done by the Woman's Army Corps.\footnote{Indeed, the role of women in WW2 
logistics is an opportunity for developing our understanding of the role of
women win the Second World War.}   Whilst protected from the ravages of the 
ocean, these modifications had to be removed before the vehicles drove too 
many miles as they prevented the proper cooling of their engines.  

% Queen Beach still being shelled.  (appears both in 90 Coy and 13/18WD)
% buildup thus slowed.  13/18th don't get their D+9-10 residues by 18 June
% This is a failure in supply.  Delays due to weather.

% Generally fairly quiet for Bde till end of month %cite 27 Armd Bde WD
% quiet for 90 Coy until 23 Jun %cite WD

\subsection{Oh Mundanity!}

% Maybe talk about the perhapses?  Amn still was consumed, as was POL, in 
% addition, what about general transport of rations, etc.  I don't have
% a src, but these must have happened.

% Talk about that wonderfully dull stuff like stove and flour shortages in
% the 27th Bde WD Admin instructions (and defecate in the latrines!).  
% Also, traffic flow patterns, spare parts, uniforms, water points etc.
% Highlight the importance of the ordinary and mundane. 

% Talk about how a POL lorry catches fire in the 13/18 rgmt area on 20 Jun.  
% Spreads to Amn dump.  %cite 13/18WD
% Not in 90 Coy's WD.  Why not, routine fetch and carry.  Fire happens, 
% amn is replaced, maybe only a few lorry loads, no bother.  Talk about
% the need to look between the sources.  An armoured rgmt is unlikely to
% have the integral transport to replenish that dump and frankly, it's not
% the job of 1st line units to maintain a 2nd line dump.  If you just read
% the WD, you would assume they're twiddling their thumbs most days.  This
% is unimaginable!

As you can likely begin to infer, late June was not a busy time for 27
Armd Bde.  Beyond sporadic fighting, there is little of note to the tactical
situation and thus, supply runs were still taking place. However, during this
brief stabilization in 27 Armd Bde's AO, one begins to see a return to the
normality of military life as captured by the Bde's administrative 
orders.\autocite[See end of June diaries]{27wd}
Indeed, the Brigade's first Admin O was not issued until the start of this
period on 14 June likely because the Bde was simply far too busy.  
Nevertheless, these orders provide a wonderful opportunity to examine daily
life for 90 Coy and indeed, the whole of 27 Armd Bde.  

It is perhaps revealing that it had to be said that `Latrine trenches must
not be allowed to fill up.  Fresh trenches must be dug and the old sites
clearly marked'.\autocite[June Adm Order No. 3][Para 10]{27wd}  Apparently,
this was quite a problem as, two days later, the Bde was advised that, 
`Attention will be paid not only to properly constituted latrine erections but
also to the general sanitary condition of the area, particularly checking 
failure to use facilities provided' --- clearly, there was an issue getting 
the men to use the latrines provided.\autocite[June Adm Order No. 4, Para 1a]
{27wd}  
More over, it appears it is indeed true that old habits die hard for, on 
14 June, the whole Bde had to be reminded to drive on the right side of the
road, and to turn on the correct side.\autocite[June Adm Order No. 1][Para 9]
{27wd}  It was also with some amusement on reading that `Any livestock 
\textit{accidentally} killed by shell or [Small Arms] fire may be cut up 
and eaten by units if bled fresh and in good condition' (emphasis added);
however, one is left wondering just how accidental some of these killings
were as, by this time, the men may not have had fresh food for over a 
week.\autocite[June Adm Order No. 1][Para 10]{27wd}

Beyond these more humorous examples however, these Admin Os reveal a 
situation of scarcity.  For a start, in the Brigade area, there were only
three water points by which units could draw water:  Benouville, Colleville
Sur Orne, and Hermanville.  Thus, along 27 Armd Bde lines, some units or 
detachments may have been over 2 km away from the nearest water 
point.\autocite[June Adm Order No. 1][Para 4]{27wd}  Thus,
every day, either 90 Coy or the units would have had to drive dozens of water
cans to the nearest water point, fill them, then drive all the way back
consuming both time and fuel.  Ration parties and ordinance stores would also
have daily delivery runs which allows us to start to see the baseline problem
of sustainment.  

Ration requirements are easy to forecast, simply count the number of mouths to 
feed, multiply by the number of meals between supply runs, divide by the number
of meals in a case, and round up to the nearest whole case.  General stores 
such as ordinance stores however were more complicated.  `All demands [were to]
be made to [the Brigade Ordinance Officer] at Bde A Ech[lon] by 1600 hrs daily.
Available stores will be delivered next day'.\autocite[June Admin Order No. 1]
[Para 6a]{27wd}  This thus creates an elastic demand on 90 Coy where any day
could have more or fewer stores thus complicating calculations.  

The first week of Overlord also saw some real shortages.  Almost all vehicles
and, not withstanding Lee Enfiled rifles, weapons were in short supply.  In 
addition mine detectors, `binoculars \ldots compasses \ldots watches', and 
surveying equipment used by the Artillery were all in short supply.  Even
communications equipment was short.\autocite[Appendix A to 27 Armd Bde Adm 
Order No. 1 (June)]{27wd} What's worse was drivers had a habit of running
into communications cables consuming ever more supplies.\autocite[June Adm 
Order No. 2][Para 3]{27wd}  The situation was so serious that special care 
had to be taken that, if at all possible, if an officer --- someone who
would likely have binoculars, compasses, and issued watches --- or OR had
some of these controlled stores, actions needed to be taken to relieve that
individual of the goods.  

On top of this, armoured units had special stoves, the No. 2 (tank) cooker,
for the tanks so that tankers
could heat their meals or boil water in the field. Non-armoured units could 
usually rely on being well enough connected to the supply chain that they 
were to stay connected to the Cooks' lorries; however, before the invasion, a
number of non-armoured units were issued these No. 2 cookers.  By 21 June
however, any `vehicle not entitled to carry them' that had access to a mess,
were to return the stove to the Brigade Ordnance Officer so that the stove
could be reallocated.\autocite[June Adm Order No. 3][Para 3a]{27wd}

All these must seem quite minor.  Why should a serious historian concern 
themselves with something as trivial as the availability of binoculars, 
compasses, stoves, rations or water?  The answer is simple:  get these wrong, 
and you loose the war.  The trivial appearance of these stores is by design.  
The mission of the Services is to %insert FSR quote on how it eases the 
% commanders burdens
When well run, a commander, and thus, the author of most of our sources need
not think about logistics but it does not mean it is unimportant. Without
these stores, officers cannot see far or navigate and the men will starve and
dehydrate.  Put yourself in the hobnailed ammo boots of a supply officer and
you received that order on redistributing cookers.  

All of these requirements would have to be foreseen and 
prepared well in advance to ensure the required stores were available when
needed.  This is when the army was simply in stasis; however, by the end of
June, the operational tempo for 90 Coy was beginning to once more accelerate.

\section{Operation Mitten 27--28 June 1944}

% Want to show that operations impossible without such units.  Focus on what
% worked, not what didn't work for the British.  British tanks bad, arty good.
% this is a useful moment to talk about the nature of Br Arty.
% Objective (no idea TBH, check 27th Bde WD) TLDR form google search,
% eliminate a DE salient at a chateaux not on my map IVO GR 0372

% 90 Coy's Work

% 30x 3Ton loads of 105mm amn held on wheels for 3 Br div on 23 Jun.  
% Likely for M7 Priest.  Should clarify, this isn't 90 tons but 30 
% truckloads. wonder if I can find out how many 105mm shells a 3 tonner can 
% load onto it.  Does it volume out or weight out?  Emphasize that 90 Coy
% was only helping this unit.  Technically, as the 27th Bde unit, they 
% don't do arty stuff.

% Amn is delivered 26 Jun to batteries located IVO GR 0378, it's D-1.

% don't forget about the stuff for flamethrowers!!

Operation Mitten occurred mostly within a single 24 hour period from 27 -- 28
June 1944.  It aimed to destroy a German salient around 10 km North of Caen. 
This salient was anchored by two Chateaux, Chateau de la Londe, and Chateau de
la Landel.  The assault on the salient was principally attacked by the 8th 
Brigade of the 3rd British Infantry Division.  27 Armd Bde would provide
tank support and  141 Royal Armoured Corps came equipped with Churchill 
Crocodiles --- Churchill tanks who's bow machine gun was replaced with a 
flame-thrower.

As ever, 90 Coy's tasking was principally to support the armoured units; however,
their first job of the operation was to deliver some 30 lorry loads of 105 mm
ammunition to the gun lines several kilometres away from the front lines South
of a commune named Plumetot.  Here, several Gun Batteries of the Royal Artillery
were emplaced in preparation for the upcoming battle. 

	\subsection{British Artillery} % amn consumption in barrage, arty
		% usage, etc.  Show just how dependant the Br were with
		% arty.  Maybe get the mass of a 105 shell and propellant?
		% there's no way to get an accurate estimate though, if
		% only they were 25 pdrs...

Artillery is often thought of as a supporting arm; yet, the British Army of the
Second World War tended to operate on the principle that it is better to expend 
firepower rather than manpower.  In light of this, when faced with difficulty,
the British Army was liable to attempt to crush that obstacle under the weight
of artillery.  This could be from fire directed by a Forward Observation 
Officer (FOO) against a specific, observable target (fire for effect), or it 
could be a preplanned suppressive bombardment such as a creeping barrage where 
the guns are laid to bombard a moving line in advance of advancing troops.

Artillery was quite an effective and flexible tool.  Take the example of 
Lt Boyle of 17 Field Regiment RA who acted as a FOO for 38 Irish Bde in 
Sicily.  He and an infantry company commander once saw a large number of German 
troops massing, likely in preparation for a counter attack.  Thus, Lt Boyle 
got on the wireless, adjusted fire onto the German unit, and order `10 
rounds gunfire' from an artillery regiment of 24 guns.  Soon, `240 shells
landed within an area less than a football ground'.\autocite[133]{gunfire}
The company commander was impressed and asked for another salvo.  The FOO
simply said the proword 'REPEAT' and another 240 shells once more
saturated the target area.\autocite[133]{gunfire}  It is this ability to 
rapidly concentrate firepower on any point within range of the batteries
by simply making a call on the radio that lies behind the power of the
artillery.

Doctrinally, it could be used to kill an opponent,
neutralize them (force them to keep their heads down for long enough for 
friendly infantry to kill or capture that opponent), demoralize them, or 
`partially destroy' them (kill or wound 2\% of entrenched forces or 20\% of 
troops in the open).\autocite[133]{gunfire}  Unlike the First World War, by 
the Second World War, it was relatively uncommon for the British to fire
multi-day preparatory bombardments to entirely destroy enemy positions.  
These bombardments were too wasteful of ammunition, destroyed the ground, and
were not terribly effective at destroying an entrenched enemy.  
Second World War bombardments tended to focus on providing the enemy with a
`short, sharp shock'.  Of course, if troops were in the open, FOOs were quite 
happy to kill them but the usual aim was to suppress them.  

Despite these ammunition saving measures however, to do this required a vast 
expenditure of ammunition.  It was assumed that for a unit with
25-pounders\footnote{Common British field gun} to partially destroy enemy unit
equipment in a 100x100 yd square, the unit would need to expend 40 rounds of
ammunition.  Demoralization would require 40 rds/hr over 4 hours (160 rds 
total) or 100 rds/minute for 15 minutes (1500 rds).  Neutralizing the enemy 
would require 8 -- 32 rds/minute for as long as the enemy was to remain 
neutralized.\autocite[133]{gunfire}  Whilst batteries would have some
ammunition on hand, %e find qty
they were unable to maintain stock of the volumes of ammunition required; thus,
the RASC would dump and replenish ammunition for the guns.  
The stocks required to maintain this instant access to firepower could be 
enormous.  In the lead up to Operation Goodwood, which we will discuss later,
each gun was issued with 750 rds of ammunition just to ensure that the RA would
be able to meet demand.  Without logisticians, the British would likely have 
had to use more costly, manpower based attacks which come with higher 
casualties.  

On humanitarian grounds, high casualties are not desirable; however, on 
strategic grounds, this would have been disastrous.  Unlike the US or
even the Canadians, by 1944, the British were running out of men.  The 
British were unable to replace casualties from drafting more troops from 
home.  The manpower no longer existed.  The British could not be wasteful 
with men for each death or wounded man would mean the army in North-West 
Europe would shrink.  %colossal cracks stuff
Just to make up the numbers for Overlord required the British to draw down
staff from training establishments across the country.  It was acknowledged
that reinforcements were unlikely to become available. %cite colossal cracks
Already the British Army was constraining its operations and being less 
daring to conceive manpower.  This was with ready access to ammunition and
firepower.  Without regular supplies of ammunition to the guns providing this
instant access to overwhelming firepower, it is difficult to imagine how the 
British Army would be able to maintain sustainable casualty rates to continue
operations in Europe.  Once again, whist supply does not --- or at least,
should not --- fight per se, effective fighting is impossible without them.



	\subsection{Support to Operations}
		% Operation used flamethrower tanks from 141 RAC (Churchill
		% Crocodiles)  Needed special fuel and nitrogen cylinders.
		
		% Coy sets up dump of 3000 Gal (13640L) of fuel and 90 N cyl
		% at Gazelle (GR 0276) on 27th, withdrawn on 29th.

		% D+1, Maint Point set up at Le Vey (GR976757) distributing
		% 2400 Gal of FTF.  

		% D+1 for Staffs Yeo, 0300, 1082 rds 75mm HE delivered, 
		% 600 rds at 1600 hrs, 1200 rds at 2000 hrs delivered --- what
		% where the Staffs doing?  Find out  Also for context, give
		% amn capacity of tank  

		% in Op Inst No 2 for 27th Bde (28 Jun 44) sqn shoots limited
		% to 50 rds/gun.  Tanks were to harass Germans as long as 
		% Br infantry weren't harmed by this fire and the tanks move
		% as soon as counter battery fire is encountered
		%cite 27th Armd Bde WD, Jun 44 p12
		% This tells us there's no great fear of running out of 
		% ammo.  If you could fire off half your ammo in sqn shoots
		% and still be operationally ready for further operations or
		% movement, you're confident you could be replenished.
		
Whilst Operation Mitten started with artillery prepairations, 90 Coy was the
RASC unit for 27 Armd Bde.  In light of this, the majority of their 
work was in support of the armoured component of the operation.  The 3rd
British Infantry Division was, as usual, supported by 27 Armd Bde, but for
Mitten, the Div was also supported by Churchill Chrocodile flamethrower tanks
from 141 Royal Armoured Corps.  These tanks were unusual because, in addition
to the usual fuel, spare parts, and ammunition, the Chrocodiles also required
fuel for the flamethrower as well as compressed nitrogen cylinders for 
propelling that fuel towards the enemy.  In light of this, on 27 June, the 
first day of the operation, 90 Coy establishes
a dump 1--2 km from the combat area at Gazelle consisting of 3000 Gal (13640 L)
of flamethrower fuel, and 90 nitrogen cylenders to keep the Chrocodiles in
service.\autocite[27 June 1944]{90wd}

Alas, the assault that evening by %unit name
supported principally by the Staffordshire Yeomanry (Staffs Yeo) %hole
was unsuccessful.  Thus, over night plans and prepairations were made to try 
again with more forces the next morning.  Thus, 90 Coy spends much of the 
night replenishing the Bde.  Indeed, by 0300 hrs on the 28th, 90 Coy had 
delivered 1082 rds of 75 mm High Explosive (HE) ammunition to the squadrons
of the Staffs Yeo providing a picure as to the ammunition expediture in the 
evening prior.  The next day, 27 Armd Bde was ordered to support the renewed
assault by patroling the area and providing harrassing fire as needed.  
Squadron shoots were limited to 50 rds/gun.\autocite[Operation Instruction 
No 2 (see June appendix)]{27wd}  These 50 rounds represented 
approximately half of the capacity of each tank.  
%edouble check
The fact that ammunitition could be so freely expended is testimate to the 
effectiveness of logistics support for there was no fear that they would
run out of ammunition.  Indeed, if there were any such fears, the were 
needless.  Operations resumed around 0500 on the 28th, By 1600 hrs, 90 Coy 
had delivered an additional 600 rds of ammunition to the Squadrons. An hour 
after the successful end of Oepration Mitten around 1700 hrs, 90 Coy begins
to fully replenish the Bde and by 2000 hrs, they delivered an additional 
1200 rds of 75 mm ammunition, and likely fuel and rations as well, to Staff 
Yeo.\autocite[28 June 1944]{90wd}

In addition, as 141 RAC was still being supported by them, 90 Coy sets up a 
temporary maintenace point in Le Vey, approximately 5 km West of the Gazelle 
maintenance point, to replenish 141 RAC.  Over the course of three hours, they
deliver 2400 gal (10910 L) of flamethrower fuel to 141 RAC and, the next
day, the stores at the Gazelle dump are withdrawn and returned to 
depot.\autocite[28 -- 29 June 1944]{90wd} %hole 101 recieted it, did they?

All this work was done for a simple two day operation and it is indeed right
and proper that we remember the eormous loss of life suffered by the combat
--- indeed, over the course of June, 3 Div suffered disporportionate 
casulaties and we owe it to the fallen not to forget them --- but from an
operational perspective, to simply focus on the dashing infantry is 
imbalanced.\autocite[112]{assault-div} Over the course of 
around 36 hours, Staffs Yeo had consumed 2882 rounds of ammunition to kill 
six tanks.\autocites[June appendix, Operation Mitten Intelligence Diary, 
Entry 58]{27wd}[on ammunition expediture, 27-8 June 1944]{90wd}  Of course, 
a significant amount of that ammunition would have been HE ammunition fired
against infantry; however, this expediture is still quite large.  Moreover,
without fuel or rations, the attack would have round to a halt quite quickly.
Without the support of 90 Coy, the armoured component of this assault would
have been impossible.  Infantry casualties for Operation Mitten were already
higher than casulaties for the rest of the British Army %e check
and they would only have been higher if not for the armoured support.  To stop
our analysis of war with just fighting ingores what makes wars possible in the
first place.
Moreover, note the significant amount of activity after the closure of 
oeprations.  Effective logistics work is not simply to support the current 
fighting but to ensure the Army is ready for the next action before that 
action has even been fully thought out.%fsr quote

		% conclude this section maybe this doesn't actually need a 
		% subsection.  This much work was needed for a simple 2 day
		% operation.  Work vital but it was doable.  Sum it back up.
		% Maybe the tanks weren't the best, but their inadequacies were
		% floated by a supply chain that kept up so that they could be
		% bad but still effective.

\section{Operation Aberlour}
% Never took place, do I wanna talk about it?  Issued 27 Jun 44.  A lot of 
% admin stuff for 90 coy like amn dumps, etc.  (Cite p26 27th WD for Jun)
% more on p30.  I almost like the idea of talking about an op that never
% takes place.  It eliminates the fog that arises from what happens when
% there's contact made with emy.  Here we can focus on how operations
% take place.  Hmm...

\section{} % Run up to Op SHERWOOD

	% extensive minefields to be laid (do sq footage) 
	% (See pg 13 of 27th Bde WD)

\section{The Lead up to Charnwood}

% Use this section to warm up the reader as to how logistics works
% in operations.

% Context for Charnwood

% D = 8 Jul

% Will continue using the Gazelle ammo dump until exhausted (smart, why move 
%it)  This is the only initially authorized dump. 

% 90 Coy sets up POL pt just S of Hermanville 2100-2400 8 Jul (likely units 
% moving through, north to south, fuelling as they pass)
%cite 90 WD 8 July

% AP set up NW of Cresserons at Coy HQ-ish area %cite 90 WD 8 July
% Rgmts draw ammo next morning, ERY withdrawn 1400 when Caen's captured

% 90 Coy is carrying 1000 Gal FTF, 35 N bottles on wheels for crocs at
% Cresserons

% 3 days compo rats issued, additional 3 days AFV packs carried in tks if
% compo not avail.

% Blankets to be distr as situation permits


% tracked traffic to use tracks rather than roads --- degradation
%cite 27th Jul orders 34-6  Sandbags used to store casualty's kit

\section{Pre Goodwood}
% 27th HQ moves to Douvres 10 Jul

% CONTEXT
% Post Charnwood, Bde is withdrawn on 24 hrs notice on 10th, increased to
% 48 hrs notice on 13th.  Men get some rest but in the background a number
% of conferences occur in preparation for Goodwood.  Vehicles are likely
% being maintained and losses replenished in this time.  2nd Army's being
% reorganized and a CO's conference on battle lessons occurs on the 13th.
%cite 27th WD


% talk about how this is often seen as an interlude but life continues.
% first bread ration issued on 11 Jul from 35 Fd Bakery at Luc sur mer.
% 2 oz issued / man increased to 4 oz next day --- men would have eaten 
% largely tinned food for a month.  Fresh food likely quite welcome.
%cite 90 coy 12 Jul

% Moreover, important work continues.

% Note the shortage of messtins and KFS at hospitals and the chronic shortage
% of tires.  Requirement to return amn casings, empty ammo cans, etc.
% 27 Bde Jul orders p37

% note how issuance of rations, units were overdrawing, and fresh rats were
% scarce.  Officers sent round to audit.  Talk about how this isn't just about
% REMF harassing front line troops, but a necessary part of military admin.
%cite 27 Bde Jul Os p38

% TRANSITION

\section{Goodwood (18-20 Jul 44)}
% ARGUMENT:  Use Goodwood to really start to show the centrality of logistics
% to ops.  No food, no amn, no POL, no fighting. (Goodwood detailed)
% Argue that trucks, not just tanks are what it means for an army to be mobile.

% STRUCTURE:  go day by day in the Run up and keep very chrono-narrative.  
% each day seems to sort nicely into themes; thus, use those themes and
% talk about them

% Point of operation  How do i want to manage this transition?  This is
% too abrupt

% THEME FOR 14TH:  Battle sustainment
% WngO (formal or informal) likely received for Goodwood 14th late morning or
% early afternoon.  90 Coy spends the afternoon organizing, then
% 14 Jul, over five hours (1700-2200), 90 Coy 
% converts the 13/18 rgmt amn & POL dumps IVO Ranville into dumps fit to 
% supply the Bde (likely approx 2 sq km in size).  
% dump consists of 38 lorry loads of amn to the 13/18th Rgmt dump enough
% to provide 97 rds/tk (a full load of amn) for the whole Bde.
% POL:  18 lorry loads of Pet & Derv (diesel) (7000 (either units or gal) of
% each).  This is enough POL to take the Bde 30 mi.  Coy does this via the BAD
% and PD.  Vehs fuelled as well before return to Coy HQ by 2400.  Speculate this
% was in receipt of WngO On 15th at 1900, Coy formally 
% establishes a det at the Ranville dumps to control R&I.  Det consists of 
% Capt Duffus, 46 ORs and 16 lorries.  Useful time to talk about fighting
% range and what it means.
%cite 90 Coy WD

% By the 15th, preparations for Goodwood occur in earnest across the Bde.  
% 3 Div holds another conference and, over the next few days, the Bde starts
% moving E of Orne likely via Ranville.  %cite 27th WD 15th Jul-ish
% Ranville det loads load 4 days compo into 6 lorries
% (enough to sustain whole Bde) as a mobile reserve.  8 lorries are devoted to
% keeping croc flamethrowers amned.  2 lorries leftover for odds and ends.
% Useful time to talk about importance of flexibility maybe?  
%cite 90 coy WD

% 16th:  RATIONS
% Bde HQ moves closer to Ranville but stays W of Orne.  Bde Cmdr holds 
% conference with Bde COs. %cite 27th WD
% As a result of this, all units W of Orne received 3 days extra compo, E of
% Orne, 4 days. %cite 90 WD
% Likely, whilst not stated in the WD, that 90 Coy made these regular 
% deliveries to the battalions.

% Total rations thus 4 days in vehicles, and 4 at 2nd line transport.  Ready 
% to move for 8 days.  Therefore, maximum planned speed for brigade's advance
% is approximately 30 mi (let's be realistic, likely 20 mi to account for 
% general movements) over 8 days by accounting for POL.  A good day of 
% advance would be 3.75 mi (6km)/day.  


% 17th:  Comms
% 27th HQ issues it's OpO for Goodwood today.  The Bde will help hold 8 Corps'
% left flank as 8 Corps breaks through.  Bde establishes it's battle net today.
%cite 27th WD and Orders
% 90 Coy receives two wireless lorries, one at Coy HQ, and 1 at the Ranville
% det.  Each lorry comes with 3 wireless ops.  %cite 90th WD
% Curiously, the OpO hardly mentions logistics.  Log is just expected to work.

% 18th:  D-day H-hr 0745
% Bde attack successful.  Objectives met around 1100.  At 2000 Coy HQ sends
% 3 lorry loads of PET and 3 of DERV (figure out range from 
% 18 loads = 30 mi) to Det Ranville. They're stranded, traffic is heavy,
% only E bound traffic permitted over the bridges.  %cite 27/90 WDs
% Start talking about what sustainment looks like.  If there are 6 lorries
% prepared, surely they'll be needed.

% 19th:  DE send 8 JU88 to attack Orne bridges.  Bde makes minor gains.
% ditto 20th.  Little change happens at this point as the front stabilizes.
%cite 27th WD
% 90 Coy uses this as a time to resupply units from Det Ranville.  
% talk about how ammunition moves.  The Ranville dump goes from X lorry 
% loads to 45 loads of amn when the dump closes and the amn is brought back
% to BAD Hermanville. %cite 90th WD for 25 Jul

% If Goodwood failed to break out, it wasn't logistical constraints.
\section{Post Goodwood}

% use it as a chance to talk about infantry transport --- Br infantry don't 
% have organic transport.  Also how more normal RASC transport units work

% 22nd, the Bde is informed they'll be broken up.  They get assigned to 
% 2nd Army's 22 Transport Coln at months end.  
% Det Ranville shrinks by 45 lorry loads of amn
% and the coy moves to Camilly (GR 933704).  Det Ranville is handed over to 
% different unit and rejoins the Coy.  Coy provides transport to Staff Yeo
% to Arromaches docks.  Coy returns the 2nd line amn holding of 30x 3ton 
% truckloads and 27 6-ton loads to BAD.  %cite 90 WD p.7 mention this holding
% earlier on too.

\section{} %not sure what to call it, something about being an ordinary
% 2nd line RASC unit


\section{Criticality of Supply}

% Ask the question of what would happen if not for these activities. 
	% Amn
	% Beans
	% Water (note location of WP)
	% POL
	% Everything else (white paint for turrets lest one gets shot lol)

% Invite community to see an army as a system and merely disconnected fighting
% groups.

% Discuss my gaps, I barely touch on maintenance units, etc.  What I could
% have talked about, what other sources say.  Consider Tiger or Panther without
% Maint units

% talk about sources, how 90 Coy WD doesn't really mention resupplies unless
% it's major, but just because it's not written, doesn't mean it wasn't done.
% by asking the question of who would do X, one starts to uncover the Y, and Z.
% talk about how entries like 6 Jun are very long --- likely because of the
% initial excitement.  The WD's progressively shorter and more concise as time
% wares on, likely because it's tedious and, "I want to go to bed!".  Supply
% logs are tedious.  Talk about how this study is actually quite constrained
% by sources in so far as a heavy dependence on electronic sources.  War diaries
% don't mention routine replenishments but it doesn't mean they don't happen.
% alas, I doubt they kept the waybills!


\section{Conclusion}

\newpage

\printbibliography

\end{document}
