\documentclass[noraggedright]{turabian-researchpaper}

\title{Logistics Under Fire}  %e maybe not?
\subtitle{The Logisticians of 90 Company RASC and the Logistics of Operation Overlord}

\date{\today} % to?% 
% author{}

% Sets UK Date format
\usepackage[british]{babel}

% Auto-punctuation outside of quotes
\usepackage{csquotes} 

% Auto link URLs, hyphenate long URLs in appearance.
\PassOptionsToPackage{hyphens}{url}\usepackage{hyperref}

% Find pkg to wrap long URLs

% Sets Bibliographic Style 
\usepackage[notes]{biblatex-chicago}
\addbibresource{src/sec.bib} 
\addbibresource{src/pri-uk.bib}
\addbibresource{src/pri-can.bib}
\addbibresource{src/iwm.bib}
\addbibresource{src/regs.bib}


%%%%%%%%%%%%%%%%%%%%%%%%%%%%%%%%%%%%%%%%%%%%%%%%%%%%%%%%%%%%%%%%%%%%%%%%%%%%%%%
% Citation Aids
% All citation aids will start with capital letters for sake of namespaces

% For citing 27 course's Lecture No. 12 Petrol p 35-7.  Largely on bulk vs 
% container
\newcommand{\Petrol}{Precis of Lecture No. 12:  Petrol}
%cite Author is Maj W. P. Pessell RASC.  Figure out how to do it when LAC 
% website works.  It's on p 35 of my scan
% For use with {27course}

% For citing 27 course's Lecture No. 5, has diagram on supplies.  PDF pg
% 41-2
\newcommand{\MaintProj}{Precis of Lecture No. 5:  ``Key Plans and Maintenance Projects''}
% For use with {27course}

% For Supplies in War Lecture pt 2 by Lt.Col. P. L. SPAFFORD, 0.B.E.
% For use with {27course}
\newcommand{\SupInWar}{Precis on Lecture ``Supplies in War'', (Part II)}

% For 27 Armd Bde Adm Order No. 7 pertaining to support fo Charnwood (was 
% issued with the Op O).
\newcommand{\CharnAdm}{27 Armd Bde Adm Order No. 7, 7 July 1944}

% For the 8th British Infnatry Bde's operation overlord int summary
\newcommand{\EightOverlordInt}{Diaries for May, Operation Overlord 
\textit{8 British Infantry Brigade Intelligence Summary}}
%%%%%%%%%%%%%%%%%%%%%%%%%%%%%%%%%%%%%%%%%%%%%%%%%%%%%%%%%%%%%%%%%%%%%%%%%%%%%%%

% Formatting notes, do I want to divide this into multiple source code files?
% Outline:  60-120 Lines

% Question:  Was the logistics of the British Army fit for purpose in supporting
% the British Army in operations in NW Europe from 1944-1945.  

% /*****************************************************************************/
% Outline as Follows:

% No indentation is a top level <h2> type heading / Latex \section{} unit.  
% Each successive tab indentation reduces the heading by one level.  Thus, a 
% single tab denotes <h3> / \subsection{} whereas two tabs denote 
% <h4> / \subsubsection{}.  Constrained to no more than three tabs/levels of 
% sectioning.

% Comments will use either C style comments for embedded remarks (/* comment */) 
% or Latex comments where the whole line after `%` will be ignored on output.

% Outline begins on line 25 and will continue till line 85-145.

% Output target length 50-70 pages. 

% Note:  POL stands for petrol, oil, & lubricants.  
% /***************************************************************************/

%%%%%%%%%%%%%%%%%%%%%%%%%%%%%%%%%%%%%%%%%%%%%%%%%%%%%%%%%%%%%%%%%%%%%%%%%%%%%%%
% MARKS SET

% i	Introduction
% h	Historiography
% c	Conclusion
% m	Mark Set (this)
% w	Working Point
% f	WTF are we doing here!?
% 

%%%%%%%%%%%%%%%%%%%%%%%%%%%%%%%%%%%%%%%%%%%%%%%%%%%%%%%%%%%%%%%%%%%%%%%%%%%%%%%

% Conops, perhaps use a narrative flow moving from before invasion for the
% prep work and training, to the actual invasion for execution and adaptation,
% to transition around or after TOTALIZE?

% TLDR:  if you accept that the British CA soldier was tactically worse than
% the Germans but were able to defeat the Germans by unloading an ungodly
% amount of steel and explosives over their head at the first sign of 
% difficulty, then the British logistical soldier was far superior.  A war 
% cannot be won by killing alone!

% Thesis: 

% L1:  Historical community is quite hard on the British for lacklustre
% results vis-a-vis armoured warfare and armour-infantry co-operation; 
% however, what this is missing is that the British Army --- Army, not 
% economy --- work quite well at a logistical level.  This is what, in the
% field, enabled the British Army to defeat the Germans despite tactical
% mediocrity.  Logistics worked to support flawed tactics.

% L2:  Criticality of the services to the effective waging of war.

% L3:  We ought to stop viewing armies as mere fighting forces.  Understanding
% them using a systems approach explains why we can win wars.  Merely 
% examining how an army can outfight its opponent at a tactical level misses
% the operational reasons the tactical level can even function.


% CONOPS:  What if I do this as a microhistory of 90 Coy?  I can use them as 
% a way to talk about the criticality of logistics.  What if I used them as
% the pivot around which an army (tbh, 27 Armd Bde, and 3 Div) can turn?  I 
% can use it to skirt around the wider issue that I don't have evidence for
% a macro view but I have 90 Coy's WD.  Alas, I can't find as many decorations
% for them as I would like to really make it personal.  Could I use examples 
% from other units as a `take Cpl Bloggins from X, note the work he did'?
% time to run git branch I suppose.  It also allows me to be narrative which
% is always fun.  Their 4 day history was already 7 pages, I could breath life
% into.  I can harp on about them for 8-10 x that, right? 1-2 pages 

% introducing (let's do it as if they hit a beach and they're driving a 
% convoy up the road to Benouville with that preload of supplies for 6 Airborne
% whence we pause to discuss what we currently talk about in the 
% historiography.   I have 5-10 pages on what other people have been writing 
% about Normandy, from there, perhaps another 5ish to narrow down on Sword 
% Beach, the whys, the objectives, who was involved etc. --- I should re-read
% return of martin guerre I think.  

	% so 20-30pgs here, I'll probably do more once I footnote it all.
% With this context, we return to the convoy and lay out the theoretical 
% framework by which logistics operates.  Depots, lengths of supply lines,
% how distribution happens, etc.  How this work integrates with the rest of
% the army.  We deliver the supplies and talk about the RV.  Let's also start
% talking about the arduous work over the next few days to supply 3 Div and 
% 27th Bde as there are no other 2nd line transport units.  As we continue 
% here, I can intersect the sources for 27th Armoured Bde to give context
% as to what 90 Coy was doing.  I suppose I"ll have to do some probables and
% perhapses.  We can talk about how 90 Coy supported the Bde until the Bde was
% disbanded.  Then, continue talking about how the Coy supported other units
% till Totalize maybe?  Then, have a final almost pre-conclusion discussing
		% Perhaps this takes 4-6 pages-ish?
% how, whilst we don't write about it, these operations would have been 
% impossible without the supporting arms and concluding with how the British
% ability to have tactical mediocrity and an unwillingness to spend lives
% meant that their logistics had to be good, really good.  

% Expand this to how armies function, and who this paper doesn't cover.  I 
% don't discuss workshops, clerks --- the whole of A branch actually --- MPs, 
% BADs, beach detachments, etc. yet they're all important. (5 pgs)  Maybe 
% paint a picture of the whole rear area and why it's important (landing 
% tickets maybe?).
% Tie it back to historiography and methods like  how a lot in the sources 
% requires you to intersect doctrine with the WDs to figure out what supply 
% units were doing --- making a meal dely isn't something that really gets 
% recorded in the WD, etc.  (5 pgs)

% Finally, conclude.

\begin{document}

\maketitle

\section{Introduction}

	%Estimate 2-3 pgs?

	% Set scene, we're in Normandy, 6 [time] Jun 44, we just landed, we're
	% driving from Queen Beach to Benouville to link up with 6 Abrn.  
	% you're X lorries are carrying preloads of ammunition, rations, and
	% other supplies for 6 Abrn.  They're holding the Anglo-American 
	% left flank from a possible German counter attack.  what if I put
	% it in the perspective of that Lt who did the initial recce?  You 
	% weren't originally scheduled to land yet but an accident of war
	% means here you are.  Describe more about what 6 Airborne is doing
	% you don't belong to 6 Airborne but are instead of 27 Bde, you're 
	% just tasked to help them on D day, their unit 2nd line unit will
	% arrive from Juno later /* Come back to this when discussing import */

%e change time
It's 1430 hrs on 6 June 1944, Captain Foreman, C Platoon (Pl), 90 
Company (Coy) 
Royal Army Service Corps (RASC) just stepped LST 382
onto a beach called Queen.  Before him was a war zone clogged with traffic,
and struck by intermittent shelling and sniper fire.  Only a few hours before,
the men of the 8th British Infantry Brigade and the \mbox{27th Armoured Brigade
(27 Armd Bde)} --- the latter, their home Brigade --- hit the beach, drove off 
the German defenders and were now proceeding inland.  Behind him was 11 of the 
36 3-ton Bedford lorries of his platoon.\autocite[31 May 1944 and 6 June 1944
entries both found in June diaries]{90wd}
In these lorries was ammunition and other critical %e check
supplies urgently needed by the Paras of the 6th Airborne Division who, the
night previous, jumped %e or stepped?  What about gliders?
into the flack-filled skies over France to secure the Anglo-American left 
flank.

As C Pl took their first steps onto France, %e into?
they would have looked for sign-posts, a Military Policeman, an RAF Airman,
or anything else to direct them towards a beach exit, and onto the congested 
roads that would take them to Colleville-sur-Orne, 3 km away.  The trip took
some 90 minutes but, when Capt Foreman finally arrived --- doubtlessly 
irritated at the slow state of the roads --- he hoped 
to find his Commanding Officer (CO), Major Cuthbertson, CO 90 Coy RASC who 
landed earlier that morning at 0925 hours to try to establish contact with 
elements of the 6th Airborne Division who were eagerly waiting a 
resupply.\autocite[6 June 1944]{90wd}

Not far away, Lt Glenny of B Pl was also landing with nine lorries, not 
with supplies for paras, but with fuel for his colleagues fighting in the 
tanks of 27 Armd Bde.\autocite[6 June 1944]{90wd}
Unlike the usual dramatis personae of works concerning Overlord however, once 
these supplies were delivered, they would not take up arms and join the 
fight but instead, find out what the units they supported required, and return 
to the beaches to bring more supplies forward.

This paper does not tell the typical story of D-Day and the subsequent 
capture of Caen.  Instead of focusing on the fighting done by the troops or
the decisions made by commanders, this paper focuses on the work that enabled
the material battle to take place.  
It is 
not a story of supply from the factory to the harbour nor from ship to shore. 
This is a story of supply only a few kilometres from the front lines.  
It is logistics under fire --- an aspect so critical to warfare but so rarely 
understood.
We will look into the logistics of those critical last few miles from
the shoreline to the combat zone. Without the work, forethought, and
attention paid by soldiers like the men of 90 Coy, the British Army would have 
ground to a halt.  This study thus examines the contribution of these
logisticians to the final victory in Normandy, showing, through concrete cases,
how the British Army overcame the immense challenge of sustaining an army in 
the field.  
%e concerned this thesis isn't quite strong enough

To do so, we will 
follow Major Cuthbertson and 90 Company RASC as they work their way across 
the English channel.  We will continue to follow the Company as they land
in Normandy and follow them as the units they 
support attempt to capture the city of Caen, and we will examine their role 
in the closure of the Falaise Pocket in August. %e will I really go to Falaise?
Along the way, we will first examine how the British Army structured logistics 
administratively, before joining 90 Coy as they supply the 27 Armoured 
Brigade as they partake in the Battle for Caen.  Along the way, we will look
look into the challenges of providing sufficient ammunition, food, and fuel
to the British Army.
After 27 Armoured Brigade is broken up at the end of July, we will see how 
90 Coy integrated into a larger and longer supply column as they support 
infantry units through Normandy. Following this, we will have a brief 
discussion on historical methods and how they apply to military logistics.  

%
% harbour near Colleville.  An hour earlier, he and the 11 lorries of C Platoon
% 90 Company RASC (90 Coy) disembarked the LSTs they had been stuck on for the 
% past six days waiting to cross the English Channel to support Operation 
% Overlord, the Anglo-American invasion of Normandy 
% France.\autocite[1--6 June 1944]{90wd}  Loaded in these 
% 11 lorries were supplies for 6 Airborne Division currently operating to secure 
% the British left flank over the Orne.  These loads consist of `pet[rol], 
% [ammunition], R[oyal] E[ngineer] stores, and water', stores vital for the 
% paras of 6 Airborne Division to resist a German counter 
% attack.\autocite[S \& T Report (June History Report) p 4]{90wd} 
% %cite check this later
% Alas, despite the urgency of these stores, Major Cuthbertson, 90 Company's 
% Officer Commanding has yet to make contact with 6 Airborne so C Platoon has 
% little to do but wait for contact to be 
% established.\autocite[6 June 1944]{90wd}
% Thus, doubtless, the men of C Platoon, 90 Coy would have dismounted their 
% lorries and pause.  Likely, they would have appreciated being once more
% on dry land having spent the last few days being bounced up and down in the
% English Channel.  A few kilometres away, the men of the 6th Airborne Division,
% the 3rd British Infantry Division, and 90 Coy's home brigade, 27th Armoured
% Brigade were, in the case of 6th Airborne, guarding the British flank, or
% in the case of 3 Div and 27 Armoured Bde, pushing inland to try to reach
% Caen.

% In many ways, this story of 90 Coy is like any number of D-Day stories.  The
% unit splashes down on a more or less secure beach, gets stuck in a traffic
% jam whilst proceeding inland, gets organized in some sort of assembly area 
% and gets to work either fighting or supporting the fight.  What is different
% about this paper however is that we are less concerned with the fighting and
% more concerned with laying out the conditions so that fighting can take place.
% In other words, what is different is that I will use the fighting as the
% backdrop to see how logistics enabled the British Army to fight, and 
% investigate the significance of the contribution made by logisticians to the
% ultimate, albiet lackluster, success of the British Army in Normandy.  I
% will also show how the reasons for why certain logistical decisions were
% made in the manner they occurred, and how, at a systems level, logisticians
% interacted with the combat arms units they supported.  
% 
% The men of 90 


%  C2y are thus, not 201 you might initially 
% think of when you i2k of soldier.  Whilst 201 infantry fought with 
% rifles, and armour, their 
% tanks; the men of 90 Coy, though certainly armed with rifles, fought mostly
% with the 3-Ton Bedford lorry.  Unlike the combat arms, a supply and transport
% company like 90 Coy's role was to \textit{deliver the goods} wherever and 
% whenever they were most neded.
% Thus, we will follow Major Cuthbertson and 90 Company RASC
% as they cross the English Channel and land in Normandy.  We will their journey
% through Normandy, not fighting per se, but ensuring those who fought, had what
% they needed to be able to fight as they attempted to capture Caen and close
% the Falaise pocket.  
% Along the way, we will examine how the British Army
% structured logistics at an administrative level, before joining 90 Coy as they
% support the 27th Armoured Brigade as they partake in the battle for Caen.  
% After 27 Armoured Brigade is broken up at the end of July, we will see how 
% 90 Coy integrated into a larger and longer supply column as they support units
% through Normandy.  Following this , we will have a brief discussion on the 
% historical method and how it applies to military logistics.  


% This paper is structured to de-emphisise the battle as, regardless of what 
% goes on, men still need fuel and water, their machines still take petrol, 
% and their weapons still need ammunition.  Instead, we will look
% into a general continuum of work as this is the world of the logistician.
% Rather than a discussion that revolves around the landing and the succeeding
% battles between D-Day and Falaise, %e or Caen
% this paper is concerned with daily ammunition runs, the pre-emptive 
% establishment of ammunition and petrol points, the desperate scrounging of
% ammunition and equipment to make up for shortfalls, and the refit of units 
% after the battle is over.  

%%%%%%%%%%%%%%%%%%%%%%%%%%%%%%%%%%%%%%%%%%%%%%%%%%%%%%%%%%%%%%%%%%%%%%%%%%%%%%%%
% STOP WORK POINT HERE new intro is above this demarcation
%%%%%%%%%%%%%%%%%%%%%%%%%%%%%%%%%%%%%%%%%%%%%%%%%%%%%%%%%%%%%%%%%%%%%%%%%%%%%%%%

	% THESIS
	% You know what you're doing is important, without these supplies and 
	% the war will ground to a halt.  

	% You wonder what the historians who write about these events 80 years 
	% in the future will think about it all, what drives your focus?

	% Maybe have a bitter reflection, it's always the fighting troops or
	% the generals who get the cheers.  No-one applauds the cooks!  Use
	% the perhapses of history for this

	% Shift to that historiographical frame:  With the benefit of 
	% hindsight, we know that the drive to Caen would not be a quick 
	% drive, but as a logistician, one asks oneself, was there a 
	% significant logistical constraint or was it due to something
	% else?

\section{Historiographical Review}

%ep Show how what I'm doing is idfferent from standard military history,
% explain how non-eliete logistical labourers contribute to 
% war.  

% Note:  important how there's no Tim Cook without a stacy even if Stacy
% is a little boring

	% `Of course we know that 80 years after the fact, historians...'

	% Complain loudly and long-windedly at the lack of discussion on log 
	% in /*insert list */  Consider subsections or flow of text?

	% Discuss the current logistical scholarship done
		% Namely historical work in 50s and Supplying War, Julian
		% Thompson, Great Feat...

	% Introduce Section

	% General observations:
		% Much work done on great men, tactics, the merits of German
		% Armour; and some work has been done on American logistics,
		% as well as logistical peculiarities like the mulberry 
		% harbours.  I fear little has been done on the actual
		% military administration of the war.  Indeed, the logistical
		% section of military history is fairly poorly written about
		% by historians, some more work done military academies
		% reflecting its importance to them

	% talk about tooth to tail ratio

%%%%%%%%%%%%%%%%%%%%%%%%%%%%%%%%%%%%%%%%%%%%%%%%%%%%%%%%%%%%%%%%%%%%%%%%%%%%%%%%
% New Section
%%%%%%%%%%%%%%%%%%%%%%%%%%%%%%%%%%%%%%%%%%%%%%%%%%%%%%%%%%%%%%%%%%%%%%%%%%%%%%%%

It is of couse impurdent to make any study of Operation Overlord without paying
due consideration to the current, vast liturature as Overlord can cardly be 
considered a poorly studied operation.  There are, of course texts like 
\mbox{Russel A. Hart's} \mbox{\textit{Clash of Arms}} where he argues that the 
British Army was doctrinally inflexible in facing the new developments requied
to meet the challenge of fighting the German Army and that the British Amry's
doctrine was fundamentally unsuited fighting the Germans.%cite p 8

%e insert more general critiques of british operaions

This is contrasted by \mbox{Stephen Ashley Hart} who, in %name
argues that the British Amry's doctrine of \mbox{\textit{Colssal Cracks}} was 
wholly appropriate given how, unlike the Americans who could afford to take 
casulaties, the British could not.  The British were, by 1944, rapidly 
running out of troops --- casualties simply could not be replaced and over
the course of the Normandy campaign, 21st Army Group would have to reorganise
several times to make up the shortfall.  Moreover, Hart argues that the British
had to account for the legacy of the First World War which menant that they
understood that the moral of the men would not take kindly to the feeling that
their lives were disposible or being spent needlessly.  Hart argues that this 
need to maintain both moral and troop numbers meant that mean that the 
set-piece battle, a form of battle that allowed the British to hammer the 
Germans with immense, preprepaired fire-power before advancing under a curtian 
of artillery, was wholly appropriate to the practical constraints facing the
British Army in 1944.  This reliance on firepower was only possible due to the
British Army's effective internal logistics --- an aspect Hart leaves out.
%e check this

The literature pertainent to the Second World WAr is not constrained to more
general or strategic level histories however.  Many do narrow down into more
specific aspects of the war.  \mbox{Stephen Napier's}
\mbox{\textit{The Armoured Campaign in Normandy}} examince `the performance
and deficiencies' of Allied Armour, the intensions of all senior commanders 
involved, the relationship between Eisenhower and Montgovery, and he does so
from the perspective of the Anglo-Canadian-American armoured units fighting
in Normandy.  Napier gets quite close to the troops with sectiosn discussing
aspects of daily live such as the provision of rations and ammunition.  Beyond
noting that these critical stores arrive in company \textit{harbours} however,
Napier is relatively silent on how they get there.

Stig H. Moberg's \textit{Gunfire!} takes a similarly detailed line examining in
great detail, the \textit{Queen of Battle} that is the Royal Artillery.  Moberg
outlines the creation of the British Artillery post First World War and 
expores the role of the 25-Pounder field gun --- a stable of Second World War
service.  He examines specificities such as the intricacies of gunlaying, the
adjustment of fire, barrage plans, and artillery tactics.  He notes the 
British's heavy reliance on artillery to suppress defending enemy troops and 
the high consumption of artillery, at times exceeding %e number
rounds per gun  With %e number
guns per battery, the British consumption of artillery ammunition was 
astronomical.  Moberg notes the importance of the RASC in stock piling the vast
reservces of ammunition requried by the gunlines in advance of any major 
operation.  Understandably, his study of artillery ammunition does not stray 
excessively far from the gunlines though, as we will examine later, the 
transport of artillery ammunition would prove quite a burnden on the British 
Army's supply and transport resources.

Along similar lines, studies have been made of the Mulburry Harbours --- two
sets of rather ingenious floating peirs and accompanying breakwaters.  %e name
was an example of this.  He however is mostly concerned with the development
and use of the harbours themselves and is less concerned with the supplies that
ran over its peirs and what happened to those stores once they made it ashore.

Along similar lines, \mbox{Christopher A. Yung's} \mbox{\textit{Gators of 
Normandy}} studies the naval aspect ofthe invasion.  Yung covers aspects such
as shipping space, landing craft, Army-Navy cooperation, and shore bombardment.
Whilst he devotes sum attention to naval logistics, he understandably does not 
spend much time on the Army's.  

Despite these modest complaints however, it would be unfair to say that 
logistics is wholly ignored.  Instead, logistics tends to be shifted to the 
background.  However, there are a number of exceptions though they do not 
always focus squarely on the Second World War.  One of the more seminal 
academic works is \mbox{Martin Van Creveld's} \mbox{\textit{Supplying War}}.  As
one of the first academic histories of logistics, Van Creveld moves through 
some 400 years of Western military history to argue that it is impossible to 
adequately understand wars without understanding the logistical constraints 
imposed on armies by reality.  He looks into the logistics of promenent 
commanders like Wallenstine, Napoleon, Moltke, Rommel, and indeed Eisenhower by
examining the constraints and opportunities logistics imposed on these armies
such as the constraints of ammunition, fuel, and fodder; as well as the speed
of marching and railways. 

Specifically regarding the North-West Europe campaign, Van-Craveld writes 
less on Normandy and more on the subsequent drive from the beachhead through
France, Belgium, and the Neatherlands.  In this area, he first does a general
analysis of the logistical situation in September 1944 before analysing the
logistical feasibilyt of Montgomery's plan to advance teh British Army into
Germany through a thin, knife-thrust of some 400 miles.  Of the general 
situation, Van-Creveld notes how Patton's 3rd Amry had a habit of outrunning
their supply lines.  His logisticians subsequently struggled to keep up with
his rapid advance.  Van Creveld notes how 3rd Army's logisticians had to take 
emergency measures such as foregoing the transport of clothing and equipment
and prioritizing fuel, ammunition, and rations.  This came to a head as units
took to scrouncing in order to make up for the shortfall in stores leading to
some level of chaos in rear areas. 

Van Creveld notes how this problem was not duplicated in British areas.  The 
British tended to operate loser to the Channel ports greatly simplifying their
supply lines and thus, reducing shortages.  This lead to the British generally
having tighter supply discipline than the Americans.  This general availabilty
of prots also meant that, for the British, the capture of Antwerp --- a major
Belgian port and one of the largest in Europe --- was not strictly nessessary.
Van Creveld notes how if Eisenhower was willing to support Montgomery's plans
for a major thrust into Germany, this attack could have been logistically
supported --- though only just.  

Where Van Creveld falls short however is that he writes as if the logistical 
system was designed for the North-West Europe campaign.  He notes how the 
Allies had a complex method for requisitioning supplies --- what the British
called `indenting' --- without considering how oridinary the act of submitting
paperowkr for recieving supplies was a tt eh user level.  Armies not only 
\textit{fight}, but \textit{live} on their supply lines.  Thus, for an end 
user, the need to submit paperwork to get stores would likely have looked 
reasonably similar to normality with much of the complexities of meeting
such requistions being absorbed by Quartermaster staff.  Even in garrison,
it is common for paperwork to aquire such vast array of stores ranging from
rations to tent pegs.  In a sense, the pen is mighter than the sword not 
merely for it's ability to express complex and profound ideas, but also 
because the pen completes the paperwork.  The pen can order swords but 
without the pen, those who hold the sword will starve.  In this paper,
I will therefore examine those more mundane day-to-day aspects of logistics.
Whereas Van Creveld is concerned with logistics stratigy, I will look into 
the metaphorical --- and occasionally literal --- mechinery of logistics.

\mbox{Julian Thompson's} \mbox{\textit{The Lifeblood of War}} expands on 
\mbox{Van Creveld's} work, using the same methodlology to study more conflicts 
such as the %e examples
and Vietname.  Neither of these books concentrate purely on the Second Word
War though both draw examples from it.  

This is not the case with %e name's 
\mbox{\textit{Supplying the Troops}} which is essentially a biography of 
American General Sommerville who was %e jobtitle.
\textit{Supplying the Troops} provides a decent overview on strategic level 
logistics and the military-industrial complex but it lacks the same
\textit{boots on the ground}, practical perpective of \mbox{Van Creveld}
or Thompson. 

\mbox{\textit{War of Supply}} by \mbox{David D. Dworak} solves this 
defficiency.  Dworak aruges that the allies learnt the finer parts of 
logistics in the Mediterrainian --- skills they would later need in the
North-West Europe campaign.\autocite[2]{wos}
Dworak goes into great detail on specificiites such as the labeling of
boxes, the coordination of landing craft, and the organization of beach
groups.  Alas, he takes a very American perspective.  Whenever the Americans
learn something from the British, on feels as if one reads it as if it is
advice taken from outside of the hypothetical reader's in-group.  In a sense,
this book is very much not about the Allies per se as much as it is about the
Americans.  Nevertheless, I draw much from Dworak's methods and content.

A highly American focus is common on treatments on Second World European
logistics such as %e US Army in WW2 Europ, expand this

The role of logistics to the campaign in Normandy was not confiened to the
Allies of course, the Germans were also highly dependant on their supply
chains.  Ulike the Anglo-American forces however, the Gemrans, as the defender, 
had the luxury of being able to pre-position resources and to intricately plan
their supply lines to effectively resist an invasion.  Unfortunately for the 
Germans, that also made them a target.  \mbox{Russel Hart} looks into this in 
\mbox{\textit{Feeding Mars:  The Role of Logistics in the German Defeat in
Normandy 1944}} Where he argues that the Allied areal bombing campaing to 
destroy Gmeran logistics in Normandy was absolutely critical to teh eventual
German defeat in theatre.  The Germans were simply never able to maintain 
adaquate stocks of ammunition and fuel make up for expediture in Normandy.
This, Hart argues, is a major factor in the effective collaps of German 
resistance during Operation Cobra --- it was simply impossible for the 
Germans to keep fighting when they only had %e number
of the fuel they consumed and %number
of the ammunition.  Conversely, it was this eventuality of having to fight
in a theatre without the supplies to fight a war that the logisticians of 
the Anglo-American armies had to avoid --- something the presence of the
English channel and the shortage of sealift certainly did not help.  

The challenge of obtaining and maintaining sufficient sealift was a major 
theme in \textit{Neptue} by \mbox{\textit{Craig Symonds}}.  He looks into 
Operation Neptune, the seaborne component of Overlord, arguing that we ougt to
not look at Operation Neptune as if success was assured.  Instead, to 
understand Neptune, it is nessessary to understand the enormous amount of 
work done to prepare for the operation.  This ranges from high-level planning
at COSSAC, to the shortage of LSTs, and some notes on the actual operation in
practice.  Unlike many books on Normandy, Symonds pays a reasonable amount of
attention to logistics, mostly highlevel logistics.  This is quite reasonable
for a book more concerned on the operaitonal and strategic levels of war.  
Nevertheless, Symonds tends to keep his attention closer to the sea with 
most of his forays ashore being kept relatively near the coastline. 
%cite xvi for thesis

Finally, of particular importance to this study of the Briitsh Army's logistics
is \mbox{Clem Maginis's} \mbox{\textit{A Great Feat of Improvisation}} where he
argues that the logisticians of the British Expeditionary Force (BEF) 
effectively saved the BEF in the 1940 campaign in France.\autocite[1]{feat}  
As part of this, Maginis spends a good deal of time discussing the interwar 
periods.  Here, he examines the role of the post Frist World War disarmament, 
the remarament in the 1930s, and the shift in the British Army from being a 
mostly horse-drawn army, to being a fully motorized army when they returned 
to French soil in 1939.  Maginis's work is quite excellent but, alas, his 
study effectively terminates in 1940 after the arrival of the shattered BEF 
in the UK and the movement of those troops back to their UK garrisons.  He 
makes breif mention of the challenge of rearming the shattered army but, 
quite frankly, he makes almost no mention of the British return to France 
four years, less two days, later.  It is from here that we will return to 
Q Sector, Sword Beach, %e number
km North of Caen.  

%%%%%%%%%%%%%%%%%%%%%%%%%%%%%%%%%%%%%%%%%%%%%%%%%%%%%%%%%%%%%%%%%%%%%%%%%%%%%%%%
% Old section below
%%%%%%%%%%%%%%%%%%%%%%%%%%%%%%%%%%%%%%%%%%%%%%%%%%%%%%%%%%%%%%%%%%%%%%%%%%%%%%%%
%
% %e massage a transition into this
% 
% %e much has been written:  Baaaad phrasing!
% The Battle of Normandy is of course, a well studied topic.  Much has been
% written on this battle from books on the Second World War at large to 
% publications that focus squarely on operations and tactics in Normandy.  What
% is often ignored however is the work that went into that which enable 
% operations to take place.  There has been some, but relatively limited work
% on British logistics and, where extant, that work tends to be quite marginal.
% 
% 
% %e I'm feeling narative and I don't wnat to sum up these books again.  Later
% % problem!
% 
% Of course, books on the Second World War are quite common and they often
% grapple with concepts such as \textit{Why the Allies Won} by Richard Overy
% who notes the importance of the various technical and material advantages the 
% Allies had over their German counterparts in explaining the successes of 
% Anglo-American forces at a fairly high level.%cite
% From there, historians have narrowed down on specific components of that
% success.  
% 
% There is generally a debate that rages between British and American scholars
% of the war in the merits of Montgomery as a C-in-C Allied Ground Forces as
% well as a C-in-C 21st Army Group.  Russell Hart in \textit{Clash of Arms} takes
% the view that the British appeared overly cautious and lacked initiative whilst 
% the Americans by contrast often seemed quite daring.%cite
% He argues that the British were generally slow and
% were uninovative.  Stephen Ashley Hart in \textit{Montgomery and ``Colossal
% Cracks''} takes a deeper look at this.  He argues that the British were 
% generally more slow and cautious than their American counterparts, but that 
% this slowness was justified as the British were unable to wear casualties from
% both a supply and moral perspective.  Put plainly, Stephen Hart argues that by 
% 1944 the British had run out of men --- indeed, just scraping together the 
% numbers to launch Overlord took a significant amount of effort\autocite[56-7]
% {cracks}
% --- and that, even if the British had the men to spare, the British soldier 
% and public would not accept what they 
% viewed as needless losses.\autocite[24-5]{cracks}  
% Montgomery, Stephen Hart argues, was keen to maintain the `fighting spirit'
% of the British soldier.  Thus, Stephen Hart argues that, whilst the British
% methods were slow and methodical, that the British could afford to do so.  
% %e add more books
% 
% The scholarship also goes into areas of greater specificity.  Books like 
% Christopher Yung's \textit{The Gators of Normandy} and Craid Symond's 
% \textit{Neptune: the Allied Invasion of Europe and the D-Day Landings} 
% look into the seaborne components of Overlord.  \textit{Neptune} is 
% principally concerned with the actual seaborne invasion and it's associated
% events such as training, ensuring the proper buildup of goods and equipment, 
% as well as the planning of the landings.\autocite[xvi-xvii]{neptune}  
% Meanwhile, \textit{Gators} very much more concerned with naval component of 
% Overlord --- a narrative so often ignored given the general focus on leg 
% infantry, airborne infantry, and armoured warfare. %e confirm
% Both these books touch on logistical challenges, most notably the shortage
% of LSTs and other landing craft, or the immense challenge of getting 
% sufficient sealift capacity to move the requisite men and materiel across
% the English channel, but their discussion of logistics more or less ends
% at the shoreline sparing little thought as to how logistics functions once
% those supplies reach the army in the field.
% 
% Along similar lines %e insert name of book
% is concerned with the Mulberry harbours.  These were two portable harbour 
% consisting of a pontoon bridge, anchored on one side ashore and the other
% to a pier head a few hundred metres off shore, and a breakwater.  These
% harbours were located at Gold and Omaha beaches and were intended to aid
% with the unloading ships.  It was not viewed as favourable to land Allied
% forces too close to a harbour as harbours tended to be better defended 
% than stretches of open beach; thus, if the the Anglo-American forces could
% not take a harbour, they would simply bring one along.  Alas %e bookname
% once more makes little mention of logistics once the supplies that passed
% through the harbour made it ashore.

%e include section on staff work (re monty's men)

	\subsection{On WW2} % there's got to be a better name than this
		\subsubsection{\textit{Britain's Other Army:  The Story of
			the ATS}}
			% This feels kinda awkward to put here
		\subsubsection{\textit{Why the Allies Won}}

	\subsection{On Normandy}
% reorder this, chrono flow or based on argument?

% really flesh out the common critiques and defences of the British Army, 
% that it was slow to develop, under manned, unimaginative, morale problems,
% materiel over lives

		\subsubsection{\textit{Clash of Arms}}
			% Argues that the British were slow and failed in
			% innovating.  
		\subsubsection{\textit{Overlord}}
		\subsubsection{\textit{Fields of Fire:  Canadians in 
			Normandy}}

			
		\subsubsection{\textit{Montgomery and `Colossal Cracks':  
			The 21st Army Group in Northwest Europe, 1944-45}}
			% manpower Constraints
			% Morale problems Be sure to emphasize both of these
			% as supplies are critical to this
		\subsubsection{\textit{The Normandy Campaign 1944}}
		\subsubsection{\textit{Gators of Neptune: Naval Amphibious
			Planning for the Normandy Invasions}}
		\subsubsection{\textit{Neptune:  the Allied Invasion of 
			Europe and the D-Day Landings}}
		\subsubsection{\textit{From the Normandy Beaches to the 
			Baltic Sea: The North West Europe Campaign
			1944-1945}}
		\subsubsection{\textit{Feeding Mars:  The Role of Logistics
			in the German Defeat in Normandy, 1944}}
			% Toss this in Normandy or Log?

	\subsection{On Logistics}
% reorder this

%% Do I wanna put the books for WW2 logistics here or in WW2?  I'm tempted
% to concentrate it here but I also like the idea of keeping something from
% the field of military science and not history separate --- military science
% cares much more on actually executing operations.  Split this into two 
% subsubs and make the rest subsubsubs?  One for history, one for less so?
% Could also compress, maybe compressing and not sectioning this section 
% will flow better.  In any case, I wonder if it's better to chk pg ct

		\subsubsection{\textit{Supplying War:  Logistics from
			Wallenstein to Patton}}
			% Foundational in logistics scholarship, but, given
			% it's broad scope, it lacks depth
		\subsubsection{\textit{The Lifeblood of War: Logistics in
			Armed Conflict}}
			% Limited and tries to cover a lot of periods
		\subsubsection{\textit{A Great Feat of Improvisation}}
			% British but only really to shortly after Dunkirk

			% Use it to talk about how supply developed during
			% interwar years, namely, motorization.

		\subsubsection{\textit{War of Supply:  World War II Allied
			Logistics in the Mediterranean}}
			% Chiefly American in the Med
		\subsubsection{\textit{Supplying the Troops:  General 
			Somervell and American Logistics in WWII}}
			% Very much a great man history.  Logistics in the
			% form of a biography
		\subsubsection{\textit{Military Logistics and Strategic 
			Performance}}
		\subsubsection{\textit{The Story of the Royal Army Service
			Corps}}
		\subsubsection{\textit{Logistics and Modern War}}
		\subsubsection{\textit{Logistics Diplomacy at Casablanca: 
			The Anglo-American Failure to Integrate Shipping and
			Military Strategy}}
		\subsubsection{\textit{Strategy and Logistics:  Allied
			Allocation of Assault Shipping in the Second World
			War}}
		\subsubsection{\textit{The Science of the Soldier's Food}}
		\subsubsection{\textit{D Day to VE Day with the RASC}}
	
		

		

	\subsection{Tools of the Trade} % maybe rename this

	% The historiographical Gap.  Introduce an inattentiveness to log
	% here, that the materiel advantage is contingent on getting this
	% right.
	
	\subsection{A Note on My Sources}
		
		% Heavy reliance on digital records

		% paper records from Canadian sources owing to funding but
		% argue for the soundness of the methodology anyways given
		% how Canadian stuff is of Br doctrine

% Start with Context %%%%%%%%%%%%%%%%%%%%%%%%%%%%%%%%%%%%%%%%%%%%%%%%%%%%%%%%%%

\section{Overlord as Planned} % rename this

	% We return to our convoy, you're thinking and a bit day-dreamy 
	% driving through the French countryside thinking about the 
	% magnitude of what you're actually doing.  DO I want to go this
	% tone??

	% Overlord & Neptune:  Mission, invade France, push the Germans
	% across the Rhine sooner or later

	% Division of beaches, start from the W and move E.  How beaches
	% are subdivided.  Start thinking about the beach exits maybe and
	% why they matter?  Maybe reflect on the massive traffic jams?

	% We're intersted in Sword beach.  Over the Orne, Paras of 6 Para Div
	% holding the Left Flank.  From Sword Beach, 3rd Br Div was assigned
	% area from Swrod beach to Caen.  In this region, we have 27 Armd Bde.  
	% they provided tank support for 3 Br Div.  90 Coy was 27 Bde's 
	% 2nd line transport coy.  It basically provided the Bde's logistical
	% support.  On D-Day, ___ pl 90 Coy was also tasked to run supplies to
	% preloaded supplies to 6 Para over the river.  6 Para's 2nd line
	% transport was landing at Juno.

	% Talk about the 'bloody army', Q branch, A branch, and G branch.

%e toss in a map here of the beaches

Op Overlord was made up of a number of smaller operations.  The seaborne
landings were a part of Op Neptune.  This was the operation that established a
50 km wide beachhead in Normandy.  Neptune divided this section of
Normandy coastline into five discontinuous beaches.  The Allied right was 
anchored by Utah beach on the Cotentin Peninsula and the Allied left was 
anchored by the River Orne and the Caen Canal at Sword beach.  Between
these flank beaches were Omaha, Gold, and Juno beach.  The Americans were 
responsible for Utah and Omaha, whilst Anglo-Canadian forces were responsible
for Gold, Juno, and Sword beaches.  Each beach was subdivided into a 2 -- 4
sub-beaches and assigned a letter from A to R.  90 Coy, the subject of this
study, primarily supported the troops of the 3rd British Infantry Division 
and 27 Armoured Bde who landed on Sword beach; specifically, Queen 
beach ---  a stretch of smooth firm beach 400 yds deep at the low water mark,
30 yds deep at high water, and nearly 3 km wide stretching from Lion-sur-Mer 
to La Brèche d'Hermanville.\autocite[\EightOverlordInt][Para 1a]{8bdewd}

Immediately inland of Queen ran `a strip scattered with seaside houses and
gardens, to a depth of 200 yards'.\autocite [\EightOverlordInt]
[Para 3a]{8bdewd}
Along this strip were two roads that ran parallel to the beach, and 4 -- 5
roads running inland.  Three of these roads converge around 1000m inland just
North of the town of Hermanville-sur-Mer.  The road runs south through the 
town.  On the south side of Hermanville, 2000 m inland, is a road that runs 
parallel to the coast. Turn right, and head West 3 km and you will find 
Cresserons a few
hundred meters south of this road.  Travel further 1000 m and you will
reach La Delivrande.  Returning to the Hermanville crossroads facing south, 
Turn left and going East 1500 m and you will find Colleville Sur Orne. Travel 
another 4 km South-East of Colleville and you will find Benouville, 
Le Canal de Caen, the River Orne, and Ranville around 2000 m East of 
Benouville.  Everything North of this road was under the control
of 101 Beach Group, a logistics unit of brigade strength that formed 
the interface between the sea and the land.  They co-ordinated the movement
of men and materiel. %e chk

If we return to the Hermanville crossroads and travel along the road running
to the south for 10 km along this road past Beuville, Bieville, and Lebisey, 
and we will find ourselves on the outskirts of Caen.  
It is along this latter road that the 3rd 
British Infantry Division, supported by 27 Armd Bde and 90 Coy would attempt
to advance to capture Caen on D-Day Op Neptune.  This thrust was executed by 
2nd Battalion King's Shropshire Light Infantry (2KSLI) and the Staffordshire
Yeomanry (the Staffs).%cite
They were meant to march assemble around Hermanville
at 1100 hrs D-Day and advance with the men of 2KSLI riding on the Staff's 
tanks but, the Staffs were delayed for over an hour in heavy traffic on the
beach and the roads leading inland.  Minefields prevented the tanks from
going off road and thus, CO 2KSLI took the decision to advance alone with the
Staffs catching up later.\autocite[1100 -- 1230 hrs 6 June 1944]{2KSLIwd}
Their advance is halted at Lebisey that afternoon when an attack launched at 
1615 hrs was halted by snipers and MG fire at 1800 hrs a mere 3000 m North of 
Caen's outskirts.  2KSLI and the Staffs Yeo then withdrew to a more 
defensible position at Bieville with the last elements safely withdrawn
six hours later at 2315 hrs.\autocites[6 June 1944]{staffswd}[1630 -- 2315]
{2KSLIwd} The attempt to push into Caen will occupy the bulk of this study.
%e consider omitting the above

This study will also concern itself with the work done by 6th Airborne
Division as part of Op Tonga, the pre-Neptune airborne landings executed
by Anglo-Canadian forces.  Their objective was to execute a series of
airborne landings East of the River Orne, Caen Canal, and Sword Beach to 
secure the British left flank.  They were also to capture the two bridges
crossing the Orne and the Caen Canal North of Caen along a road running 
between Benouville and Ranville.  %e check this and %cite
All this was to be done during the night before the forces of
Op Neptune landed.  For approximately six hours, the paras of 6th Airborne
would be cut off.  Once the British landed at Sword beach, they would push
inland, to Benouville, cross the bridges if they were still intact, and
reinforce and resupply 6th Airborne.  %cite
That is how %e name the officer?
the 11 lorries of C Pl 90 Coy finds itself waiting in 
Colleville, around 4km away from Benouville.  They were waiting for their CO 
establish contact with the Paras.  Once contact was established, supplies 
could pour over Rugger and Cricket bridges to resupply 6 Airborne by land.
C Pl would then keep the paras supplied via Queen Beach until 6th Airborne's 
RASC unit could take over on D + 1 after landing at Juno.\autocite[1]{90wdjun}

By 1800, C pl made contact with the Paras and, as the Paras had successfully
captured the Orne and Caen Canal bridges, C pl was able to replenish the 
depleting ammunition of 6th Airborne by 2300 hrs on D - Day --- a five hour
job.  As 6th Airborne's area of operations had yet to be fully secured, the
drivers of C pl faced sniper fire throughout the day.\autocite[6 June 1944]
{90wd}

Not all of 90 Coy landed on D - Day however, whilst A and D Pls stayed in the
UK to be brought across the channel on 15 and 30 June respectively, 
B Pl landed on D - Day.  
Their tasking to simply support 27 Armd Bde primarily in terms
of their fuel requirements and to otherwise keep the Bde supplied.  Their 13
lorries were mainly loaded with fuel for the Bde's Sherman tanks.  Alas,
due to the heavy shelling of Queen Beach, only 9 lorries actually 
landed by 1200 hrs.  The lorries that landed proceeded to the 27 Armd Bde's 
A Echelon Area in Hermanville-Sur-Mer and would quickly be put to work keeping
the Bde supplied with fuel and ammunition.\autocite[6 June 1944]{90wd}
Hermanville, situated along the main road departing Queen Beach --- location
of the Beach Sector Stores --- became 90 Coy's main control point where
vehicles would check in before proceeding to the beaches or to the units.
%e awk

As a point of curiosity, you may have noticed that B Pl was not preloaded 
with ammunition but with POL.\autocite [6 June 1944]
[See second page of 6 June entry]{90wdjun} 
This was because the Bde brought their own ammunition ashore firstly with
the ammunition they carried in their tanks, but also with the ammunition 
they towed behind their tanks in \textit{Porpoise} sledges. %cite
These sledges
would be released shortly after the tanks made it ashore.  Collecting the
ammunition in these sledges also became one of B Pl's tasks in the first
hours of the invasion.%cite was this in 27 bde's wd?
%e assault division suggests they were quite bad
%e maybe also the private papers?
%e talk about the first day's influx of ammo ref 3 div AQ papers maint plan
% annex F

%e insert the stuff on logistical objectives (below somewhere) here
Perhaps as a happy co-incidence, Neptune had failed to meet it's D Day 
objective of pushing all the way to Caen --- an optimistic goal anyway.%cite
This meant that supply lines were shorter than planned which may have reduced 
the stress on the 9 lorries of B Pl at the cost of less space for disembarking
troops; and thus, adding to the issue of congestion.  It is difficult to 
understate how heavy the 
fighting was.  Indeed, there were many instances where tanks were replenished
with tanks still `in their forward positions', at times, with supply lorries 
advancing under tank escort.\autocites[6 June 1944]{90wdjun}[3]{90wdjun} 
This single under strength platoon was trying to keep a whole brigade supplied.
Tasks which would ordinarily been reasonably simple tasks were now incredibly
onerous.  Take for example the task of refuelling and reammunitioning the tanks.
What should have been a simple task done at the end of each day to ensure the
Brigade was ready for the next day's operations became a night long ordeal 
requiring the initiative of the 9 lorry drivers of B Pl who had to understand
the requirements of their client unit before returning to the beaches to try 
to obtain the critical stores required by their units.  It was paramount that
these drivers not only knew what was needed, but the priority of what was 
needed in the event that there were insufficient stores available to meet
an urgent order.  This way, lorries were always moving and stores were 
always flowing.  Fortunately, by nightfall on D - Day, a small Brigade supply
dump was beginning to form in Hermanville --- an act that would logistics 
chains.  Even still, this put a great strain on the men who were
worked day and night until D + 4.\autocite[2]{90wdjun}
%e convert this into showing how important supply is


% insert a transition into how B Pl was running around from BAD to units via
% Hermanville
Thus was the dispositions 90 Coy on D-Day, two Pls would make their way ashore:
onto support their parent unit, 27th Armd Bde and one help the Division to
their left --- 6th Airborne --- until their own RASC unit could make it. Here,
one can begin to see the role of 2nd line transport companies such as 90 Coy.  
They form the final interface between the wider supply system and the fighting
units ---  it is these units that \textit{deliver the goods} --- however, how
did these 90 Coy interface with the rest of Army?  
% mabybe rethink this transition

\section{The Supply Chain in the Field} 
% move to Hermanville, 90 Coy's control point in running convoys

Whilst admittedly, the supply system on D - Day did appear somewhat improvised
and ramshackle, there was good reason for this.  Because the British failed to
advance as far forward as planned, the supply dumps that were to be set up all
along Sword Beach failed to materialize in the same way as planned.  Still,
the logisticians of the British Army tried to beat the formal planned system
into an effective supply chain however much improvised.  This task was 
simplified by the fact that the British Army had an organic, built-in supply 
chain in doctrine.  This was after all, an army that could expect to be
deployed to not just fight a large, European Army, but also fight small wars
across vast stretches of the British Empire.  This formed a pre-existing 
framework allowing the logisticians to bring some level of order from the chaos
of D-Day.  In a sense, the Overlord logistics
plans were more about adapting the pre-existing supply chain to the specific
peculiarities of Overlord than about creating something truelynew.  Thus, when 
the Supply Officers of Overlord found that the land that was to become their 
depots were still occupied by the Germans, the Supply officers did not have 
to design a new supply chain, merely adapt the old plan to meet new conditions.

\subsection{The Three Lines of the \textit{Thin Red Line}}

In simplistic terms, British Army logistics divided itself into three lines, 
the first line, the second line, and the third line.  First line units 
consisted of the actual fighting units, units like the East Riding Yeomanry,
the Queen's Own Rifles, or the King's Own Scottish Borderers.  First line 
units had limited logistical capacities.  First line units function as the 
final point of distribution from the wider Army supply chain, and the actual 
\textit{man with the rifle}. First line units are able to effect minor repairs
to equipment, typically those that do-not require much in terms of skills to 
replace.  The sorts of repair that the layman could effect like replacing a 
blown fuse, or changing the spring on a rifle.  They are, in effect, end 
users.\autocite[s. 58(1)i, s. 102]{FSR1}  

Third line units are more conceptual.  They are, simply put, high-order depots. 
They exist at bases and railheads, and third line units are responsible for
 bulk-breaking, preparing stores for later distribution and issue, and are
capable of some operational warehousing.\autocite [s. 105(2)]{FSR1} 
The third line is capable of major 
repair and overhaul of equipment and, depending on the precise context of the 
third-line metaphor, can even include civilian contractors repairing things
that the field army is incapable of repairing.  In the perspective of first 
line units, third line units are people far away who fix things that are 
broken and send forward stores when they're needed.

Second line units function as the interface between the third line and the
first line.\autocite[s. 101]{FSR1}  The third line gets the supplies and 
prepares them for distribution to the units, the second line moves them to 
first line units, and the stores are used by the first line.  90 Coy RASC, 
the unit we are primarily concerned with, is a second line transport unit.

\subsection{The Principle of Supply}

The whole supply system was guided by the need to keep the Army functioning
by ensuring it had what it needed, when it needed it.  At it's core,

\begin{quotation}
	The principle of supply [in the British Army was] that field units 
	should always have
	with them, or within reach, two days' rations and forage, and one 
	iron ration, and that these stocks should be replenished by 
	delivery, at a point within reach of the troops, of one day's ration
	and forage each day. \autocite[s. 107(1)]{FSR1}
	%No27 Trg Course p. 32, quoted in,
	% refs F.S.R. Vol. I. Sec107(I). %validate this
	
	% add remark on fuel range
\end{quotation}

Moreover, as the British Army was fully mechanized by the Second World War,
supply was also to ensure that all vehicles would have full petrol tanks at 
the end of each day.\autocite[73, 79-80]{feat}
To enable operational mobility, 2nd line transport was also to have 
immediately available, an additional 50 miles of fuel; and 3rd line transport,
a further 25 miles instantly available for use.\autocite[\Petrol][s 3]
{27course}
Of course, it is unlikely that such vast quantities of fuel were landed on
D - Day, but it does provide a picture as to the standard logistical range of 
the Army.  The British Army was expected to be able to advance independent of 
it's bases for slightly over 75 mi over the course of three days.  Thus, this 
formed it's maximum operating range and maintaining this ability to move would
become the challenge faced by 90 Coy, the RASC as a whole, and the RAOC.

Of course, it is sub-optimal for an Army to operate for long without access to
its supply chain so, to support the Army, the supply chain was broken up into 
four main areas, ordered from furthest to nearest the front line, 
the Base Sub-Area(BSA), the Line of Communication Area (LoC), the Corps or 
GHQ Area, and finally, the Divisional Area.  Those depots that 90 Coy went to
along the beach?  Those would form the Beach Sub-Areas (BSA).  
%e beach or base?

\subsection{The Base/Beach Sub-Area and Line of Communication}

In the first days at 
Normandy, it appears that Beach and  Base Sub-Areas were treated as one and
the same.  Whatever the `B' stands for, BSAs functioned as the British Army's
initial interface between sea and land.  One can think of the BSA as a sort of
harbour responsible for that first ship-to-shore operation, and for storing 
and organizing those supplies for later use inland.  The BSA had 
%was it within it?
the docks, the base railway marshalling yard, a main supply depot, a petrol
sub-depot, field bakery, and detailed issue depot.\autocites
[\MaintProj][Paras 8 -- 9, see also the diagram on the recto of the first page
of the precis.]{27course}[\SupInWar][]{27course}
Cold storage was also 
available for rations such as sides of meat, etc --- of course, it is unlikely
that such niceties were available in the first days of the invasion, fresh
rations weren't even available for quite some time.\autocite[\SupInWar][See
diagram at end]{27course}

The BSA would then theoretically interface with the Line of Communication
Area (LofC).  The LofC can be thought of as a transport network connecting 
the BSA with field units.  These were railway networks or truck convoys that 
transported stores from the BSA to the field army. %cite
One peculiarity in 
Normandy however was that the supply lines were quite short, measuring in the
ones or tens of kilometres.  Until the Anglo-American forces broke out of their
beachheads, it was simply unnecessary to have a strict LofC area per se.  
The field army could simply draw stores directly from the BSA --- the LofC 
area really is not necessary until the field army is some distance away 
from the BSA. %cite
As the Allies advanced deeper into France, a more formal LofC
area would be established to convey the minutiae of war to the front.

\subsection{Supplies in the GHQ, Corps, and Divisional Areas}

%e is bulk breaking at BSA or coln or is it just third line?
Regardless of whether the Army was drawing stores directly from 
the BSAs or from the LofC, eventually,  the field army would have to start 
drawing stores.  To such ends, the Army was divided into two sections: the 
Corps / GHQ Area, and the Divisional Area.  Typically, the distance --- and 
thus, also depth of the Army --- from the LofC area to the delivery points
was 30 -- 40 mi (50 -- 65 km)  At the GHQ level, one begins to
see how the British Army sorted supplies.  POL and other stores were handled
in two theoretically separate systems.  In either case, it is at the GHQ level
that stores were bulk broken. %cite  

Let's handle the general stores first.  Stores are delivered to the Supply
Column (Sup Coln) where stores are bulk broken.  Think of this bulk breaking 
with the analogy of a grocery store.  A grocery store may receive it's goods in 
wholesale, bulk form, but then repackage it into smaller, more usable units to
be easier to sell --- a retail customer has little use for half a ton of
potatoes; however, two pounds could make a nice dinner.  
In the case of prepackaged stores, bulk breaking is more similar
to the procedure that occurs when a grocery store receives a palette of cereal
which is subsequently unpacked and loaded as single units on a shelf.  Thus,
the Sup Coln HQ can function as an interface where the Army's bulk handling 
meets it's piece handing functions.\autocite[\SupInWar][3]{27course}

\subsubsection{Petrol, Oil, and Lubricants (POL)}

%e this section's pretty awkward.  Connect better to thesis
Likewise, fuel could, at times be shipped in bulk initially however fuel for
the British Army was never delivered to field units as such.  It was always
containerized first into tins.  There are few modern equivalents to this in
our modern world.  When we buy fuel at the petrol station, we pump it from a 
massive underground tank into our cars where it's sold by volume.  Rarely do
we buy a pre-packed can of fuel.  This was however how the British Army 
preferred to receive it's fuel --- in 4 Gal (18L) of petrol per tin.\footnote
{\cite[\Petrol][3]{27course}.
For reference, the 2025 Toyota Corolla sedan has an approximately 50 l fuel 
tank whilst the 2025 Ford F150 Raptor pickup truck has a 136l tank.}%cite
These tins were nicknamed flimsies.  Flimsies were meant to be disposable so 
they were built cheap; however, the design teams were perhaps overzealous. 
The flimsies had an unfortunate habit
of breaking or leaking such that it was quite common for them to arrive 
damaged leading to fairly severe losses in fuel as well as a notable fire 
risk.\autocite[179, 198-9]{feat}
Indeed, the flimsies were of such low quality that the British Army began to 
simply use, and then copy captured German (Jerry) petrol cans --- hence the 
term jerrycan (a German petrol can).\autocite[179]{feat}
Moreover, by Overlord, jerrycans were 
plentiful and it appears that flimsies were mostly relegated to carrying 
water.  Even containerized fuel was arriving ashore already loaded in
jerrycans and images of POL dumps post D-Day depict stacks of jerrycans and
not flimsies.\autocite[8:30 -- 10:55]{buildup}

%e fix transition

Nevertheless, despite the questionable durability of flimsies, the British 
Army had some sound reasons for using containerized, as opposed to than bulk 
final distribution.  Firstly, tanker lorries in civilian use were not fit for 
military service; thus, if the British Army was to distribute fuel in bulk,
special military tanker lorries would have to be developed --- potentially at
great cost.\footnote{
\cite[178]{feat}  As it happens, the British do end up
creating these military tankers but, for reasons enumerated below, they were
most heavily used in rear areas.  Distribution to field units continued to 
use containerized distribution throughout the war.\cite[184-7]{feat}
}
Secondly,
containers are compartmentalized.  If a bullet pierces a tanker lorry, one may
loose thousands of litres of fuel before one notices; however, if a bullet 
travels through a containerized fuel transport (i.e. lorry full of flimsies),
one may loose only a few tins worth of fuel.\autocite[\Petrol][Table at para 8]
{27course}
Moreover, containerized fuel has
far fewer mechanical requirements.  For bulk fuelling to work, one must have a 
working petrol pump.  This could be quite inconvenient.  Imagine having a 
tanker load of fuel but no simple way to get the fuel out of the tanker.  
%e check this
Moreover, using this system, you can only fuel a few vehicles at a time.  
With containerized fuel, one merely pulls up to the vehicles, unload a few
tins at each vehicle, and each crew then subsequently fuels their vehicle
with a cheap easily replaceable funnel.  Filling the fuel containers could
be quite laborious in the field but this was partially mitigated by the
flimsies being pre-packaged at the factory.\autocite[178-9]{feat}

%e define POL

%e para is awk, solve later.  It reads like it was inserted after original 
% writing --- newsflash: it was lol!

%e deal with this.  Not everything was containerised.  Tanekrs did exist in 
% the rear
All told, the British POL supply chain was designed to provide containerized
fuel for the Army.  As designed, it was intended for the Army to be able to
advance the whole army 75 mi (120 km) using only such reserves held by the 
field army (the GHQ/Corps areas, and the Divisional Areas).  50 mi (80 km) 
of fuel would be held by the the Divisions, whilst the Corps areas would hold
the remaining 25 mi for the divisions, plus an additional 75 mi for the corps'
organic transport.\autocite[\SupInWar][3]{27course}

Having been bulk broken at the Corps or GHQ levels, it was now up to the 2nd
line transport units like 90 Coy to then bring those stores forward into the
Divisional areas and deliver them to the units of the end-user.  Depending on 
operational requirements, this may mean delivering it directly to the 
individual end-users, or it could mean delivering such stores to the units who 
could then further distribute stores internally.  This formed the basic, 
theoretical structure of the British Army's supply chain; however, just as how
no plan survives first contact with the enemy, the supply chain had to adapt
to tactical and operational necessities.  
%e can I tie this section tighter to the thesis?

% Already, you may have noticed that the 27th Armoured Brigade is a 
% \textit{brigade}.  Why does it have it's own 2nd line transport?  The answer
% is fairly simple, 27th Armd Bde's full name was 27th Armoured Brigade 
% (Armoured Assault). %e fact check
% The Bde was raised as an independent armoured brigade for Overlord.  As such,
% it needed a way to ensure it could run its own logistics.  You may also recall
% how 90 Coy was, on D-Day, delivering both POL as well as ammunition to 27 Armd
% Bde.  This shows how the supply chain had to remain flexible.  Whilst in 
% theory, there was a separate chain for POL and ammunition, in practice, this
% was impossible.  This was the advantage of containerized fuel as fuel could 
% simply be loaded into any available lorry.

\subsection{Storage and Dumping}
Finally, before we carry on with the affairs of 90 Coy, it may be prudent to 
clarify what is meant by a `dump' and other forms of storage.  In a perfect 
world, supply chains would be perfectly efficient.  Ever single item required
by an army would be produced when it's needed, sent to where that item was 
required without delay, and used immediately on receipt.\footnote{
	What we just discussed is known today as \textit{Just in Time} 
	Logistics pioneered by Toyota in the 1990s partly permitted by truly
	reliable, modern logistics. %cite
}
Alas, hiccups 
invariably appear.  Shipping gets stalled, major operations consume unusually
large quantities of supplies, supplies are lost to enemy action, etc.  Thus,
if first-line units were to receive a continuous flow of supplies, it was ---
and remains --- necessary to store a reasonable reserve of stores at various
points along the supply chain to absorb the normal ebb and flow of supplies.  

Ideally, this would be a large, dry, flat, climate controlled warehouse with 
good transport networks, but conditions in the field often are not always 
ideally suited to the logistician.  Thus, supplies were often stored by 
stacking supplies in a field or some woodland and covering them with tarpaulins
if they required protection from the weather. %cite
The precise requirements of this may seem quite
trivial and not terribly important to the profession of fighting wars; however, 
seemingly trivial tasks such as labelling and organizing are critical.  
Consider what would happen if there was a German counter attack and the supply
officer could not find the 76mm anti-tank shells because their boxes were
not properly labelled or because the dump was not given enough land so that 
the aisles were too narrow.  %e add WOS example
Moreover, what would happen to those same shells 
if they were dropped and the packaging was inadequate to protect their contents
--- and honestly, who hasn't dropped a heavy box before?  Damage to the shell
casing could prevent the casing from ejecting properly after firing leading to
a stoppage and possibly leading to the tank being out of action.  

Consider also what would happen if one of these these dumps was attacked and
caught fire.  Aisles do not merely provide access but function as fire breaks.
These fire breaks are critical for hazardous material dumps such as POL dumps
or ammunition dumps.  When these dumps catch fire, it is often too dangerous
to attempt to extinguish the fire --- POL burns and High Explosives explode.
Instead, standard operating procedures tend to relate to containing the fire
and letting it burn out on its own.  %e insert medal citation example

This may seem trivial but how do acts like this win wars?  Unlike the combat 
arms, logistics does not win wars by plunging a bayonet into the hearts of the
enemy.   
Instead, logistics wins wars by ensuring the combat arms can act without 
undue constraints.  If there is insufficient ammunition or fuel to support an 
advance, a General cannot order that advance.  If reserves are not ready when
the enemy attacks, then the combat arms will have few options but to withdraw
or fix bayonets.  
Logistics achieves nothing on its own but, through it's ability to impose or
relieve constraints, logistics is a significant factor in determining if an 
operation is achievable or foolhardy.  Let us return to Normandy in June of 
1944 to see this in play.
%e how's this transition?  I kinda don't like it.

	% Talk about how supply works, the DIDs, supply lines, POL differences

	% complaint about the flimsies and containerized fuel

	% talk about how you sustain client units, march length, the end goal
	% of ending the day with filled tanks, full ammo, etc.

	% Keep that story telling feeling, you're at the control point mourning
	% the fact that there are no other second line units, you're vastly 
	% overworked.  If only you had the additional units to support you.
	% if only ____ instead, you're left trying to maintains [sustainment
	% standards] and just keep what's needed flowing as hard as you can.

	% Use to establish what we're actually dealing with so importance
	% becomes self-evident
	
	% How supply chains work from dumps, and depots, to distribution, to
	% 1st line units.  


	% Key was placed on  on rabild concentration of force, incl sup, LofC
	% etc. to permit strategic use of force %cite 27 trg crs p. 32

		% 4 main areas, the BSA, L of C, Corps/GHQ area, Div area
		% goes from 

		%cite No 27 Trg Course p. 34ish (diagram's on 34)

		%\subsubsection{L of C Area}

			% Envisioned as a railway line, lorries will do 
			% just fine but this theory came at the dawn of
			% mechanization.  Point was to take stores from
			% the BSA and transport them near the combat zone.
			% It's basically a transport area.  IT goes from 
			% just outside the BSA to, and including the railhead.

			% If distance from the BSA to the railhead is < 12
			% hr trip, BSA will control.  Else, there will be
			% a regulating station around 6 hrs from the rail
			% head. %cite no27 Winter Trg p. 32 5c
			
			% If using trucks, it's 150 tons per 21 trucks with
			% old pat, 59 tonnes across 12 trucks (5 supplies, 2 
			% ord, 2 R E, 2 stores, 1 RAMC)  railway trucks 
			% %cite No 27 course, p. 32 para 5


	

	%\subsection{Warehousing}

		% Especially in those early weeks when a lot of it was dumping
		% rather than warehousing per-se.  

		% DID (maybe??) 3000 Tons/acre= gross stacking area.  x 4-10
		% for expansion %cite No 27 trg 32 (this might be a general
		% rule too)

		%\subsubsection{Base Supply Depots}
			% Located near Base Marshalling Yard, outside docks
			% to permit expansion and dispersion.  Consider `water
			% light, telephones, good roads, office accommodation'
			% when selecting the location. Area = Strength X 
			% stock X weight one ration / (unintelligible) = 
			% 8.5 sq ft /ton = stacking 
			% +50\% for stock spacing = gross stacking area X 4
			% to 10 for expansion, etc.  (that's a quote)
			%cite 27 trg course p.32, 4
	% zoom out a little and talk about what you'll be controlling.
	

% /* Figure out where to incorporate the fact that the British/Canadians focused
% on firepower over manpower.  This means materiel is critical --- A is for ammo,
% B is for beans, C cold water, D: diesel, E-everything else... */

% /* Do I want to expand to include things like traffic control?  Traffic jams
% on Sword Beach may have made the Br fail to capture Caen on D.  10m of dry
% beach between water and sea wall at high tide.  Perhaps an MP or two would
% have solved the issue. IIRC, RAF beach sqn dealt with it. (See
% RAF beach sqn/det  Was this a critical
% oversight?  Not a lack of tenacity or anything else, but a good, old fashioned
% traffic jam VI's-a-vis Toronto at rush-hour caused the failure to take Caen?
% lol --- what a way to win a war! */

\section{Return to the Moment} %e rename this
%New title, want to get back to what 90 Coy was up to till D+7
% sum up the Jun Hist Rep for 90 Coy 


By the morning of D+1, the situation for 90 Coy was slowly improving albiet,
with real delays forming for offloading troops from ship to shore.  As the
men of C Pl land, they delivered their preloaded stores to their intended 
recipients before joining the constant circut from Beach Sector Stores to
the dumps of 6 Airborne. %cite %e add examples
By the afternoon of D+1 however, fears were beginning to materialize of
a German counter attack targeted at the Eastern bridgehead presently held by
6 Airborne.  As such, all available transport in the 3rd British Infantry Area
were ordered to assist in preparing for this German counterattack on the 
British left flank. %e check, is this true?  %cite %e consider omission
Thus, C Pl continued to built up a reserve of supplies in the 6 Airborne
area, supplies that would be extremely useful if the German attack 
materialized.  %e if or when?

%e awk
Whilst C Pl was supplying 6 Airborne, B Pl was was busy establishing a reserve 
of stores for 27th Armd Bde, running up and down the congested road 
running between the Beach Sector Stores Dump and the Bde dumps at Hermanville.
Fears of a counter attack however disrupted these plans and B Pl was ordered
to transport a battalion of infantry 4--5 km East to St. Aubin d'Arquenas to 
plug the gap.
% find out the unit
At the time, the Pl was around half way through the process of unloading 
jerricans at the dump but this situation was urgent.  As such, the infantry 
battalion was ordered to climb on top of the jerrycans and they were rushed 
East.  Transport commitments fulfiled, B pl continued to build up the 
Hermanville dump and deliver stores to the forward elements of 27 Armd Bde.  

%e talk about the d-day logistical objective:  
% d+1 = 1	day reserves for troops ashore
% d+2 = 1.5	"	"	"   "      "
% d+3 = 2	"	"	"   "	   "
% d+14 = 5	"	"	"   "	   "
%cite 3 div aq pdf pg 162
The gravity of this buildup was quite a lot of work.  See, it was not enough
to simply bring ashore enough stores to sustain the troops currently ashore.
If they did so a storm or a larger than anticipated counterattack could have
been catastrophic.  Thus, it was nessessary to establish a reserve ashore.  
On D+1, the goal was to establish an additional day of supplies for all troops
ashore.  D+2, 1.5 additional days; D+3, 2 days; and by D+14, 5 days.
%cite 3div aq pdf pg 162  
This gradient is reflective of the deminishing returns of having a longer 
period without supplies as well the fact that there is an exponential 
relationship between troops ashore and the size of the reserves --- 
increasing the number of troops and the size fo the reserve at the same
time results in exponentially more stores needing to be brought ashore.
This operational imparative meant that, understrength logistics companies had 
to move far more supplies following D-Day than what would ordinarily be 
expected of full strength companies.  In practical terms, this meant working
around the clock with very little rest.  In the 60 hrs from landing on D-Day 
to D+2, B had only been permitted 1-2 hours of rest total.  By D+2, the men
were starting to fall asleep at the wheel!  Nevertheless, this foresight would 
pay off when the Allied Expeditionary Force was cutoff from the UK by a storm 
that would arrive a week later.  

This example is representative of the role of logistics in warfare.  Logistics 
contributes to military success by removing constraints, but this is often 
expressed, not by reacting to a threat per se, but by ensuring that the Army is
in a material condition to receive the enemy.  This is often done by 
prepositioning assets where they may \textit{foreseeably} be 
required whether that be by transporting troops or by establishing dumps.  
This establishment of dumps meant that supplies would be available \textit{if}
they were required.  Their success is only evident in the absence of failure.
Ultimately, the supply of ammunition was maintained, albiet, with some 
shortages later on.

% transition from general pereparedness to making of for losses

Given good transport links, creating a single large dump would greatly simplify
matters.  Entry and exit routes could be improved by the Royal Engineers to
help alleviate congestion and one-way circuits could be established.  Moreover,
having everything available in one place would mean that the probability of a
signle demand exceeding the avaialblity of stores in that one major dump was 
low.  Alas, any such dump would have to be huge and would likely be highly 
visible from the air.  This in turn would increase the probability of attack 
and the loss of materiel.  Thus, materiel needed to be dispersed across more
dumps, and across a wider area to make the supplies easier to hide and to 
minimize the risk of losses by any single attack.

Still, even with such forethought however, losses could be enormous.  Following
D-Day, the Luftwaffe made it a habit to sent sorties over the beaches to attack
anything of value.  On at 1345 on D+2, one such attack materialized on a POL
dump adjacent to the main beach exit.  The attack ignited the POL in the
dump and the resultant fire spread to near by supply and ammunition dumps.  
Over the next 
3 hours, 60000 gallons of POL and 400 tones of ammunition were consumed in 
the flames --- around 1/4 of the stores landed in a 
day.\autocite[8 June 1944]{1raf}  Efforts to extinguish the flames 
% firefighting details --- if I can find it agian! 
% Try the grey book, orange text

This was indeed, not the only fire.  POL fires dot the various various Army 
war diaries and RAF Operations Record Books as the Germans attempted to 
interdict British supply lines.  What is interesting however is that these
fires occurred sufficiently frequently that they begin to be rotuine.  Often
records simply show `P.O.L. Dump hit\ldots' followed by, `P.O.L. 
Dump fire extinguished'.\autocite[10 June 1944]{1raf}  %e suggest learning?
As the invasion continued, it became increasingly common for a 
quantified estimate to be omitted in the war diaries or operations record
books.  We can assume that, as these men continued to work, they learned to 
take better fire percautions such as prepositioning firefighting apparatus, 
establishing sufficiently wide firebreaks in between stacks of POL, and 
dispersing the storage locations for these dumps helped to minimize
losses.  

Whilst such percautions are admittedly, quite mundane, it is preparations such 
as these that are essential to keep an army mobile.  Consider that 90 Coy was, 
as these fires were raging, running loads of petrol forward for the tanks.  
If losses were not averted and minimized, a single fire could be allowed to
destroy a catastrophic amount of fuel.  As it was, the loss of fuel was a mere
inconvinience.  First line units still recieved enough fuel to operate.  
Once again, it is rare that logistics can win a war, but it can certainly 
loose it.  Without these standard, boring preparations taking place, it is 
probable that the British Army of 1944 would have simply been unable to fight 
in Normandy as it would have been much easier for the Germans to simply destroy 
the buildups the British were making.  Whilst these stacking and loading 
standards are quite mundane, they are important to actually winning wars.

% bad planning
Consider also unforeseen events where problems could not simply have been
planned out of existance.  The paras 6 Airborne fighting East of the 
Orne would, due to the general difficulties in providing sustainment from 
the air, often found itself short of rations or ammunition.  Mathamatically,
supplying these stores should be simple.  You know the strength of a division, 
you know how many days of rations to provide them and some simple multiplication 
reveals the number of meals required.  Take the number of meals, divide by the 
number of rations in a case, divide that by the number of cases that will fit 
in a lorry, all all that's left to do is to find the rations, load up the 
lorries and go.  Job done!  Nice and easy!

Alas, if only life was so simple!  See, dumps had to supply these rations
and this math is only accurate if the supply officers knew how many men they
had to feed.  Typically, this is solved by storing an excess of rations at 
these dumps to make up for any shortfall; however, in the first days of the
invasion, rations were in short supply so these reserves that would have 
been prudent to build up simply had not had time to amass ashore.  Thus, on
in the evening of D + 1 when Commander RASC (CRASC) 6 Airborne Division --- 
the officer in charge of supplies for 6 Airborne
--- found out that they had been reinforced and that these reinforcements 
were to be fed by him, he would have had his staff check their supplies.  
His team would have informed him that they simply did not have the rations 
available.\autocite[6]{90wdjun}
What would have then likely happened was that he would calculate 
the rations required, put a message through to Beach Sector Stores and request 
those rations.  This would set into motion several chains of events from 
clerks and officers nervously eyeing ledgers, making sure that this requisition
could actually be met off hand.  If it could not, they would be figuring out 
where they could squeeze from the supply system for a little extra.  Maybe
transfer stores from a different dump, maybe reduce the size of a shipment for
the next morning in hopes that they could fill their evening request, etc.  

Whilst all this was happening, transport officers would be liaising with 
transport units like 90 Company and pushing through orders to arrange for 
the transport (in this case, three vehicles) to then get those rations from
BSS to the end user.  CRASC 6 Airborne whilst all this was happening would be
ensuring he actually had room to put the rations once they were delivered, 
figuring out how to ensure his new troops knew where and when draw stores,
etc.  This back and forth is simple work.  The stores existed but they're 
just not where they were needed at the right time.  Thus, all that needed
to be done was find the supplies --- not hard with stores as ubiquotous as
rations --- and deliver it.  

What happens however if the supplies required simply do-not exist in the
quantities required ashore?  By the afternoon of D + 2, 6 Airborne was 
growing short of 75 mm Pack Howitzer shells.\autocite[7]{90wdjun}
As such, 6 Airborne requested that 2000 rounds be dropped during Operation 
Robroy Three --- the third in a series of four operations intended to supply 
6 Airborne by air over the first four days of Operation Overlord.\autocite
[Appendix K to June 1944][1]{6aqwd} %pdf page 73
Due to poor weather, Robroy Three was cancelled though not before five 
aircraft had already taken off with small arms ammunition and wireless sets.
As such reserves of these shells simply did not exist ashore.\autocite
[Appendix K to June 1944][2]{6aqwd}  Thus, 6 Airborne made some inquires with 
the Navy and an officer of 90 Coy was sent to 
the Navy's Command Post to liaise with them as they attempted to locate the 
stores.  Some of these shells were supposed to have been landed some time
during the first few days of Overlord but none could be 
found ashore.\autocite[7]{90wdjun}

%e talk about hand searcing too (wandering searches)
Locating stores in 1944 was difficult.  It was not like today where one can 
search a database for the stores required, find which ship the shells are on, 
and just ask that ship to expedite that delivery.  It required hours going 
through reams of paperwork trying to locate a single line in a ledger but, 
until 
someone worked out which ship these shells had been loaded onto, the paras
would not be able to use their artillery. 

By the morning of D + 3, these shells were still nowhere to be found and 6
Airborne was beginning to grow desperate.  We will discuss the importance of
artillery later, but sufficed to say, the British were reliant on their
guns.  They were so desperate indeed that, that morning, 6 lorries of 90 Coy
were held so that instant the shells made it 
ashore, they could be sped to 6 Airborne's gun lines.  To permit this, CRASC
6 Airborne made special arrangements with Beach Control to allow the DUKWs
--- amphibious lorries --- to make an inland delivery (typically the DUKWs
are just used as ferries to Beach Sector Stores as any lorry can drive far
inland but few lorries can drive into the English Channel without severe
consequences).  %cite factcheck
Thus, when the ammunition was finally located on the afternoon of D + 3 by
6 Airborne RASC HQ's Ammunition Officer, Navy contacted the reliant ship,
the ship unloaded her stores into the DUKWs, and the
DUKWs drove directly to 90 Coy's Colleville harbour, the ammunition was 
cross loaded onto 90 Coy's 3 tonners, and that ammunition was rushed to 6
Airborne's gun lines which were, at the time, stood to and actively engaged 
with repelling a German attack.\autocite[7-8]{90wdjun}
The German attack was successfully repulsed
by element's of 27 Armd Bde --- also supported by 90 Coy.  It was not until 
the next day, D + 4, that 6 Airborne's own RASC transport made contact with
their parent unit.  Until that time, the 46 lorries of 90 Coy (reduced to 20 
by D + 4) had been supporting two divisions and one Brigade, a force which 
would have been undermanned to support even a single Brigade.  

Think about what it thus meant that 6 lorries (around 1/4 of 90 Coy's remaining
strength at the time) was held, standing by to ferry that 75 mm ammunition 
instead of delivering other critically needed stores --- granted, by this time,
some of
the 3rd British Infantry Division's own transport had landed as well.  What
would have happened if those shells were not located?  6 Airborne would have
lost much of its artillery support.  Moreover, think about how complex it was
to locate and deliver even a single load of artillery.  Teams involved included
at least 6 Airborne's CRASC (at least one officer and a few NCOs), the Navy
Command Post (at least one officer, a clerk, and a signaller), at least one
officer and six drivers from 90 Company, likely around six DUKW drivers, the
teams at sea loading and unloading cargo, the Beach Control point, dozens of 
MPs controlling traffic, and doubtless more I have yet to think of.  Whilst 
the combat arms rightly gets much of the credit for fighting wars, and the
Generals credited for figuring out where to put men, spare a thought to the
staff work done by the men keeping ledgers, speaking on the radio, 
co-coordinating actions and pushing forward supplies.  When times are desperate, 
one not only needs brave men, but highly organized logisticians to ensure
that which was needed was obtained. Why else would you have drivers driving
almost non-stop for some 60 hours if their work could be ignored?
%e this is kinda messy. is there a better way?  Also, tie together the
% rations thing too.

%e do I want to mention how but D+3 0815, plans to open BMA in original 
% planned locations was decided?  \autocite[9 June 1944]{1raf}

%e Maybe talk about historiography a little?

\section{Operations to Hold Ranville} %name, if any?  Was hasty operation.  

Based on our impression of the first few days of the invasion, you would be 
forgiven for thinking that supply in general was quite a ramshackle affair.
Thus far, the picture is probably exhausted lorry drivers ferrying materiel and
troops this way and that, creating hasty dumps of essential stores, with busy
supply officers running this way and that trying to scrape together what 
resources resources were available to support operations; however, as the
situation stabilized in Normandy, supply slowly starts to become more regular
and these quick and hasty names I keep bringing up like Hermanville, the
6 Airborne's Dumps, etc. start to become more important.  It is thus
worth pausing to assess the situation and to put some order to the chaos and
really consolidate the supply chain that both we and 90 Coy were working to 
navigate. 

\subsection{The Supply Chain to Ranville}

With the exception of the Paras who were being partly supplied by air, the
supply chain supported by 90 Coy --- at least, as far as the Coy was concerned
--- originated from sea on the various transport ships loaded down with any 
number of stores.  These could be landing craft, landing ships, or any other
vessel capable of carrying a large volume and tonnage of cargo.  If these
ships such as the LST could be beached directly ashore, then they were 
typically beached and their stores discharged via their bow ramps.  These 
supplies were then taken to the Beach Sector Stores where they would be 
stacked in an organized manner taking into account the need for creating
aisles for both access, and fire protection.  

If the ships however could not beach themselves, then the stores could be
brought ashore either by Rhino ferry, or DUKW (pronounced `duck').  As 
mentioned before, the DUKW was an amphibious lorry with a 5000 lbs 
payload --- 2.25 tons --- or a tad smaller than the 3 tonners used by 90 Coy.
Whilst DUKWs were amphibious, they were really not designed to be driven for
long distances ashore over rough terrain.  Moreover, the diverting of such
special vehicles from their amphibious role lead to a shortage of DUKWs on
the beaches of Sicily.\autocite[87-9]{wos}%e double check --- too tired
Indeed, at Salerno the diversion of DUKWs from their task was so problematic 
that American \mbox{Vice Admiral Hewitt} advised that, in future operations,
DUKWs ought to fall under command of the Navy.\autocite[108]{wos}
For the British, DUKW drivers continued to belong to Army units however, 
British DUKW drivers were provided with a copy of written orders, 
`signed by the DA \& QMG 1 Corps', to be presented to anyone diverting them 
from their purpose, that they would be court marshalled if such a diversion 
was not an `operational emergency'.\autocites
	[
		27 Armd. Bde Maintenance Project - 
		2nd Edition (Appendix to May 1944)
	]
	[
		Section 14 para 3
	]{27wd}
	[
		Quotes from 3 Br Inf Div Adm Plan - 
		Second Edition (Appendix to May 1944)
	]
	[
		Section 12 Para 8
	]{3aqwd} %cite court marshal search
DUKWs were mainly used to transport stores from ship to the supply
dumps nearest the beach --- any lorry can drive several miles inland but driving
into the sea with a common 3 tonner is unwise.  Of course, in emergency 
situations as we have already seen with the shipment of 75 mm pack howitzer
shells, occasional exceptions would be made but it was generally best to use 
the DUKWs to fulfill the mission that only a DUKW could achieve.

DUKWs were useful for moving things that would fit in a lorry; however, for
transporting vehicles or if there was simply a shortage of DUKWs, then Rhino
ferries were used.  The Rhinos were essentially ungainly, spartan shallow draft 
barges assembled from sheetmetal pontoons.%e barges if motors?
They were typically 
moved with rhino tugs going back and forth between from ship to shore and back
again though they did also have two motors allowing them to sail at 2-3 
knots.\autocites
[200, see also annotation on Rhino ferries on pages 8-9 of plates]{neptune}
[144, 166]{transportation}
Rhinos had the advantage over DUKWs that they could take several 
vehicles on board at a time and, once beached, the vehicles could just be
driven off and any stores in those vehicles, offloaded at the sector stores
dumps as they drove past; however, the Rhinos were very 
unmanuverable and had an extremly low freeboard leading to the passengers 
often getting quite wet.\autocite[310]{neptune}

In any case, however the stores were brought from ship to shore, 
notwithstanding preloads destined for units deeper inland, their first
port of call in these first days of the invasion would have been the Beach 
Sector Stores.  This would rapidly evolve into the fully fledged Base/Beach
Maintenance Area (BMA) Moon controlled by 101 Beach Sub Area.\autocite[See
traces in Neptune No. 1 RAF Beach Squadron Operation Order found in Appendix C
to the May records.  Traces are located between the Operation Order and the 
Signals Plan that follows the orders.  Traces use a derivative of map sheet
7F][]%cite how do I do maps?
{1raf}  BMA Moon started along Sword Beach's Peter, Queen, and Roger
sectors and extended around 2km inland.  The full BMA with it's organized
supply dumps do not appear to have been fully developed by D + 2; however,
those dumps 90 Coy created as a Brigade ammunition dump in the vicinity of
Hermanville was likely on the land that became BMA Moon's ammunition 
dump.\autocite[Trace of BMA Moon
annexed to Neptune RAF Beach Squadron Operation Order found in]
[Legend entry 67]{1raf}   From these first dumps, logistics units like 90 Coy
would then transfer the necessary stores to dumps further inland essentially
forming a chain of operational reserves.  For example, take 6 Airborne's 
Ranville maintenance area mainly drew stores from Hermanville and units 
working in 6 Airborne's Area of Operations (AO) would then draw stores from
the Ranville dump forming smaller, often less formal dumps along the way.

%e turn it into an argument?

\subsection{Ranville}

% how do I transition to mostly Ranville focus?

% Maybe just consolidations?  Oh, how about the Dumps at Ranville

% give context for the situation IVO Ranville at the time.  British holding
% several KM^2 E of the Orne at IVO Ranville.  This land was supplied from main
% beaches over the Benouville-Ranville bridges.  Single point of failure.  
% fear of DE cntr attack as DE tps moving IVO E of region probing for points of
% failure.  27th Armd Bde involved in supporting local infantry.  90 Coy
% supporting Bde

% Operation with Paras of 13/18th with paras, 90 Coy dumping stores in support

How these dumps grow and evolve becomes of interest to the to the historian
of logistics because of what it shows us about how logisticians prepared to 
meet every likely eventuality; thus, let us return to Ranville.  Recall that 
the Germans were probing the area to see if they can dislodge the British and 
6 Airborne of their lodgement North-East of Caen and East of the River Orne
and the Caen Canal. The paras had been holding onto a number of disunited
pockets surrounding their objectives and drop zones.  At the time, the 
territory held by the paras was still quite disunited and there was no
continuous British front line per-se but pockets of British troops securing
local perimeters.  This is not a problem per se, rifle fire can have a range
exceeding 1000 m so there was no strict need to maintain a continuous line.  
Nevertheless, it did mean that the more weakly held areas in the British 
zone were subject to German attack or infiltration.  

On D + 4 (10 June), exactly this happened in the fields roughly between 
Ranville, and a town 2 -- 3 km to the North East called Breville.  The
Germans had managed to break into a DZ from Breville but their attempt
to cross the DZ was repulsed.  Having been repulsed, the Germans contented
themselves with holding a a wood near Le Mariquet using around a company
of troops.  The significance of 
this position is that it would separate `the 5th and 3rd Para Bdes, which 
had not actually made contact at this stage'.\autocite[June, Appendix 2, p 1]
{7parawd}  In light of this, 7 Para battalion, at the time holding the 
South-West corner of the drop zone (DZ) was ordered to `sweep the woods and
to clear the enemy out of them' and to do all of this in the pouring 
rain.\autocite[June 1944, Appendix 2, p 1]{7parawd}  The paras, having no 
organic armoured units, was to be supported by B Sqn, 13/18th Hussars, 27th 
Armd Bde as well as the 13/18th's Recce (reconnaissance) Troop 
(Tp).\autocites[June 1944, Appendix 2, p 1]{7parawd}[10 June 1944]{1318wd}

The plan of attack was simple.  The wood was divided into four separate
woods named W, X, Y, and Z and, at 1600 hrs, the infantry and armour would 
work together to sweep the woods.  
At this stage, we would normally expect to discuss 
infantry-armour co-operation --- it was awful with the paras not even
realizing how many tanks would be supporting them.\autocite[June 1944, 
Appendix 2, p 2] {7parawd}
We could then discuss how, despite the loss of 4 Sheramns and 2 Stuarts to 
German anti-tank guns, the attack was successful in clearing the wood and 
capturing `over 100 P[risoners of] W[ar]' and greatly improving the moral
of the Paras.\autocite[10 June 1944]{1318wd}
What is far more interesting 
however, is what came next.  
%e this has a lot of tactical details, are they nessessary?

The next day, 11 June, 13/18th Hussars joined the rest of 27 Armd Bde on the
ridge north of Periers-sur le Dan; however, B Sqn, the same Sqn that supported
the Paras' attack the day earlier, remained with the paras.  Next day, 12 June,
the balance of 13/18 Hussars join B Sqn and are attached to 6 Airborne but would
be supported by 27 Armd Bde.  This meant that 90 Coy was now responsible for
not only 27 Armd Bde located on the ridge between Hermanville and Periers-sur
le Dan, but also for maintaining the 13/18th Hussars operating in the vicinity
of Ranville.    Over the next few days, the 13/18th Hussars would support a 
variety of British units in the vicinity of Breville who's effect was to 
neutralize the threat of a German attack on the Eastern flank of the Allied
beachhead.

Meanwhile, for 90 Company, 11 June was fairly quiet.  Their activities for the
day simply consisted of a mere 5 lorry loads of general supplies for 27 Armd
Bde.  As such, the under strength Coy took the time to do some maintenance 
having been worked to the bone since D-Day keeping 27th Armd Bde and 6 Airborne
supplied.\autocite[11 June 1944]{90wd}  Given the light day, it is likely that
the tired men of 90 Coy also took a moment for themselves and got some more 
sleep or penned a letter to their friends and family.  The next day would also 
come with some pleasantries for, for the first time since boarding the landing 
ships from 1-3 June, the company at last received letters from 
home.\autocite[12 June 1944]{90wd} %e talk about casualties
There must have been a simple human joy in hearing from one's friends and 
family.  Captains Grey and Foreman must also have been quite pleased for, 
in this correspondence, they were nominated for recognition (i.e. nominated
for a medal) for their actions supporting 27 Armd Bde, and supporting 6 
Airborne division respectively only a few days earlier.  Finally, L/Cpl Jones
--- no known relation to the L/Cpl Jones of Dad's Army fame --- was nominated
for an award after rendering first aid during an air-raid on the night of 9 
and 10 June.\autocite[12 June 1944]{90wd}

It is worth noting that letters did not always bring joy.  The mail was how 
soldiers on active service received news of injuries, illnesses, and deaths 
from home.  Moreover, some letters might be from girl friends ending relations, 
or spouses recounting the difficulties of live at home in war time. 
Unfortunately, the sources I had access to included remarkably little 
personal correspondence; indeed, none between family member's and I am thus 
not in a position to make significant comment on this aspect of the war.
Nevertheless, when, in our modern world, we can usually communicate with our
friends and family pulling a smartphone out of our bags and sending a quick
text, the situation was not like that in 1944.  When the men were sealed
in transit camps pending embarkation in advance of the invasion in late May
and early June, they were largely cut off from those outside their unit.  
It is likely why this seemingly insignificant event was included in a war
diary entry was a page and a half long, as opposed to the more common several
entries per page.  

Despite the pleasantries however, there was still work to be done. In light of 
the 13/18th's attachment to 6 Airborne, on the 12th, 90 Coy started to create
a series on ammunition and POL dumps in the vicinity of Ranville to supply the 
13/18th operating in the area.  Moreover, in light of the supply chain's single 
point failure along the Benouville-Ranville road, these dumps would also serve
as an operational reserve in case the 13/18th were cut off.\autocite
[12 June 1944] {90wd}  The actual process of lorries moving to dumps and 
collecting stores started at 1800 hrs; however, it is worth also thinking about
the volume of work done by officers ahead of time.  Doubtless, a number of 
staff officers at the Company or Brigade levels would have calculated the required 
quantities of ammunition, POL, and rations likely prudent to keep on hand at 
Ranville, making forecasts of ammunition and fuel draw, etc.  They would use
mathimatical guidance --- fuel consumption is fairly predictable --- but 
doubtless also a level of judgment.  After all, on 12 June 1944, six days after 
the start of the invasion, no-one could be certain how much ammunition would
actually be consumed in this theatre of war.  

%e push log vs pull log.  Was this push or pull?  I imagine pull?  Consider
% the 75 pack howitzer shells
Having made such a judgment, these officers would have likely filed indents 
with the BMA.  %e validate
The BMA would then have to see if they could supply the the stores requested
on the indents.  Just as occurred earlier with the 6th Airborne supplies, if 
they could prudently supply the materiel, all's well.  Simply prepare the
stores to be picked up, and arrange a convenient time to draw the 
stores.  %e check
If
they could not however, there would doubtless have been efforts made across the
supply chain to acquire these stores and, only if this was impracticable, would 
it be likely that the request was denied.

Whilst all this was happening at the BMA, logisticians at 27 Armd Bde or 90
Coy must have been calculating the required number of lorries, the available
number for making the shipment, etc.  Evidently, the request was approved and
90 Coy decided it could spare 12 lorries for this dumping operation.\autocite
[12 June 1944]{90wd}  The operation continued to the next day, 13 June, when
 the Company's commitment increased to 20 lorries % as a proportion?
to the dumping operations completing the dumping program some time that day.  
The Company managed to get some rest on the 14th where, beyond some small 
deliveries, the Company had a maintenance day to look after themselves and, 
more importantly, their lorries.\autocite[14 June 1944]{90wd} 

% ask audience to consider the scale of work.
% what was the combat arms getting up to at this point?

One now might ask, why all this activity in the area around Ranville?  
Operations around this time to capture the wood are today remembered as the
Battle of Breville; however, this gives the impression of a set-piece battle
which this battle was not.  Instead, this was a brief period of fighting
surrounding this town which turned out to have strategic importance.  What 
started as a firefight to be handled by a unit fighting in their area of 
operations, evolved into a strategically significant battle involving units
drawn in from other divisions.  From a fighting standpoint, this has some 
minor interoperability concerns as well as some chain-of-command issues; 
however, simple co-operative measures such as the placing of the 13/18
Hussars under the command of 6th Airborne smoothed over these issues.  %cite

Sustainment however is a larger issue.  Place yourself as a supply officer in
this situation.  When someone moves an infantry battalion into your area,
one moves mouths to feed and rifles to fire.  Moving a unit of infantry into
an area already dominated by infantry does not cause a fundamental shift in
requirements.  All that has to happen is that the supply chain must expand to
be able to meet the requirements --- itself a challenge but less problematic
than what happens if you move units with new sustainment requirements. The 
issue is that a Second World War British infantry division typically did not 
have organic armour --- certainly not the Paras nor 51 Highland division also
operating in the vicinity of Ranville.  This means that the supply chain 
must now be prepared not only for increased volumes, but also for different 
supplies.  An infantry division has lower POL requirements than
an armoured division and the ammunition requirements change --- infantry have 
no use for 75 mm tank rounds.  Moreover, tank units are tied to the supply 
chain in ways the infantry is not.  Infantry can forage and men can be put on 
half rations for short durations without significant consequences; however, a 
moving tank will always consume roughly the same amount of fuel if driven the 
same way, on the same terrain.  

Thus, if one wishes to use tanks --- tanks being quite useful in warfare
during the Second World War --- one must have sound logistics.  This is
where the flexibility of logistics units come in play.  At this point, 90 Coy
still has a mere %number
vehicles; however, forethought, contingency planning, and adaptability was
doubtlessly helpful.  Detaching 13/18 Hussars from 27 Armd Bde was simple for
the combat arms but 90 Coy needed to think deeper.  It had to think about how
to schedule supply runs some five kilometres away from the main body of the 
Brigade along busy roads which doubtlessly meant traffic jams.  Moreover, 
it had to consider contingencies.  What would happen if
the bridges at Benouville or Ranville were taken out of service and the 
13/18th's sector was attacked?  Bridging units were available and standing by
for such contingencies but building a bridge under active air attack is not
an enviable task.  In light of this, the decision was taken to expand the
dumps at Ranville so that it could support a few infantry divisions as well as
an armoured regiment. Doubtless, it was helpful that 90 Coy and the Paras 
likely already had a close working relationship seeing as how, just a few
days earlier, it was 90 Coy that supplied them; however, it is likely that
some significant effort was needed in order to establish and maintain the
new dump.  In a sense, whereas moving an combat arms unit is akin to moving
a body of men, moving the supply chain involves setting up new infrastructure
and it is this infrastructure that is critical for the effective conduct of 
modern war.  Once again, this work to maintian current requirements and to
prepare for future eventualities is the key contribution of logistics units.
%e is this repetitive?  Can I play with this idea of infrastructure as war?
%e sort out transition

	% 12 June, mail arrives, first letters since D Day

% 13/18 moved E of river to defend against counter attack %cite 13/18 WD 12 Jun
% A, B Sqns supporting attack of German positions by 7 Para & black watch.  
% on 10 Jun, Germans were seating themselves into a wood IVO Le Marquette, 
% threatened to separate 5 & 3 para Bde as well as supporting infantry units.  
% Paras ordered to sweep woods to dislodge Germans with 13/18th in support.
% part of wider fears of a German counter attack.  
%cite 7 para jun WD appendix p. 5
% 90 Coy established a dump using 12x 3tonners.  Mainly Amn & POL, started
% 12 1800 Jun 44.  Increases to 20x lorries next day (how many lorries in 
% a coy again?)
% dump at Ranville (1173) %cite 90 coy WD 12 Jun 
% Do I wanna mention decorations here for 90 Coy or maybe do it earlier
% IVO Ranville.  Positioning of this dump wise.  Only route to supply troops
% E of Orne and canal is via Benouville/Ranville bridges.  This would permit
% bridgehead to keep fighting if the bridges were destroyed.  Infantry armour
% co-operation was a noted problem.  6 tks KO.  Paras take 40 prisoners, 0 KIA,
% 9 Wounded. 
%cite 7 para jun WD appendix p. 6


% Highlight that preparation is key.  If there was a counter attack and the
% bridge was blown, gallantry matters naught if you aren't supplied.  What 
% will you do, fix bayonets --- well, yes actually...?
% C Pl 90 Coy continue to provide light support till 16 Jun

\section{The Arrivals of A \& B Pls (14 -- 23 June)}
% non critical section, setting the tone of a lull.  A/B Pl arrive 
% 15 2000 Jun 44.  Go to Coy HQ in Cresserons GR 0379.  Brings with them 
% 59 Veh, 165 Pers.  Arrives 2300 hrs. Some DE bombing.  Over the next 3 days,
% new platoons involved in dewatering and reorganizing in prep for ops.  
% sporadic bombing and shelling of veh pk.  Veh dispersed 75 yds between 
% each veh --- quite exposed with no protection then (they're in an open 
% field).  

The Battle of Breville and establishment of the Breville dump having been
completed, both B and C platoons of 90 Coy spent the 14th of June maintaining 
their vehicles and, doubtless, getting some rest at Coy HQ located in a field,
some 500 m NW of Cresserons.  Here and there, the Coy
do some minor transport details --- delivering rations, ammunition, fuel, the
usual minutiae of war --- but the situation was quiet.  The next few days are 
fairly quiet for the Brigade.  Most of its forces are in defensive positions 
across the 3rd British Infantry Division front north of Caen or in the area 
East of the Orne.  Here and there, the Brigade takes small action defeating 
German strong points or repelling minor attacks but nothing that, from a 
logistical standpoint, couldn't be managed through the usual supply runs.

Back in England, A, and the part of B platoon 90 Coy RASC that did not land on 
D-Day were, at this time, mounting their lorries and driving onto the LSTs 
for a their channel crossing.  The 59 vehicles and 165 personnel of this 
platoon group arrives and begins disembarking at Queen Beach around 2000 hrs 
on the 15th.  Three hours later, they finish the 6-7 km journey to 
Cresserons to join the rest of
the company.
Their arrival doubtless involved the greetings of friends, as
well as some good natured ribbing experienced by new troops joining old 
troops.  There must have been questions asking about the present situation,
the location of latrines, mess arrangements, and the usual questions one asks 
living in the field; however, the sporadic bombing likely helped to emphasize
the fact that there was indeed \textit{a war on}.\autocite[15 June 1944]{90wd}
In light of this, the 
Company dig slit trenches to provide some cover against bombardment.  

Withe the new intake of vehicles and men, the Coy spend the next few days
reorganizing and dewaterproofing their new vehicles and doubtlessly, handling
routine supply runs, all whilst being sporadically shelled.  Beyond slit 
trenches there was little to be done beyond spread out the vehicles with 75 yds
between them to minimize the damage of a single bomb or shell.  One must wonder
what a dreadful inconvenience this must have been to have to go possibly
hundreds of metres just to get to one's lorry.  In addition, one wonders the
nature of the earth works in these areas as, with such dispersed vehicles, it
must have been dreadfully open to have been caught in the open during a 
shelling.  This harassing fire must have been irritating as the Germans did 
not do very much heavy shelling.  Instead, using 17 June as an example, the
Germans would lob a few shells (six in this case) over the course of a day and 
hope they hit something.  One wonders if slit trenches were dug at every vehicle
or if you hear the whistle of an incoming shell, if you just lie down and 
pray.  It is likely that sleeping positions were in slit trenches --- indeed,
some REME units even managed to scrounge beds and place them in trenches with
armoured sheeting above them %cite 
--- but even a trench was not always enough to protect the men.  Every few 
days, a man would be evacuated with wounds from shelling or 
bombing.\autocite
	[Consider entries from 
	10 June (6 wounded), 
	13 June (1 wounded), 
	17 June (1 wounded), 
	21 June (1 wounded).  Casualties were heavier at the start of the
	month but eased up towards the end]{90wd}

As an aside, I should 
note that when I say `dewaterproof', it's not so much making it so that the 
vehicles would not leak, but that they removed a series of minor modifications 
made to their lorries to ensure they would not be damaged during the crossing 
of the English Channel as well as when they waded ashore.  Unmodified vehicles
could typically wade some 18'', but modifications were made to all 
vehicles involved in Overlord to permit them to wade in up to 4'6'' of 
water.\autocite[165]{com-ops-org}
Much of this work
was done by the Woman's Army Corps.\footnote{Indeed, the role of women in WW2 
logistics is an opportunity for developing our understanding of the role of
women win the Second World War but alas, women were only deployed to NW Europe
in limited quantities; thus, this area is out of the scope of this study.
\cite[122-6]{brit-oth-army}
}   
Whilst these modifications allowed vehicles
to operate in water and protected them from the ravages of the 
ocean.  Some work involved sealing certain components or the removal of filters
--- filters that could clog when wet. 
These modifications had to be removed before the vehicles drove too 
many miles as the modifications could be harmful to the vehicles on dry 
land.\autocite[Adm Instr No 2. 16 Jun 194]{6aqwd}  

% Queen Beach still being shelled.  (appears both in 90 Coy and 13/18WD)
% buildup thus slowed.  13/18th don't get their D+9-10 residues by 18 June
% This is a failure in supply.  Delays due to weather.

% Staffs short of running viehicles by 17 June 
% \autocite[17 June 1944]{staffswd}

% Generally fairly quiet for Bde till end of month %cite 27 Armd Bde WD
% quiet for 90 Coy until 23 Jun %cite WD

\subsection{Oh Mundanity!}

% Maybe talk about the perhapses?  Amn still was consumed, as was POL, in 
% addition, what about general transport of rations, etc.  I don't have
% a src, but these must have happened.

% Talk about that wonderfully dull stuff like stove and flour shortages in
% the 27th Bde WD Admin instructions (and defecate in the latrines!).  
% Also, traffic flow patterns, spare parts, uniforms, water points etc.
% Highlight the importance of the ordinary and mundane. 

% Talk about how a POL lorry catches fire in the 13/18 rgmt area on 20 Jun.  
% Spreads to Amn dump.  %cite 13/18WD
% Not in 90 Coy's WD.  Why not, routine fetch and carry.  Fire happens, 
% amn is replaced, maybe only a few lorry loads, no bother.  Talk about
% the need to look between the sources.  An armoured rgmt is unlikely to
% have the integral transport to replenish that dump and frankly, it's not
% the job of 1st line units to maintain a 2nd line dump.  If you just read
% the WD, you would assume they're twiddling their thumbs most days.  This
% is unimaginable!

As you can likely begin to infer, late June was not a busy time for 27
Armd Bde.  Beyond sporadic fighting, there is little of note to the tactical
situation and thus, supply runs were still taking place. However, during this
brief stabilization in 27 Armd Bde's AO, one begins to see a return to the
normality of military life as captured by the Bde's administrative 
orders.\autocite[See end of June diaries]{27wd}
Indeed, the Brigade's first Admin O was not issued until the start of this
period on 14 June likely because the Bde was simply far too busy.  
Nevertheless, these orders provide a wonderful opportunity to examine daily
life for 90 Coy and indeed, the whole of 27 Armd Bde.  Moreover, it will
allow us to begin to explore two core roles of the troops of the British Army
that were not in 'G' branch --- i.e. the fighting services --- namely
\begin{quote}

	iii. The excercise of foresight to ensre the timely anticipation of
	difficulties likely to be experienced, or of material likely to be
	required by fighting troops and services in the execution of orders

	[and]

	iv. The arragements of all matters with a view to removing anticipated
	difficulties and facilitiating the prosecution of the commander's
	plan of operations.\autocite[s. 13(iii -- iv)]{FSR1}
\end{quote}
%FSR vol 1 chap 3 sec 13 iv.

On a light hearted note, is perhaps revealing that it had to be said that 
`Latrine trenches must
not be allowed to fill up.  Fresh trenches must be dug and the old sites
clearly marked'.\autocite[June Adm Order No. 3][Para 10]{27wd}  Apparently,
this was quite a problem as, two days later, the Bde was advised that, 
`Attention will be paid not only to properly constituted latrine erections but
also to the general sanitary condition of the area, particularly checking 
failure to use facilities provided' --- clearly, there was an issue getting 
the men to use the latrines provided.\autocite[June Adm Order No. 4, Para 1a]
{27wd}  
More over, it appears it is indeed true that old habits die hard for, on 
14 June, the whole Bde had to be reminded to drive on the right side of the
road, and to turn on the correct side.\autocite[June Adm Order No. 1][Para 9]
{27wd}  It was also with some amusement on reading that `Any livestock 
\textit{accidentally} killed by shell or [Small Arms] fire may be cut up 
and eaten by units if bled fresh and in good condition' (emphasis added);
however, one is left wondering just how accidental some of these killings
were.\autocite[June Adm Order No. 1][Para 10]{27wd}  
By this time, the men may not have had fresh food for over a week and the
compo-rations must have been getting monotonous by this point.

%e add remark on water.  See 6 airborne Aq pdf pg 79 (WO171/426)
Beyond these more humorous examples however, these Admin Os reveal a 
situation of scarcity.  Regarding food, whilst compo-rations were 
monotonous, the troops were forbidden from purchasing fresh bread from the
French civilians.  `Flour for civilians [was] in short supply.  If troops buy
bread it will cause a serious shortage'.\autocite
[June Adm Order No. 2.][Para 1]{27wd}  Whilst disciplinary action was 
threatened, it was not unknown for troops to scrounge for food anyway.

Moreover, water was rationed to a scale of `half
a gallon per man per day' for the able bodied, and `2 gallons per man per day
for wounded' (2.27 L and 9 L respectively).  This would have resulted in a 
water requirement of %number
L per day.  One jerry can has a capacity of 20 L; thus, %number
jerricans of water per day.
%cite bde maint proj 2nd ed section 13 para 1 calculate the strength of units
In the Brigade area, there were only
three water points from which units could draw water:  Benouville, Colleville
Sur Orne, and Hermanville.  Thus, along 27 Armd Bde lines, some units or 
detachments may have been over 2 km away from the nearest water 
point.\autocite[June Adm Order No. 1][Para 4]{27wd}  These water points would
have to be shared with all other troops in the area.  To get water,
every day, either 90 Coy or the units would have had to drive dozens of water
cans to the nearest water point, fill them, then drive all the way back
consuming both time and fuel.  Ration parties and ordinance stores would also
have daily delivery runs which allows us to start to see the baseline problem
of sustainment.%cite %e add FSR or stuff from 90 Coy's August disbandment stuff  

Ration requirements are easy to forecast, simply count the number of mouths to 
feed, multiply by the number of meals between supply runs, divide by the number
of meals in a case, and round up to the nearest whole case.  General stores 
such as ordinance stores however were more complicated.  `All demands [were to]
be made to [the Brigade Ordinance Officer] at Bde A Ech[lon] by 1600 hrs daily.
Available stores will be delivered next day'.\autocite[June Admin Order No. 1]
[Para 6a]{27wd}  This thus creates an elastic demand on 90 Coy where any day
could have more or fewer stores thus complicating calculations.  

The first week of Overlord also saw some real shortages.  Almost all vehicles
and, not withstanding Lee Enfiled rifles, weapons were in short supply.  In 
addition mine detectors, `binoculars \ldots compasses \ldots watches', and 
surveying equipment used by the artillery for gun laying were all in short 
supply.  Even
communications equipment was short.\autocite[Appendix A to 27 Armd Bde Adm 
Order No. 1 (June)]{27wd} What's worse was drivers had a habit of running
into communications cables consuming ever more supplies.\autocite[June Adm 
Order No. 2][Para 3]{27wd}  
The situation was so serious that an order was issued stating that if someone 
who issued binoculars, a compasses, or a watch was wounded, that they should 
be relieved of those goods before they were evacuated if at all 
possible.\autocite[June Adm Order No. 3][4a]{27wd}  Of course, implicit in 
this was that saving the life of the wounded man would still come first.

It was not just special, precise equipment that was short however.  Armoured 
units were issued special petrol stoves that could be stowed in tanks, the 
No. 2 (tank) cooker.  This was so that tankers
could heat their meals or boil water in the field. Non-armoured units could 
usually rely on being well enough connected to the supply chain that they 
were to stay connected to the Cooks' lorries or rely on tommy cookers. Before 
the invasion however, a number of non-armoured units were also issued these 
No. 2 cookers presumably as an expedient to ensure the men would not need to
go without hot food or tea.  By 21 June
however, any `vehicle not entitled to carry them' that had access to a mess,
were to return the stove to the Brigade Ordnance Officer so that the stove
could be reallocated.\autocite[June Adm Order No. 3][Para 3a]{27wd}

All these must seem quite minor.  Why should a serious historian concern 
themselves with something as trivial as the availability of binoculars, 
compasses, stoves, rations or water?  The answer is simple:  get these wrong, 
and you loose the war.  The trivial appearance of these stores is by design.  
The mission of the Services is to `[remove] anticipated dificulties and 
facilitat[e] the prosecution of the commander's plan of operations'\autocite
[s. 13(iv)]{FSR1}
When well run, a commander --- the author of many of our sources --- need
not think about logistics. This however does not mean it is unimportant. 
Without these stores, officers cannot see far or navigate and the men will 
starve and dehydrate.\footnote
	{It is not immediately clear to the writer how the British Army in 
	the North-West European theatre would have been able to continue 
	operations without tea in sufficient quantities.}  
Put yourself in the hobnailed ammo boots of a supply officer and
you received that order on redistributing cookers.  %e missing stuffs?
All of these requirements would have to be foreseen and 
prepared well in advance to ensure the required stores were available when
needed.  This is when the army was simply in stasis; however, by the end of
June, the operational tempo for 90 Coy was beginning to once more accelerate.

\section{Operation Mitten 27--28 June 1944}

% Want to show that operations impossible without such units.  Focus on what
% worked, not what didn't work for the British.  British tanks bad, arty good.
% this is a useful moment to talk about the nature of Br Arty.
% Objective (no idea TBH, check 27th Bde WD) TLDR form google search,
% eliminate a DE salient at a chateaux not on my map IVO GR 0372

% 90 Coy's Work

% 30x 3Ton loads of 105mm amn held on wheels for 3 Br div on 23 Jun.  
% 1 3 Tonner likely holds 135 rds 21 TPT Coln p130
% Does it volume out or weight out?  Emphasize that 90 Coy
% was only helping this unit.  Technically, as the 27th Bde unit, they 
% don't do arty stuff.

% Amn is delivered 26 Jun to batteries located IVO GR 0378, it's D-1.

% don't forget about the stuff for flamethrowers!!

Operation Mitten occurred mostly within a single 24 hour period from 27 -- 28
June 1944.  Its aim was to destroy a German salient around 10 km North of Caen. 
This salient was anchored by two Chateaux, Chateau de la Londe, and Chateau de
la Landel.  The assault on the salient was principally attacked by the 8th 
Brigade of the 3rd British Infantry Division.  27 Armd Bde would provide
tank support and they would be supplemented by the Churchill Chrocodiles of
 141 Royal Armoured Corps --- Churchill tanks who's bow machine gun was 
replaced with a flame-thrower.

As ever, 90 Coy's tasking was principally to support the armoured units of 27
Armd Bde; however, their first job for the operation was to deliver some 
30 lorry loads of 105 mm
ammunition to the gun lines several kilometres away from the front lines South
of a commune named Plumetot.\autocite[26 June 1944]{90wd}
%e include hodling amn on wheels on 23?
  Here, several Gun Batteries of the Royal Artillery
were emplaced in preparation for the upcoming battle. 

	\subsection{The Royal Artillery} % amn consumption in barrage, arty
		% usage, etc.  Show just how dependant the Br were with
		% arty.  Maybe get the mass of a 105 shell and propellant?
		% there's no way to get an accurate estimate though, if
		% only they were 25 pdrs...

Artillery is often thought of as a supporting arm; yet, the British Army of the
Second World War tended to operate on the principle that it is better to expend 
firepower rather than manpower.  In light of this, when faced with difficulty,
the British Army showed an inclination to crushing that obstical under the 
weight of artillery.%cite
This could be from fire directed by a Forward Observation 
Officer (FOO) against a specific, observable target (fire for effect), or it 
could be a preplanned suppressive bombardment such as a creeping barrage where 
the guns are laid to bombard a moving line in advance of advancing 
troops.\autocite[130-1, 136]{gunfire}
%e is it worth checking to add targetted suppression?

Artillery was quite an effective and flexible tool.  Take the example of 
Lt Boyle of 17 Field Regiment RA who acted as a FOO for 38 Irish Bde in 
Sicily.  He and an infantry company commander once saw a large number of German 
troops massing, likely in preparation for a counter attack.  Thus, Lt Boyle 
got on the wireless, adjusted fire onto the German unit, and order `10 
rounds gunfire' from an artillery regiment of 24 guns.  Soon, `240 shells
landed within an area less than a football ground'.\autocite[133]{gunfire}
The company commander was impressed and asked for another salvo.  The FOO
simply said the proword `REPEAT' and another 240 shells once more
saturated the target area.\autocite[133]{gunfire}  It is this ability to 
rapidly concentrate firepower on any point within range of the batteries
by simply making a call on the radio that lies behind the power of the
artillery.

Doctrinally, it could be used to kill an opponent, neutralize them 
(force them to keep their heads down for long enough for 
friendly infantry to kill or capture that opponent), demoralize them, or 
`partially destroy' them (kill or wound 2\% of entrenched forces or 20\% of 
troops in the open).\autocite[133]{gunfire}  Unlike the First World War, by 
the Second World War, it was relatively uncommon for the British to fire
multi-day preparatory bombardments to entirely destroy enemy positions.  
These bombardments were too wasteful of ammunition, destroyed the ground, and
were not terribly effective at destroying an entrenched 
enemy.\autocite[132]{gunfire}
Second World War bombardments tended to focus on providing the enemy with a
`short, sharp shock', keeping enemy heads down whilst friendly forces advanced.
Of course, if troops were in the open, FOOs would only be too willing to kill 
them but the usual aim was to suppress them.\autocite[133]{gunfire}

Foregoing the Frist World War's habit of massive bombardments did save some
ammunition, but even with these measures that saved ammunition, the British
Army's ammunition requirements were still vast. 
Consider these planning baselines for what a given number of rounds expended
would do to a recipient.  It was assumed that for a unit with
25-pounders\footnote{Common British field gun} to partially destroy enemy unit
equipment in a 100x100 yd square, the unit would need to expend 40 rounds of
ammunition.  Demoralization would require 40 rds/hr over 4 hours (160 rds 
total) or 100 rds/minute for 15 minutes (1500 rds).  Neutralizing the enemy 
would require an 8 -- 32 rds/minute bombardment continued for as long as the 
enemy was to remain neutralized.\autocite[133]{gunfire}  
Whilst batteries would have around 32 rds/gun in the gun's 
limbers, this small quantity of ammunition was 
insufficient for all but a brief fire mission.\autocite[51]{gunfire} 
Major operations could easily consume over 600 rds/gun. Thus, the RASC
would make regular deliveries to the gunlines to feed the Royal Artillery's
insatiable appitite for ammunition.\autocite[122-3, 125]{gunfire}
The stocks required to maintain this instant access to firepower could be 
enormous.  In the lead up to Operation Goodwood, which we will discuss later,
each gun was issued with 750 rds of ammunition just to ensure that the RA would
be able to meet demand.%cite

These measures were nessessary to permit Montgomery's preference for the 
set-piece battle --- an operational method, though currently unpopular 
compared to maneuver warfare, chosen principly to reduce casualties and
maintian the tenuous moral of British troop by minimizing the risk of 
failure.\autocite[100-1]{cracks}  Part of this was heavily reliant on the
British ability to concentrate a significant amount of firepower along a
single target area.  Logisticians were critical in both concentrating that
firepower through the supply of ammunition, as well as for physically 
concentrating the troops required into the battle area through the provision
of motor transport.
%99-100 are more on the risks involved in a counter-attack though sooo...
%e maybe reword this section to talk about how less logistics puts pressure on
% commanders encouraging risk taking; risks that the British Army really were 
% in no position to take. (pdfp 63, rereading it is advisable)

%cite maybe CC 63 worth looking at 62, 
% 77 for German ability to exploit allied audacity to inflict local defeats
% 44's a maybe?  stunning victory but loosing the peace
% 58 Prolly worth looking at.  Compares the American approach to the Brits

Of course, it is sometimes nessessary to take risks in war and commit to a
bolder, less cautious approach to operations.  This means pushing the combat
arms as far forward as possible, making use of temporary weaknesses amongst
enemy forces.  Such an operation may be seen by US General Patton's thrust
through vast swathes of France during Operation Cobra.\autocite[100]{cracks}
During Cobra, he largely left his logistiicans to figure out how to supply 
him whilst he drove forward.  %cite supplying war
With this drive was the real risk of the units involved being
counterattacked by the Germans with the possible destruction of the units
involved.  Clearly, this counterattack did not materialise at Patton's risk
proved successful; however, unlike the British, the Americans were in a
position to take such a risk.\autocite[100-1]{cracks}  

Unlike the US or even the Canadians, by 1944, the British Army was literally 
running out of men.  
The British were unable to replace casualties by drafting more troops from 
home.  The manpower no longer existed.  The British could not be wasteful 
with men for each dead or wounded man would mean the army in North-West 
Europe would shrink.\autocite[5]{cracks}  
Just to make up the numbers for Overlord required the British to draw down
staff from training establishments across the country.\autocite[52]{cracks}  
It was acknowledged that reinforcements were unlikely to become 
available.\autocite[50-1]{cracks}
Already the British Army was constraining its operations and being less 
daring to conceive manpower.\autocite[69-70]{cracks}  %cite
This was with ready access to ammunition and
firepower.  Without regular supplies of ammunition to the guns providing this
instant access to overwhelming firepower, it is difficult to imagine how the 
British Army would be able to maintain sustainable casualty rates to continue
operations in Europe.  Once again, whist supply does not --- or at least,
should not --- fight per se, effective fighting is impossible without them.
The logisticians of the British Army permitted the Army to concentrate 
overwhelming firepower to defeat their enemy.



	\subsection{Support to Operations}
		% Operation used flamethrower tanks from 141 RAC (Churchill
		% Crocodiles)  Needed special fuel and nitrogen cylinders.
		
		% Coy sets up dump of 3000 Gal (13640L) of fuel and 90 N cyl
		% at Gazelle (GR 0276) on 27th, withdrawn on 29th.

		% D+1, Maint Point set up at Le Vey (GR976757) distributing
		% 2400 Gal of FTF.  

		% D+1 for Staffs Yeo, 0300, 1082 rds 75mm HE delivered, 
		% 600 rds at 1600 hrs, 1200 rds at 2000 hrs delivered --- what
		% where the Staffs doing?  Find out  Also for context, give
		% amn capacity of tank  96 rds per tank 90wd July

		% in Op Inst No 2 for 27th Bde (28 Jun 44) sqn shoots limited
		% to 50 rds/gun.  Tanks were to harass Germans as long as 
		% Br infantry weren't harmed by this fire and the tanks move
		% as soon as counter battery fire is encountered
		%cite 27th Armd Bde WD, Jun 44 p12
		% This tells us there's no great fear of running out of 
		% ammo.  If you could fire off half your ammo in sqn shoots
		% and still be operationally ready for further operations or
		% movement, you're confident you could be replenished.
		
Whilst Operation Mitten started with artillery preparations, 90 Coy was the
RASC unit for 27 Armd Bde.  In light of this, the majority of their 
work was in support of the armoured component of the operation.  The 3rd
British Infantry Division was, as usual, supported by 27 Armd Bde, but for
Mitten, the Div was also supported by Churchill Crocodile flamethrower tanks
from 141 Royal Armoured Corps.  These tanks were unusual because, in addition
to the usual fuel, spare parts, and ammunition, the Crocodiles also required
fuel for the flamethrower as well as compressed nitrogen cylinders for 
propelling that fuel towards the enemy.  In light of this, on 27 June, the 
first day of the operation, 90 Coy established
a dump 1--2 km from the combat area at Gazelle consisting of 3000 Gal (13640 L)
of flamethrower fuel, and 90 nitrogen cylinders to keep the Crocodiles in
service.\autocite[27 June 1944]{90wd}

Alas, the assault that evening by 1/South Lancashire Regiment
principally supported by the Staffordshire Yeomanry (Staffs Yeo) %hole
was unsuccessful.  Thus, over night plans and preparations were made to try 
again with the whole of the 8th Infantry Brigade the next morning.\autocite
[111-2]{assault-div}
%e do a seach for Cuthbertson.  Also, lok at pdfp 69 of Assault Division
Thus, 90 Coy spent much of the 
night replenishing the Bde.  Indeed, by 0300 hrs on the 28th, 90 Coy had 
delivered 1082 rds of 75 mm High Explosive (HE) ammunition to the squadrons
of the Staffs Yeo providing a picture as to the ammunition expenditure in the 
evening prior.\autocite[28 June 1944]{90wd}
The next day, 27 Armd Bde was ordered to support the renewed
assault by patrolling the area and providing harassing fire as needed.  
Squadron shoots were limited to 50 rds/gun.\autocite[Operation Instruction 
No 2 (see June appendix)]{27wd}  These 50 rounds represented 
approximately half of the capacity of each tank.  
%edouble check
The fact that ammunition could be so freely expended is testimate to the 
effectiveness of logistics support for there was no fear that they would
run out of ammunition.  Indeed, if there were any such fears, the were 
needless.  Operations resumed around 0500 on the 28th, By 1600 hrs, 90 Coy 
had delivered an additional 600 rds of ammunition to the Squadrons. An hour 
after the successful end of Operation Mitten around 1700 hrs, 90 Coy began
to fully replenish the Bde and by 2000 hrs, they delivered an additional 
1200 rds of 75 mm ammunition, and likely fuel and rations as well, to 
the Staffs.\autocite[28 June 1944]{90wd}

%e compare to the Germans?

In addition, as 141 RAC was still being supported by them, 90 Coy set up a 
temporary maintenance point in Le Vey, approximately 5 km West of the Gazelle 
maintenance point, to replenish 141 RAC.  Over the course of three hours, they
deliver 2400 gal (10910 L) of flamethrower fuel to 141 RAC.  The next
day, the flamethrower stores at the Gazelle dump are withdrawn and returned to 
depot.\autocite[28 -- 29 June 1944]{90wd} %hole 101 recieted it, did they?

All this work was done for a simple two day operation and it is indeed right
and proper that we remember the enormous loss of life suffered by the combat
--- indeed, over the course of June, 3 Div suffered disproportionate 
casualties and we owe it to the fallen not to forget them --- but from an
operational perspective, to simply focus on the dashing infantry is 
imbalanced.\autocite[112]{assault-div} Over the course of 
around 36 hours, Staffs Yeo had consumed 2882 rounds of ammunition to kill 
six tanks.\autocites[June appendix, Operation Mitten Intelligence Diary, 
Entry 58]{27wd}[on ammunition expenditure, 27-8 June 1944]{90wd}  Of course, 
a significant amount of that ammunition would have been HE ammunition fired
against infantry and other soft targets;
 nevertheless, this expenditure is still quite large.  Moreover,
without fuel or rations, the attack would have round to a halt quite quickly.
Without the support of 90 Coy, the armoured component of this assault would
have been impossible.  Infantry casualties for Operation Mitten were already
high enough to disuade the British Army from further operations in this 
area.\autocite[111-2]{assault-div} These casualties would likely have been 
higher if not for the armoured support.  To stop
our analysis of war with just fighting ignores that which makes wars 
possible in the first place.
Moreover, note the significant amount of activity after the closure of 
operations.  Effective logistics work is not simply to support the current 
fighting but to ensure the Army is ready for the next action before that 
action has even been fully thought out.%fsr quote

%e reconnect to thesis

%e add note on how I'm playing fast and loose with the QMG and the MGO

		% conclude this section maybe this doesn't actually need a 
		% subsection.  This much work was needed for a simple 2 day
		% operation.  Work vital but it was doable.  Sum it back up.
		% Maybe the tanks weren't the best, but their inadequacies were
		% floated by a supply chain that kept up so that they could be
		% bad but still effective.

% \section{Operation Aberlour}
% Never took place, do I wanna talk about it?  Issued 27 Jun 44.  A lot of 
% admin stuff for 90 coy like amn dumps, etc.  (Cite p26 27th WD for Jun)
% more on p30.  I almost like the idea of talking about an op that never
% takes place.  It eliminates the fog that arises from what happens when
% there's contact made with emy.  Here we can focus on how operations
% take place.  Hmm...

% \section{} % Run up to Op SHERWOOD

	% extensive minefields to be laid (do sq footage) 
	% (See pg 13 of 27th Bde WD)

\section{The Lead up to Charnwood}

In the week following Mitten, the situation for 27 Armd Bde and 90 Coy as a 
whole consisted of operations planned and operations cancelled, minefields laid
and withdrawn, etc.%cite Gussing 27wd?
It is likely that 90 Coy was involved in providing the 
mines to the Regiments but these would probably have been carried in routine 
supply runs so precisely how the mines found their way to the Regiments is not
recorded in the sources.  Generally, the situation was quiet.  On 3 July, the 
balance of the Company finally began to trickle in.  
From 3 -- 5 July 55 vehicles and 162 personnel, the Company workshops, and 
other assets trickle into Coy HQ in Cresserons.
This buildup was inline with a general massing of the Army in preparation for
the upcoming Operation Charnwood.  In light of this, 90 Coy's headquarters was
moved a few hundred meters to make room for a Medium Artillery Battery.  As
usual, despite the general crowding in the area, the Coy was sure to disperse
themselves over several hundred square meters across several open fields to 
minimize the potential losses from sporadic shelling or air 
attacks.\autocite[1 -- 5 July 1944]{90wd}  

Following 5 July, activity for 90 Coy began to pick up in earnest.  
Unlike Operation Mitten, which aimed to capture two chateaux,  Operation 
Charnwood was a much larger operation aimed at capturing Caen.  27 Armd Bde's
task was to push support the 3rd and 59th Infantry Divisions in their push 
South to the Caen -- Bayeux railway and River Orne therein capturing the bulk 
of Caen North of the river.  Having secured Caen, Charnwood was then to secure 
bridgeheads across the Orne to act as a springboard for future 
operations.\autocite
[Operation Charnwood, 27th Armoured Brigade Operation Order No 2 (See appendix
to July diary following papers pertaining to Goodwood)]{27wd}

Whereas Mitten was fought with a single infantry division and an 
armoured brigade; Charnwood would be fought with three infantry divisions, 
three armoured brigades, and would begin with a large scale bombing campaign.
Charnwood's altogether grander scale meant far deeper operational entanglements.  
Charnwood required more POL points, ammo points, provisions for rations, etc.
Frequently overlooked are the more extensive traffic control 
requirements. %cite
We're used to thinking of armies as symbols on a map that can
be moved at will as this is how they appear in books.  Reality however is 
far more complicated.  An Army is comprised of divisions that are capable of 
nominally fighting semi-independently.  These divisions range in size from
around 10 000 -- over 20 000 men.\footnote{For context, the Airbus A380, the
worlds largest passenger aircraft, is certified to seat not more than 853
passengers.%cite \cite{a380}
}  To move this volume of men across poor rural roads 
is no mean feat.

Challenges range from the obvious issue of the traffic jams that can arise from
funnelling large volumes of men across narrow roads to more obscure issues like 
road wear.  Wet, unpaved roads subjected to the simple tramp of boots will
slowly be chewed into a muddy mess.  Wheels will likewise wear deep ruts even 
in dry roads.  Tracks of course are the worst.  Their ability to grip nearly
any surface also means that a brigade of armour moving through an area can
rapidly destroy even a metalled road.  In light of this, tracked were 
typically routed along dirt tracks whilst wheeled vehicles used roads.  Much
to the chagrin of staff officers and, presumably also the Engineers who had to
maintain these roads, this practice was not always adhered to
leading to the deterioration of the road networks.\autocite[27 Armd Bde Adm 
Order No 8 (See July Appendix)][Para 3]{27wd}
These movements would be co-ordinated using detailed movement tables that 
divided the units to be moved into a number of \textit{serials}. 
These tables would provide start points, end points, routing 
information, routing, and timings to minimize congestion and spread road wear.
Ideally, only a single serial would be on any single road segment at a time.
Too many units on a single road could easily result in units becoming 
intermixed and getting lost.  Beyond the simple headache of being stuck in 
traffic, units getting lost means they are not where they need to be and
relocating units, disentangling them, and sending those units back to where
they need to be could take hours.  Hours where the affected units are not
as effective as they were supposed to be.  
%e talk about why 90 Coy not included in movement table, talk about MPs
%e really talk about the importance of transport

As it happens, just prior to Charnwood, 27 Armd Bde and attached units
had to move from the Bde's main quarters in the vicinity of Cresserons 
to a point closer to their pre-attack assembly area some 4 km south at 
Gazelle.  Some units to be 
attached to 27 Armd Bde were further afield and they would also have to
make their way to Gazelle to permit the various units to gather prior to 
D-Day.  This road march occurred on the night of 6/7 July (D-2/D-1) from
roughly 2300 hrs on the 6th to 0100 hrs on the 7th --- a few minor exceptions
occurred with minor units.\autocite[Appendix C, 27 Armd Bde Operation Order 
No 2, 6 July 1944]{27wd}  This road march would have occurred in
blackout driving conditions and, as far as the sources describe, was likely
successful with units arriving at Gazelle more-or-less on-time despite the
heavy showers that were occurring at the time.\autocite[6 June 1944]{1raf}
Moreover, the march covered roads that, according to contemporary British
Army maps, were 3 -- 6 m wide or were secondary roads.  %cite
Even today, Google Streetview images shows that these roads remain 1-2 lane 
roads just wide enough for two tanks to pass each other on that road. %cite
Effecitve transport planning, a core part of logistics, was critical just
to allow the Army to move.

Preparations continued after the Brigade's arrival.  Once again, Churchill
Crocodiles would be attached to the brigade so a 3-ton lorry loaded with 
flamethrower fuel was attached to the first echelons of each of 27 Armd Bde's
three regiments.\autocite[7 July 1944]{90wd}  Around 2150 hrs D-1 (7 July), 
some 300 Lancasters bombed the Northern Approaches to Caen in preparation 
for the coming assault. Witnesses watching from the Brigade area note that 
the bombing was `most spectacular'.\autocite[7 July 1944]{27wd} 
%e draw this out maybe

% Use this section to warm up the reader as to how logistics works
% in operations.

% Context for Charnwood

% D = 8 Jul

% Will continue using the Gazelle ammo dump until exhausted (smart, why move 
%it)  This is the only initially authorized dump. 

%cite 90 WD 8 July

% AP set up NW of Cresserons at Coy HQ-ish area %cite 90 WD 8 July
% Rgmts draw ammo next morning, ERY withdrawn 1400 when Caen's captured

% 90 Coy is carrying 1000 Gal FTF, 35 N bottles on wheels for crocs at
% Cresserons

% 3 days compo rats issued, additional 3 days AFV packs carried in tks if
% compo not avail.

% Blankets to be distr as situation permits

D-Day arrived on 8 July and,
after a cloudy evening on ground that may have still been quite damp, 
units began forming up at 0200. 
H-Hour arrived at 0420 hrs with a `tremendous [creeping] barrage' paving the 
way.\autocite[8 July 1944]{27wd}  Progress for the Brigade was rapid and
despite encountering some minor resistance, by the end of the day, the 
Brigade had reached the outskirts of Caen.\autocite[8 July 1944]{27wd}

D-Day Charnwood for 90 Coy was busy.  On top of supporting the troops of
27 Armd Bde, 90 Coy was also responsible for supporting all other armoured
units supporting 3rd and 59th British Infantry Divisions.\autocite
[\CharnAdm][Para. 3]{27wd}  These units mostly consisted of the Chochodiles
of 141 RAC, as well as Royal Engineer (RE) vehicles.  These consisted of 
Armoured Vehicles Royal Engineer (AVRE) and Flail tanks.  

AVREs were 
Churchill tanks specially modified to assist with engineer tasks.  They
could carry and deploy fascines (bundles of tree branches for filling
ditches), fire a special spigot mortar affectionately referred to as a 
flying dustbin --- useful for demolitions.  AVREs also gave RE sappers 
cover if they needed to leave their vehicles, often in rather exposed 
positions, to demolish or build something.  
Flails were modified Shermans equipped with a demining flail.  These
vehicles would clear lanes through minefields permitting penetration by
infantry or armoured units.  Needless to say, the addition of these
units added to the logistical burden of work by 90 Coy. 
%e add note on mine tape?

Nevertheless, the prime role for 90 Coy was to keep the Brigade supplied
with ammunition.  Conveniently, there were still stocks of ammunition left
over in Gazelle from Operation Mitten that were never withdrawn.  Thus, for
Operation Charnwood, 90 Coy would maintain the Gazelle dump until stores 
were exhausted before opening a new ammunition point at Cresserons later 
that day, presumably as the Gazelle dump was exhausted.\autocites[\CharnAdm]
{27wd}[8 July 1944]{90wd}
This was advantageous as it meant that the closure of the Gazelle dump would 
not require transport or labour to move them to a new dump.  The ammunition 
in Gazelle would simply be expended, and the dump closed.  
%e talk about transport circuits ?

In addition, they also kept `1000 Gals [of flamethrower fuel]
and 35 nitrogen bottles on wheels \ldots at Cresserons' and were to be
available to dump further supplies on call.\autocite[\CharnAdm][Para 6]
{27wd} In addition, organic transport for the Regiments comprising 27 Armd
Bde also had their own ammunition lorries ready to ferry ammunition whenever
they were needed for final delivery.  
These requests would have been made using the 
wireless and the request would be routied to the most appropriate unit to 
speed the ammunition to critical 
sectors.\autocite[13th/18th Royal Hussars Operation Order No. 1, Operation
Charnwood (See July appendix)][Para 9b]{1318wd}

Conveniently, prior to the attack, three days compo rations were issued to
all armoured troops in addition to the usual three days reserve of AFV
packs that were carried in tanks.\autocite[\CharnAdm][Para 7]{27wd}  
This helped to simplify the burden on the supply chain.  This prudent decision
meant that the supply chain could focus on ammunition, fuel, and water.  
What is perhaps more interesting is that Charnwood was planned as a two day
operation but six day's ration was issued.  This prudence allowed for more
flexibility.  It meant that if it took longer than expected to consolodate
Charnwood's gains and expand the supply chain, the supply chain could operate
on a temporary basis, operate on a reduced capacity giving time for the
supply chain to catch up.  Moreover, it meant that, if the British ordered an
admittedly uncharacteristic pursuit, logisticians would have eased just one 
more constraint tying the hands of decision makers.  Of course, if further 
rations were required, they would probably have been delivered but making 
up for the occasional combat loss of rations is a miniscule logistical load 
compaired to delivering rations for a whole Bde.

Curiously, the 27 Armd Bde's Administrative Orders for Charnwood does not
mention fuel but we can be absolutely certain fuel was consumed.  The absense
of fuel in the orders suggests that the Bde felt that the supply of fuel was
sufficiently normal that it was unnessessary to write down how fuel would
be obtained.  Thus, it is probable that fuel would just be delivered through 
the \textit{usual means}.  First line units would have gone to battle with full 
fuel tanks and would carry a small 
reserve of additional fuel. The lack of formal orders on how to obtain more 
fuel suggests that the this fuel would have lasted a day's fighting.   
90 Coy had a 
Signals unit attached at Coy HQ and it would have likely been routine to
load up a lorry with fuel and make an urgent delivery.  

That evening, from 2100 -- 2359 hrs,
90 Coy set up a POL point just south of Hermanville.\autocite[8 July 1944]
{90wd}
Whilst the sources do not state this plainly, the location chosen for this
POL point appears to allow for one way traffic through the point thus allowing
there to be a constant flow of vehicles travelling through the point.  It is 
likely that as first line units were beginning to return to their harbours that 
evening, first line transport would have gone to the POL point 4 -- 6 km away 
to collect enough fuel for their units.  These transport units would then have 
had to make the trip to the nearest dump either at Gazelle or Cresserons.  All
this would have likely occurred prior to the return of the units to their 
harbours.  The absence of these details from the orders suggests that the work
done by logistical planners was so effective that the Officers of G Branch
could largely ignore it.  Instead, Officers could focus on important matters
such as defeating the Germans, taking care of the men, or getting some much
needed rest.

%e ^ this is awful!

Once the units were in harbour, it is easy to picture the scene that likely
unfolded with tired men lifting endless jerrycans of fuel onto the decks of 
their tanks, emtpying the cans, and tossing down the empties.  
Surely, these same men would have also stood forming human chains to pass
ammunition into turrets and stowed inside tanks.  One also wonders a more human
component.  Would rations be heated on stoves at this point, or would tins 
having been left in the warm engine bay earlier now be at a decent 
temperature?  One wonders if tea was being brewed and if the men opted to sleep
in slit trenches for protection, or on their tanks for comfort.  Surely, for
some, it would have been a sleepless night as briefings, planning, and other
preparations took place through the night in  preparation for the resumption 
of operations the next morning.  All of this would only have been possible 
because someone saw fit to deliver enough fuel and ammunition to the units. 
Moreover, soemone would have to collect the empties and exchange them for filled
cans.  All this would have been done to enable smooth operations.

Operations resumed around 0500 hrs with the Bde deploying patrols forward to 
the positions they left some six or seven hours previous.\autocite[9 July 1944]
{1318wd}  90 Coy spent the morning issuing ammunition at their ammunition 
points and pushing flamethrower fuel to 141 RAC.  
From 1400-1600, as the various units were reaching their final objectives, 
units of 27 Armd Bde returned to their harbours for refit.  Doubtless, once 
the battle ended, the logisticians of 27 Armd Bde would have to ammunition, 
fuel, and generally replenish the Bde.\autocite[9 July 1944]{90wd}  Again,
whilst this work seems trivial compared to the work of actually fighting
the enemy, the work completed by units such as 90 Coy were critical for the
conduct of the war.  Once more, the work of logisticians \textit{enabled} the 
fight but unlike previous smaller operations like Mitten, or more improvised
operations such as those that occurred that occurred shortly after Neptune's 
D-Day, in Charnwood, 90 Coy RASC operated a planned, deliberate role during
the operation as part of a much wider machine.  Unlike around the Neptune
landings, where 90 Coy was dispatching transport units in every direction,
here their task was clear:  keep the Brigade supplied with ammunition and fuel.
Without them, the operation could not have continued onto the second day for
there would be no fuel to power the tanks, and no shells for the guns to 
shoot.



% tracked traffic to use tracks rather than roads --- degradation
%cite 27th Jul orders 34-6  Sandbags used to store casualty's kit
%%%%%%%%%%%%%%%%%%%%%%%%%%%%%%%%%%%%%%%%%%%%%%%%%%%%%%%%%%%%%%%%%%%%%%%%%%%%%%%
% Stop Work Point 15 1736 Dec 25
%
% Done post Charnwood.  pre-goodwood next
%
% Should add section that daily ration and fuel deliveries were routine.  See
% Aug diaries of 90 Coy
%
% Adding stuff on historiography.  Need to re-read parts of Neptune and Gators
% for this section.  Prior to this, was working principly on pre-Charnwood 
% activities.
%
% Adding stuff from the June Admin Os
%
% Add jerry cans common by D day (IWM film A70 70-1 8:30)  Use of human 
% chains also common (9:40) (10:55) jerry cans stacked to head height
%
% ADM 1995 10:05 POL shore tanks
%
%
% Current point 14-18 June.  
%
% Connect the D-Day supply difficulties to mention the historiography. Talk 
% how they talk about fighting but are blind to how razor thin the armoured
% war was.  They were very close to not being able to be fuelled.  Ref Armoured
% Campaing in Normandy index enteries for 27th armd Bde (at end of Br units)
%
% Ranville, wrote the transition, nwo talk about the combined operations 
% between 6 Airborne, 27 Armd Bde, and 90 Coy, but first, I need a break!
%
% dont' forget to metnion how supply evolves but plans work as a framework.  
% take Hermanville how the Brigade ammo dump was conviniently on land of BMA 
% Moon's amn dump or Ranville in the pre-conclusion
%
% Crosscheck citations for DHH.
%
% I should probably define 1st line, 2nd line, and 3rd line units sometime.
%
% Preparing to talk about what happens to the company after the 27th is 
% disbanded.  Starting work on August diary.  Talk till end of Aug to show
% other side of log.
%
% Flesh out \section{Criticality of Supply} Do I want to make the 
% historiographical section a subsection or it's own section?
%
% Do I want to use more subsections?  I like how they organize.
%
% Working note, fleshing out the analysis more
%
% Should flesh out the explanation of sup as sys in earlier section
%
% Wondering what to do about the sectioning where I talk about the structure
% of supply. 
%
% Play to the theme that there is so much more to war than fighting
%%%%%%%%%%%%%%%%%%%%%%%%%%%%%%%%%%%%%%%%%%%%%%%%%%%%%%%%%%%%%%%%%%%%%%%%%%%%%%%

\section{Post Charnwood/Pre Goodwood}
% 27th HQ moves to Douvres 10 Jul

There was a slightly quiet lull post Charnwood.  On the 10 July, Bde
HQ moves a few kilometres to Douvres and the whole Bde is placed initially on 
24 hours notice, later extended to 48 hours notice, which allowed the men to 
rest, take care of their vehicles, and recuperate.\autocite[10, 13 July 1944]
{27wd}  
This recouperation was likely much helped by the first issuance of fresh
bread to the Brigade on 11 July.
Whilst this happened, the British Second Army was reorganizing to make up 
losses as the British were beginning running out of men to make up the Army. 
For now, these reorganizations would have only a limited impact on 27 Armd 
Bde and 90 Coy.  
% talk about MPs  Maybe use the road march as a segue

A lull followed Charnoowd during which the buildup of men and materiale in 
France continued.  On 10 July, Bde HQ moved to Douvres and the rediness of 
the Bde reduced first from 24 hours notice to 4 hours notice.  This allowed
the men some time to rest, maintain their vehicles, and to recouperate.  Part
of this recuperation involved the first delivery of fresh bread with the 
arrival of 35 Mobile Field Bakery --- also a unit of the RASC --- in theatre.
Their arrival meant that the men would recieve, at first, 2 oz, then soon 
after, 4 oz (approximately 60 g, then 110 g) of bread per day.  This would
have been the first amount of officially issued food the men recieved since 
stepping off their landing craft a month earlier --- it is however likely that 
the men would have recieved fresh food from unofficial channels.

%e talk about shortages as insert 1

Whilst this happened, the Britsh Army was reorganizing.  The losses sustained 
by the British Army to date had been vast and the army was running out of men.  
This reorganization helped ensure the British could still field 
combat-strength units  to take place in the upcoming Operation Goodwood.  
Goodwood itself was a larger operation than Charnwood and produced more 
paperwork at the Bde level.  Thus, Goodwood allows us to better examine themes
such as battle sustainment and management, rations, and signals.  Unlike 
%e add map on Goodwood/Charnwood
Charnwood which was fought from the land West of the Orne and North of Caen,
Goodwood's thrust would come from East of the Orne from land first captured
by the paras on D-Day.  Whereas Charnwood could be launched from land the Bde
had occupied since D-Day, the Bde would have to move West of the Orne in order
to sit ready on the start line when H-Hour arrived.

The Warning Order for Goodwood came on 14 July for an operation four days 
hence.  %e check this
For 90 Coy, this meant preparing new supply lines, setting up new supply 
dumps, and reparing new ammunition and POL points. 
%e perhaps personify in body of the CO?
90 Coy likely recieved the orders to prepare for upcoming operations some time
in the late morning or afternoon of 14 July.  The work ahead of the company 
was vast, the whole Brigade's combat strenght would have to be supplied over
the river.  Some logistics, HQ, and support personnel would however remain
West of the Orne to simplify the siting of units over the bridgehead.

%e add distance from Coy HQ and BDE dump area to Ranville dump
Helpfully for 90 Coy, the 13/18th Hussars had already been operating in the 
area in the weeks preceeding and they already had a smaller regimental dump in
the Ranville area.  For a Bde however, the dump would be far too small.  As
such, the Company, started up their lorries and, over the course of 5 hours
from 1700 to 2200, dumped 38 lorry loads of ammunition and 18 loads of petrol
and diesel in the location of the new Bde dump.  This was enough to supply 
each tank in the Bde with 97 rounds --- a full load of ammunition for the
Bde's tanks --- and enough fuel to drive the Bde 30 miles (48 km). 
Likely tired after this work around 2200, the Company refueled their viehicles
and returned to the Company HQ by midnight.

Logisticas however does not just happen.  It needs to be managed.  Someone 
must keep track of where everything goes, how they get there, manage 
competing priorities of supplies and transport, and arrage for the resupply
when needed.  It is a responsible job that requires competence, forethought, 
and leadership.  Managing a Bde supply dump is thus not something that should 
be left to be organic or unsupervised.  It is likely that the client units 
would otherwise strip the dump of all its resources and no-one would be able
to allocate limited resources, or arrange for resupply.  

Thus, at 1900 on the 15th, the day after the dumping operation, Capt Duffus
of %e subunit?
was detached from the main body of the company with `46 men, 1 car, [and] 16'
3-ton lorries.  Of his 16 lorries, 14 were to be used to keep a mobile reserve
leaving two lorries for routine tasks.  Of the 16 lorries, 6 would be used to 
hold 4 days compo rations for the Bde, and 8 would be used to transport fuel
and nitrogen cylinder for 141 RAC's Churchill Chrocodiles.

Unlike for Mitten and Charnwood, the Ranville reserves had a greater 
importance due to its geography.  Whereas these two operations had several 
viable alternative supply lines if one got gut, Goodwood was different.  
Cutting Rugger and Cricket briges would effectively strand British forces
east of the ruiver until a new bridge could be errected.  This is why having a
formal detachment was importatn.  If forces were cut off, good and reliable 
officers were needed to take charge and manage the Bde's supply chain.  At any
moment, that officer must know what is on hand, make decisions on
where materiale will go, and be able to handle the various tasks that 
inevitably emerged.  If the Bde's supply lines were cut, such an officer had
to be able to act independant of their headquarters, without the guidence of 
their superiors, to take the right decisions nessessary for the Bde to meet
their objectives.

Dealing with these shortages was not merely a theoretical problem.  Orders 
from 14 July, the Friday before Goodwood, noted a number of logistical 
difficulites.  `There [were] few replacement tyres available' due to a 
shortage of rubber. %cite
This was exasperated by the men lowering the tyre 
pressure for a softer, more comfortable ride; however, this resulted in 
significantly more tyre wear.  Hospitals too were unable to keep up with the
demand for cutlery, mess tins, and washcloths.  As such, it was recommended
in the strongest possible terms that casualties ought to be evacuated with
such stores if at all practicable. 

There was also the problem of returned ammunition.  Empty casings as well 
as unfired ammunition are of military value but they often were not being
returned properly.  Unfired ammunition can be repackaged and reissued whilst
the casings of fired ammunition can be returned and reloaded; however, if 
these stores are not packaged properly, it can lead to difficulties counting
or protecting them.  Moreover, there was a problem where unfired ammunition
was being returned as if it was empty.  This may appear to be a minor 
inconvienience but fired ammunition casings are really just brass or copper
tubes and are thus, basically inert.  Unfired ammunition will still have their
primers and charges present.  Whilst unlikely, there are a number of 
unfortunate situations that could lead to either of these detonating.  This
lead to a few injuries for British Forces.

Finally, there were problems with units over drawing rations exasperating a 
general shortage of rations.  The problem was so severe that snap inspections
were ordered to ensure that there were enough rations to go around.  Whilst 
the officer or NCO carrying out these inspections were likely unpopular with
the troops, their work was nevertheless important to ensure no troops went 
hungry just because a different unit wanted a more comfortable reserve of 
rations.

Likewise, many of these other problems were issues that could be remedied
through good logistics officers.  Officers could check demands against 
entitlements and raise questions if a unit was asking for more supplies than
they apparently needed.  Officers and NCOs were often also in a position to
impress upon others, the importance of carefully inspecting ammunition before 
it was reissued or returned to depot. %e confirm in FSR
Logisticians were also in a position to recieve excess equipment from 
casualties and, through second line transport, send excess equipment to other
units who required such equipment.  Obviously, there was little to be done 
aobut supplies like tyres which had a national shortage, but logisticians
could at least ensure that unused tyres were stored properly to prevent 
pre-installation failures.  All of these mundane tasks were nessessary to
ensure the British Army was ready to fight the enemy whenever it occured.

%e awkward transition
As it happens 16 July, D -- 2 for Goodwood, prepariations were nearing 
completion.  Conferences were still being held across the Bde as COs flesh out
orders.  Whilst Bde HQ stays West of the Orne, the regiments move over the 
river thus nessessitiating changes to the supply chain.  Fortunately, 90 Coy's
establishment of Detachment Ranville would bear fruit as the Det began to 
supply the Bde. 

It is at this point that rations for Goodwood were issued.  90 Coy issued
to the Bde's units West of the Orne 3 days additional ration whilst units 
East of the Orne --- mostly combat units and the units directly supporting 
them --- were issued with 4 additional days rations.  
This is in excess to the 4 days ration being held
on wheels by Capt Duffus at Det Ranville.  Thus, the disposition of reserves
in possessed by the Bde for this two-day operation was 8 days rations, and 
enough fuel for the whole Bde to travel 30 miles.  Moreover, the Bde's tanks
could fire every last round of ammunition in their panniers and the Bde would
still have enough ammunition left over for a full replenishment.  All these 
stores were available without having to contact units from outside the Bde 
requesting assistance.  The Bde was capable of supplying the materiel, 
transport, and personnel nessessary for any replenishment --- though they
might need to contact transport for permission to use the roads.  

The final prepairation for logistics in the Bde came on 17 July (D -- 1) when
two wireless lorries with their accompanying signelmen were attached to the 
Coy. One lorry would remain with Coy HQ at %e location
and teh other was posted at Det Ranville.  If the Bde needed anything, it was
a simple call away.  Thus, as the Bde loaded the last of their ammunition, 
rations, and fuel into its tanks; wrote their last letters, and issued some
final orders for a day that promised some hard fighting, the Bde's 
logisticians were ready to ensure that it would not be for want of supplies 
that the advance would halt.  Past H-Hour, if the combat arms could push the
Germans back, the supply chain was in a position to keep up with the advance.



% CONTEXT
% Post Charnwood, Bde is withdrawn on 24 hrs notice on 10th, increased to
% 48 hrs notice on 13th.  Men get some rest but in the background a number
% of conferences occur in preparation for Goodwood.  Vehicles are likely
% being maintained and losses replenished in this time.  2nd Army's being
% reorganized and a CO's conference on battle lessons occurs on the 13th.
%cite 27th WD


% talk about how this is often seen as an interlude but life continues.
% first bread ration issued on 11 Jul from 35 Fd Bakery at Luc sur mer.
% 2 oz issued / man increased to 4 oz next day --- men would have eaten 
% largely tinned food for a month.  Fresh food likely quite welcome.
%cite 90 coy 12 Jul

% Moreover, important work continues.

% Note the shortage of messtins and KFS at hospitals and the chronic shortage
% of tires.  Requirement to return amn casings, empty ammo cans, etc.
% 27 Bde Jul orders p37 14 July

% note how issuance of rations, units were overdrawing, and fresh rats were
% scarce.  Officers sent round to audit.  Talk about how this isn't just about
% REMF harassing front line troops, but a necessary part of military admin.
%cite 27 Bde Jul Os p38 14 July

% TRANSITION

\section{Goodwood (18-20 Jul 44)}
% ARGUMENT:  Use Goodwood to really start to show the centrality of logistics
% to ops.  No food, no amn, no POL, no fighting. (Goodwood detailed)
% Argue that trucks, not just tanks are what it means for an army to be mobile.

% STRUCTURE:  go day by day in the Run up and keep very chrono-narrative.  
% each day seems to sort nicely into themes; thus, use those themes and
% talk about them

% Point of operation  How do i want to manage this transition?  This is
% too abrupt

% THEME FOR 14TH:  Battle sustainment
% WngO (formal or informal) likely received for Goodwood 14th late morning or
% early afternoon.  90 Coy spends the afternoon organizing, then
% 14 Jul, over five hours (1700-2200), 90 Coy 
% converts the 13/18 rgmt amn & POL dumps IVO Ranville into dumps fit to 
% supply the Bde (likely approx 2 sq km in size).  
% dump consists of 38 lorry loads of amn to the 13/18th Rgmt dump enough
% to provide 97 rds/tk (a full load of amn) for the whole Bde.
% POL:  18 lorry loads of Pet & Derv (diesel) (7000 (either units or gal) of
% each).  This is enough POL to take the Bde 30 mi.  Coy does this via the BAD
% and PD.  Vehs fuelled as well before return to Coy HQ by 2400.  Speculate this
% was in receipt of WngO On 15th at 1900, Coy formally 
% establishes a det at the Ranville dumps to control R&I.  Det consists of 
% Capt Duffus, 46 ORs and 16 lorries.  Useful time to talk about fighting
% range and what it means.
%cite 90 Coy WD

% By the 15th, preparations for Goodwood occur in earnest across the Bde.  
% 3 Div holds another conference and, over the next few days, the Bde starts
% moving E of Orne likely via Ranville.  %cite 27th WD 15th Jul-ish
% Ranville det loads load 4 days compo into 6 lorries
% (enough to sustain whole Bde) as a mobile reserve.  8 lorries are devoted to
% keeping croc flamethrowers amned.  2 lorries leftover for odds and ends.
% Useful time to talk about importance of flexibility maybe?  
%cite 90 coy WD

% 16th:  RATIONS
% Bde HQ moves closer to Ranville but stays W of Orne.  Bde Cmdr holds 
% conference with Bde COs. %cite 27th WD
% As a result of this, all units W of Orne received 3 days extra compo, E of
% Orne, 4 days. %cite 90 WD
% Likely, whilst not stated in the WD, that 90 Coy made these regular 
% deliveries to the battalions.

% Total rations thus 4 days in vehicles, and 4 at 2nd line transport.  Ready 
% to move for 8 days.  Therefore, maximum planned speed for brigade's advance
% is approximately 30 mi (let's be realistic, likely 20 mi to account for 
% general movements) over 8 days by accounting for POL.  A good day of 
% advance would be 3.75 mi (6km)/day.  


% 17th:  Comms
% 27th HQ issues it's OpO for Goodwood today.  The Bde will help hold 8 Corps'
% left flank as 8 Corps breaks through.  Bde establishes it's battle net today.
%cite 27th WD and Orders
% 90 Coy receives two wireless lorries, one at Coy HQ, and 1 at the Ranville
% det.  Each lorry comes with 3 wireless ops.  %cite 90th WD
% Curiously, the OpO hardly mentions logistics.  Log is just expected to work.

% 18th:  D-day H-hr 0745
% Bde attack successful.  Objectives met around 1100.  At 2000 Coy HQ sends
% 3 lorry loads of PET and 3 of DERV (figure out range from 
% 18 loads = 30 mi) to Det Ranville. They're stranded, traffic is heavy,
% only E bound traffic permitted over the bridges.  %cite 27/90 WDs
% Start talking about what sustainment looks like.  If there are 6 lorries
% prepared, surely they'll be needed.

D-Day  arrived on 18 July.  The Bde was to support the 3rd Br Inf Div.  
Attached to, and supported by the Bde was the Churchill Chrochodiles of B Sqn
141 RAC, the Sherman Flails of B Sqn Lothian and Border Horse, and the AVREs
of 77 Assualt Sqn RE.  Together, they would advance and hold the British left
flank.  To their right, 8 Corps would punch through German lines, travel South
with Caen to their right, and advance to the high ground South of Caen.  

H-Hour was preceeded with a large RAF bombardment consisting of `over 8000 tons 
of bombs' mostly dropped in the vicinity of Caen though, three of those bombs
were accidentally dropped near 90 Coy at 0715 --- fortunately no-one was hurt. 
Then, when H-Hour arrived at 0745, 3 Div's 8th Bde, supported by the 13th/18th 
Hussars advanced 3km along a narrow front.  By 1100 hrs, they gained most of 
their objectives and halted.  The Staffs continued the avance Southwards to the
line of the Caen-Toran railway line, whilst ERY and 9 Bde turned left and began
a slow crawl East towards Toran.  

The fighting continued until around 1800 when British Forces halted for the
night.  Most of the Bde was withdrawn to unit harbours where ammunition and 
POL were waiting for them.  It is likely that water was also made availabe to
the troops but, owing to the pre-issue, rations were already available at the
units.  

Haivng issued a day of fuel and ammunition to the Regiments, Det Ranville 
itself would need to be resupplied.  As such, 6 3-ton lorries was to be
sent from Coy HQ to Det Ranville loaded with petrol and diesil.  Traffic
however was very heavy that evening so Cricket and Rugger bridges were assigned
to be one-way, eastwards only bridges.  The lorries were thus stranded over the 
river until the restrictions were lifted.  I suspect however that Capt Duffus 
would have ensured the 6 lorries and their drivers did not go idle.  Over the
course of the operation and the days preceeding, the dump's holdings were 
expanded from 38 lorry loads of ammunition, to over 45 lorry loads of 
ammunition --- this despite units drawing ammunition for the dump.  Thus, it
is doubtless that Capt Duffus would have appreciated 6 additional lorries and 
men to help keep the dump organized.

% 19th:  DE send 8 JU88 to attack Orne bridges.  Bde makes minor gains.
% ditto 20th.  Little change happens at this point as the front stabilizes.
%cite 27th WD
% 90 Coy uses this as a time to resupply units from Det Ranville.  
% talk about how ammunition moves.  The Ranville dump goes from 38 lorry 
% loads to 45 loads of amn when the dump closes and the amn is brought back
% to BAD Hermanville. %cite 90th WD for 25 Jul

Operations resumed early next morning with revillie soudning at 0430.  19 July
saw the front beginning to stabalize as the Germans contained the attack.  
19 July saw some hard fighting but it did not result in all units being 
deployed.  Some units spent the whole day in harbours.  As such, the material
requirements of the Bde were lighter.  Nevertheless, that evening, Det 
Ranville continued to issue ammunition and POL to the Bde to meet demands.  
Operations wound down in earnest by the 20th.  That afternoon, around 1600, the
heavens opened up with `a tremendous thunderstorm' turning `every road [into] a
river and every field a bog'.  Tracks begame flooded and nearly impassible as
`the ground [was] feet deep in mud and all slit trenches flooded'.

The Bde was thus immpbilised by the 36 hours of rain.  The men sitting in their
trenches were likely very wet, and very cold. Luckely for the men, Det Ranville 
had some rum available and, on Bde HQ's instructions, issued rum to the entire 
Bde to help keep them warm.  This was repeated on the 21st when the rain simply
refused to stop.  This is enough rum for %number 
thousand men!  Thank god for the prepared logistician! 


% If Goodwood failed to break out, it wasn't logistical constraints.
\section{Post Goodwood}

% use it as a chance to talk about infantry transport --- Br infantry don't 
% have organic transport.  Also how more normal RASC transport units work

% 22nd, the Bde is informed they'll be broken up.  They get assigned to 
% 2nd Army's 22 Transport Coln at months end.  
% Det Ranville shrinks by 45 lorry loads of amn
% and the coy moves to Camilly (GR 933704).  Det Ranville is handed over to 
% different unit and rejoins the Coy.  Coy provides transport to Staff Yeo
% to Arromaches docks.  Coy returns the 2nd line amn holding of 30x 3ton 
% truckloads and 27 6-ton loads to BAD.  %cite 90 WD p.7 mention this holding
% earlier on too.

Alas, Goodwood would be the last operation to be fought by the Bde.  The 
British Army of 1944 was a `wasting army' that was once more reorganizing to
make up casulaties.  The British simply could not replace casulaties and so,
units were reoganized to bring under manned units to combat strenght. %cite
The Bde's armoured units would be assigned to other units or sent back to the
UK whilst the Bde's administrative units would be assigned to whatever unit 
most needed them throughout 21 Army Group.

Still, being broken up did not mean a ceassation of work for 90 Coy.   On 25 
July, 90 Coy was placed under Deputy Director Supply and Transport (DDST) 
Second Army though, for the time being, 90 Coy would still be responsible for 
the usual supply runs for the Bde's units.  Det Ranville however would be 
drawn down and transferred to 33 Armd Bde.

The drawing down of the Ranville Dump however would take a significant amount
of work.  What was originally 38 lorry loads had greatly expanded.  33 Armd 
Bde simply did not need so much ammunition.  As such, before the handover, 90 
Coy lifted 45 3-Ton lorry loads of ammunition from Det Ranville to BAD 
Hermanville.  The remaining ammunition at Ranville was handed over to 33 Armd 
Bde.  27 Armd Bde however still had more ammunition.  On top of Det Ranville, 
there were also the Bde's second line holdings consisting of 57 lorry loads of
ammunition (30 3-Ton loads, and 27 6-Ton loads).  
All this was returned to 12 BAD on 28 July.  

On 31 July, 90 Coy RASC was transfered to Commander RASC, 22 Transport Column
(22 Tpt Coln).  22 Tpt Coln would see 90 Coy's role change.  Whereas 27 Armd
Bde was an Armoured unit and thus, mostly had organic transport capabilities,
22 Tpt Coln mostly serviced infantry units who, in the Second World War, 
generally did not come with organic transport.  As such, supply and transport
companies like 90 Coy were responsible not only for supplying the Infantry 
Battalions, but also for ferrying them about in the back of their lorries.  
Over the next two weeks with the closure of the Falaise pocket, it would be
by riding in the back of 90 Coy's 3-ton and 6-ton lorries that the infantry
would speed their way forward.  Without units like 90 Coy, the 15 km closure
of the Falaise pocket would have had to be done at a walking pace.

\section{} %not sure what to call it, something about being an ordinary
% 2nd line RASC unit


\section{Criticality of Supply}

% Ask the question of what would happen if not for these activities. 
	% Amn
	% Beans
	% Water (note location of WP)
	% POL
	% Everything else (white paint for turrets lest one gets shot lol)

% Invite community to see an army as a system and merely disconnected fighting
% groups.

% Discuss my gaps, I barely touch on maintenance units, etc.  What I could
% have talked about, what other sources say.  Consider Tiger or Panther without
% Maint units

% talk about sources, how 90 Coy WD doesn't really mention resupplies unless
% it's major, but just because it's not written, doesn't mean it wasn't done.
% by asking the question of who would do X, one starts to uncover the Y, and Z.
% talk about how entries like 6 Jun are very long --- likely because of the
% initial excitement.  The WD's progressively shorter and more concise as time
% wares on, likely because it's tedious and, "I want to go to bed!".  Supply
% logs are tedious.  Talk about how this study is actually quite constrained
% by sources in so far as a heavy dependence on electronic sources.  War diaries
% don't mention routine replenishments but it doesn't mean they don't happen.
% alas, I doubt they kept the waybills!


\section{Conclusion}

\newpage

\section{On the Use of Artificial Intelligence (AI) and Machine Learning 
(ML) Tools}

I have used AI/ML tools in the writing of this MRP.  This MRP was typeset
using the Latex type setter and citations were resolved using Biblatex.  As
these are markup languages, they require a very specific syntax documented in
a wide variety of manuals.  When writing the Biblatex bibliography files, 
ChatGPT was used to find the appropriate parameters to tag bibliographic 
information to ensure Biblatex could correctly typeset that bibliographic 
entry.  I could not find the correct methods in the manual and 
trial-and-error is time consuming.  This was done with the knowledge of my
supervisor.

I have also used Apple's Vision Application Programming Interface (API) and
ocrit --- a simple Swift open source program using the Vision API I found on 
Github --- to aid my research.  As all my primary sources had been digitized, 
I found it convenient to OCR all primary sources to simplify the finding of 
information I had read in previous.  I found that ML OCR engines such as 
Apple's Vision API are simply far more accurate than their conventionally
programmed counterparts.  Apple's API could be integrated into my workflow
quickly and at least cost; thus, Apple's API was chosen.

It is likely that the Google search engine used regularly in the course of
my research used AI/ML and this admittedly may have impacted some of my 
sources however, I took care to use other catalogues and databases to 
minimize the effects this risk.  Moreover, I am aware that MacOS now more
heavily integrates AI/ML and neural networks in the OS.  Notwithstanding that
which is written here however, I have not knowingly used AI/ML tools in the
conduct of this research, and the work is my own.

\newpage

\printbibliography

\end{document}
